\documentclass[12pt, titlepage]{article}

\usepackage{booktabs}
\usepackage{tabularx}
\usepackage{hyperref}
\hypersetup{
    colorlinks,
    citecolor=black,
    filecolor=black,
    linkcolor=red,
    urlcolor=blue
}
\usepackage[round]{natbib}
\usepackage{pdflscape}
\usepackage{longtable}

\input{../Comments}
%% Common Parts

\newcommand{\progname}{Software Engineering} % PUT YOUR PROGRAM NAME HERE
\newcommand{\authname}{Team \#2, Team Name
\\ Zihao Du 
\\ Matthew Miller
\\ Firas Elayan
\\ Abhiram Neelamraju
\\ Michael Kim} % AUTHOR NAMES                  

\usepackage{hyperref}
    \hypersetup{colorlinks=true, linkcolor=blue, citecolor=blue, filecolor=blue,
                urlcolor=blue, unicode=false}
    \urlstyle{same}
                                


\begin{document}

\title{Verification and Validation Report: \progname} 
\author{\authname}
\date{\today}
	
\maketitle

\pagenumbering{roman}

\section{Revision History}

\begin{tabularx}{\textwidth}{p{3cm}p{2cm}X}
\toprule {\bf Date} & {\bf Version} & {\bf Notes}\\
\midrule
Mar 4 & 1.0 & Add functional requirements evaluation\\
\bottomrule
\end{tabularx}

~\newpage

\section{Symbols, Abbreviations and Acronyms}

\renewcommand{\arraystretch}{1.2}
\begin{tabular}{l l} 
  \toprule		
  \textbf{symbol} & \textbf{description}\\
  \midrule 
  JMeter & Load testing tool for analyzing and measuring the performance\\
  \midrule 
  SRS & Software Requirements Specification\\
  \midrule 
  UI & User Interface\\
  \midrule 
  VnV & Verification and Validation\\
  \bottomrule
\end{tabular}\\

\newpage

\tableofcontents

\listoftables %if appropriate

\listoffigures %if appropriate

\newpage

\pagenumbering{arabic}

This document describes the test results of the verification and validation (VnV) plan for CampusConnections. The VnV plan was continuously updated as the project evolved. The following document records the results of the current version of the VnV plan. It provides results of functional and non-functional requirements tests, unit tests, changes that will be implemented in the system as a result of the tests, and various traceability tables.


\section{Functional Requirements Evaluation}
The following section outlines the results of functional requirements testing. The process and test performed follow the \href{https://github.com/beatlepie/4G06CapstoneProjectTeam2/blob/main/docs/VnVPlan/VnVPlan.pdf}{VnV Plan}. To summarize, all the tests are tested manually and passed, indicating that all the functional requirements in the Software Requirements Specification (SRS) document are covered.

\subsection{Pre-Registration Settings}
This section covers all tests related to functional requirements about pre-registration settings.
\begin{enumerate}
\item \textbf{FRT-PR1}

\textbf{Name:} Agree To Consent Form

\textbf{Initial State:} The user does not have an account, and they starts to register an account. A consent form appears asking for access to the device and permission to collect user data

\textbf{Input:} The user agrees to all the terms and conditions and clicks `Agree' and continues to complete the registration process
					
\textbf{Expected Output:} A notification shows the registration succeeds and the user is redirected to the login screen

\textbf{Actual Output:} A notification shows the registration succeeds and the user is redirected to the login screen

\textbf{Results:} Pass

\item \textbf{FRT-PR2}

\textbf{Name:} Disagree To Consent Form

\textbf{Initial State:} The user does not have an account, and they starts to register an account. A consent form appears asking for access to the device and permission to collect user data
					
\textbf{Input:} The user rejects the terms and conditions and clicks `Disagree' and continues to complete the registration process
					
\textbf{Expected Output:} The registration fails and a warning will show up notifying the user that they cannot create an account unless they agree to the consent form

\textbf{Actual Output:} The registration fails and a warning will show up notifying the user that they cannot create an account unless they agree to the consent form

\textbf{Results:} Pass
\end{enumerate}
\subsection{User Account}
This section covers all tests related to functional requirements about the account and user profile.
\begin{enumerate}
\item \textbf{FRT-UA1}

\textbf{Name:} Successful Account Creation

\textbf{Initial State:} The user does not have an account and is not logged in to the application

\textbf{Input:} All information needed to create an account:
\begin{itemize}
\item Email: testUA1@gmail.com
\item password: FRT-UA1
\item nickname: UA1
\end{itemize}
					
\textbf{Expected Output:} An Account with corresponding information is created in the database with the account initialized to INITIAL\_USER\_STATE

\textbf{Actual Output:} An Account is created in the database with the following attributes and other attributes are initialized to INITIAL\_USER\_STATE:
\begin{itemize}
\item Email: testUA1@gmail.com
\item password: FRT-UA1
\item nickname: UA1
\end{itemize}

\textbf{Results:} Pass

\item \textbf{FRT-UA2}

\textbf{Name:} Unsuccessful Account Creation

\textbf{Initial State:} The user does not have an account and is not logged in to the application

\textbf{Input:} All information needed to create an account:
\begin{itemize}
\item Email: qtest@gmail.com (this is an existing test account)
\item password: FRT-UA1
\item nickname: UA1
\end{itemize}
					
\textbf{Expected Output:} Account creation fails with a warning telling the user the email has already been used

\textbf{Actual Output:} Account creation fails with a warning telling the user the email has already been used

\textbf{Results:} Pass

\item \textbf{FRT-UA3}

\textbf{Name:} Successful Account Login

\textbf{Initial State:} The user has an account and is not logged in to the application

\textbf{Input:} All information needed to login:
\begin{itemize}
\item Email: FRT-UA3@test.com (this account exists in the system already)
\item password: FRT-UA3
\end{itemize}
					
\textbf{Expected Output:} User successfully logs into the application and goes to the menu page

\textbf{Actual Output:} User successfully logs into the application and goes to the menu page

\textbf{Results:} Pass

\item \textbf{FRT-UA4}

\textbf{Name:} Unsuccessful Account Login

\textbf{Initial State:} The user has an account and is not logged in to the application

\textbf{Input:} All information needed to login:
\begin{itemize}
\item Email: FRT-UA3@test.com (this account exists in the system already)
\item password: FRT321 (wrong password)
\end{itemize}
					
\textbf{Expected Output:} Login fails with a warning telling the user the password is wrong

\textbf{Actual Output:} Login fails with a warning telling the user the password is wrong

\textbf{Results:} Pass

\item \textbf{FRT-UA5}

\textbf{Name:} Account Deletion

\textbf{Initial State:} The user has an account and is logged into the application
\begin{itemize}
\item Email: FRT-UA5@gmail.com (this is an existing test account)
\item password: FRT-UA5
\item nickname: UA5
\end{itemize}

\textbf{Input:} User clicks on the delete account button on the profile page and confirms the deletion
					
\textbf{Expected Output:} The user is redirected to the login page and the account cannot be logged in any more

\textbf{Actual Output:} The user is redirected to the login page and the account FRT-UA5@gmail.com cannot be logged in any more

\textbf{Results:} Pass

\item \textbf{FRT-UA6}

\textbf{Name:} Reset Password

\textbf{Initial State:} The user has an account:
\begin{itemize}
\item Email: campusconnections@gmail.com (this is an existing test account)
\item password: qtesting
\end{itemize}

\textbf{Input:} Email address and new password
\begin{itemize}
\item new password: QTesting
\end{itemize}
					
\textbf{Expected Output:} Password is successfully reset

\textbf{Actual Output:} Password is successfully reset to be QTesting

\textbf{Results:} Pass

\item \textbf{FRT-UA7}

\textbf{Name:} Avatar Creation and Modification

\textbf{Initial State:} The user has an account with DEFAULT\_AVATAR

\textbf{Input:} URI represents the new avatar:
\begin{itemize}
\item URI: https://upload.wikimedia.org/wikipedia/commons/2/2f/Google\_2015\_logo.svg
\end{itemize}
					
\textbf{Expected Output:} The user changes the avatar to a Google logo

\textbf{Actual Output:} The user changes the avatar to a Google logo

\textbf{Results:} Pass

\item \textbf{FRT-UA8}

\textbf{Name:} Email Verification

\textbf{Initial State:} The user has an account whose email has not been verified yet
\begin{itemize}
\item Email: fuz15@mcmaster.ca (this is an existing test account)
\item password: password
\end{itemize}

\textbf{Input:} User clicks on `Verify Email` button on user profile page and follows instructions on the email sent from the system
					
\textbf{Expected Output:} That email above is verified as a valid email address

\textbf{Actual Output:} fuz15@mcmaster.ca is verified as a valid email address in the system

\textbf{Results:} Pass

\item \textbf{FRT-UA9}

\textbf{Name:} Edit Profile

\textbf{Initial State:} The user has an account 
\begin{itemize}
\item Email: qtest@gmail.com (this is an existing test account)
\item password: qtesting
\item newProgram: Computer Science
\end{itemize}

\textbf{Input:} New Profile:
\begin{itemize}
\item newProgram: Computer Science
\item newLevel: 4
\end{itemize}
					
\textbf{Expected Output:} The program and level are updated

\textbf{Actual Output:} The program and level are updated to be Computer Science and 4

\textbf{Results:} Pass
\end{enumerate}
\subsection{Social Networking System}
This section covers all tests related to functional requirements about interactions between friends.
\begin{enumerate}
\item \textbf{FRT-SN1}

\textbf{Name:} Successful Friend Request

\textbf{Initial State:} The user is logged in with the following account:
\begin{itemize}
\item Semail: FRT-SN1@test.com
\item password: testing
\end{itemize}

\textbf{Input:} A valid email to send the request:
\begin{itemize}
\item Temail: FRT-SN1-F@test.com
\end{itemize}
					
\textbf{Expected Output:} A Request is sent to the target user

\textbf{Actual Output:} A Request is sent to the target user

\textbf{Results:} Pass

\item \textbf{FRT-SN2}

\textbf{Name:} Friend Request Acceptance

\textbf{Initial State:} A friend request was sent from an account (Semail) to the target account (Temail):
\begin{itemize}
\item Semail: FRT-SN1@test.com
\item Temail: FRT-SN1-F@test.com
\end{itemize}

\textbf{Input:} The request is accepted
					
\textbf{Expected Output:} Two users are added to each other's friend lists

\textbf{Actual Output:} Two users are added to each other's friend lists

\textbf{Results:} Pass

\item \textbf{FRT-SN3}

\textbf{Name:} Successful Friend Rejection

\textbf{Initial State:} A friend request was sent from an account (Semail) to the target account (Temail):
\begin{itemize}
\item Semail: FRT-SN1@test.com
\item Temail: FRT-SN1-F@test.com
\end{itemize}

\textbf{Input:} The request is rejected
					
\textbf{Expected Output:} The request is declined and no friend is added for both accounts

\textbf{Actual Output:} The request is declined and no friend is added for both accounts

\textbf{Results:} Pass

\item \textbf{FRT-SN4}

\textbf{Name:} Friend Deletion

\textbf{Initial State:} A friend (Femail) exist in the friend list of the test account (Temail):
\begin{itemize}
\item Temail: FRT-SN4@test.com
\item Femail: FRT-SN4-F@test.com
\end{itemize}

\textbf{Input:} User deletes the chosen friend
					
\textbf{Expected Output:} The corresponding friend is deleted from the list

\textbf{Actual Output:} The corresponding friend FRT-SN4-F@test.com is deleted from the friend list

\textbf{Results:} Pass

\item \textbf{FRT-SN5}

\textbf{Name:} Friend Messaging

\textbf{Initial State:} A friend (Femail) exist in the friend list of the test account (Temail):
\begin{itemize}
\item Temail: FRT-SN5@test.com
\item Femail: FRT-SN5-F@test.com
\end{itemize}

\textbf{Input:} Message: `Hello World'
					
\textbf{Expected Output:} The corresponding message is sent to the friend

\textbf{Actual Output:} The corresponding message is sent to the friend

\textbf{Results:} Pass

\item \textbf{FRT-SN6}

\textbf{Name:} Friend Sharing Event

\textbf{Initial State:} A friend (Femail) exist in the friend list of the test account (Temail):
\begin{itemize}
\item Temail: FRT-SN6@test.com
\item Femail: FRT-SN6-F@test.com
\end{itemize}

\textbf{Input:} Message that contains event name and follows some specific pattern:
Hey, check this event: \textunderscore E\textunderscore[EXPO]
					
\textbf{Expected Output:} User is redirected to the event page with that event once they click on the message

\textbf{Actual Output:} User is redirected to the event page with a filter on event name: EXPO

\textbf{Results:} Pass

\item \textbf{FRT-SN7}

\textbf{Name:} Friend Sharing Lecture

\textbf{Initial State:} A friend (Femail) exist in the friend list of the test account (Temail):
\begin{itemize}
\item Temail: FRT-SN7@test.com
\item Femail: FRT-SN7-F@test.com
\end{itemize}

\textbf{Input:} Message that contains lecture code and follows some specific pattern:
`Hey, are you in this lecture: \textunderscore L\textunderscore[SFRWENG 4G06]'
					
\textbf{Expected Output:} User is redirected to the lecture page with that lecture once they click on the message

\textbf{Actual Output:} User is redirected to the lecture page with a filter on lecture code: SFRWENG 4G06

\textbf{Results:} Pass
\end{enumerate}
\subsection{Lectures and Events}
This section covers all tests related to functional requirements about lectures and events and how users can interact with them.
\begin{enumerate}
\item \textbf{FRT-LE1}

\textbf{Name:} Save Event

\textbf{Initial State:} A sample event:
\begin{itemize}
\item Name: EXPO
\end{itemize}

\textbf{Input:} On the event page, user clicks on the save button on the pop-up window with details of the sample event
					
\textbf{Expected Output:} The event is saved to the user's event list

\textbf{Actual Output:} The event EXPO is saved to the user's event list

\textbf{Results:} Pass

\item \textbf{FRT-LE2}

\textbf{Name:} Unsave Event

\textbf{Initial State:} A sample event that is already been saved:
\begin{itemize}
\item Name: EXPO
\end{itemize}

\textbf{Input:} On the event page, user clicks on the unsave button on the pop-up window with details of the sample event
					
\textbf{Expected Output:} The event is removed from the user's event list

\textbf{Actual Output:} The event EXPO is removed from the user's event list

\textbf{Results:} Pass

\item \textbf{FRT-LE3}

\textbf{Name:} Save Lecture

\textbf{Initial State:} A sample lecture:
\begin{itemize}
\item Code: SFWRENG 4G06
\end{itemize}

\textbf{Input:} On the lecture page, user clicks on the save button on the pop-up window with details of the sample lecture
					
\textbf{Expected Output:} The lecture is saved to the user's lecture list

\textbf{Actual Output:} The lecture SFWRENG 4G06 is saved to the user's lecture list

\textbf{Results:} Pass

\item \textbf{FRT-LE4}

\textbf{Name:} Unsave Lecture

\textbf{Initial State:} A sample lecture that is already been saved:
\begin{itemize}
\item Code: SFWRENG 4G06
\end{itemize}

\textbf{Input:} On the lecture page, user clicks on the unsave button on the pop-up window with details of the sample lecture
					
\textbf{Expected Output:} The lecture is removed from the user's lecture list

\textbf{Actual Output:} The lecture SFWRENG 4G06 is removed from the user's lecture list

\textbf{Results:} Pass

\item \textbf{FRT-LE5}

\textbf{Name:} Administrator Add Event

\textbf{Initial State:} User is logged in as an administrator
\begin{itemize}
\item email: campusconnections@gmail.com
\item password: testing
\end{itemize}

\textbf{Input:} Sample event:
\begin{itemize}
\item name: Test event
\item description: Sample event for system test
\item time: 0
\item duration: 0
\item location: Online
\item isPublic: true
\item organizer: Team 2
\end{itemize}
					
\textbf{Expcted Output:} The event is added to the event list

\textbf{Actual Output:} The event Test event is added to the event list

\textbf{Results:} Pass

\item \textbf{FRT-LE6}

\textbf{Name:} Administrator Edit Event

\textbf{Initial State:} User is logged in as an administrator
\begin{itemize}
\item email: campusconnections@gmail.com
\item password: testing
\end{itemize}

\textbf{Input:} Sample event name and new location:
\begin{itemize}
\item name: Test event
\item location: ITB AB102
\end{itemize}
					
\textbf{Expected Output:} The test event location is updated to the new one

\textbf{Actual Output:} The test event location is updated toITB AB102

\textbf{Results:} Pass

\item \textbf{FRT-LE7}

\textbf{Name:} Administrator Delete Event

\textbf{Initial State:} User is logged in as an administrator
\begin{itemize}
\item email: campusconnections@gmail.com
\item password: testing
\end{itemize}

\textbf{Input:} Sample event (already in the system) name:
\begin{itemize}
\item name: Test event
\end{itemize}
					
\textbf{Expected Output:} The event is deleted and disappears from the list

\textbf{Actual Output:} The Test event is deleted and disappears from the list

\textbf{Results:} Pass

\item \textbf{FRT-LE8}

\textbf{Name:} Administrator Add Lecture

\textbf{Initial State:} User is logged in as an administrator
\begin{itemize}
\item email: campusconnections@gmail.com
\item password: testing
\end{itemize}

\textbf{Input:} Sample lecture:
\begin{itemize}
\item code: TEST 1T03
\item name: Test lecture
\item time: 12:00 - 13:00, Mon
\item location: Online
\item instructor: NA
\end{itemize}
					
\textbf{Expected Output:} The lecture is added to the lecture list

\textbf{Actual Output:} The lecture TEST 1T03 is added to the lecture list

\textbf{Results:} Pass

\item \textbf{FRT-LE9}

\textbf{Name:} Administrator Edit Lecture

\textbf{Initial State:} User is logged in as an administrator
\begin{itemize}
\item email: campusconnections@gmail.com
\item password: testing
\end{itemize}

\textbf{Input:} Sample lecture name and new location:
\begin{itemize}
\item code: TEST 1T03
\item location: ITB AB102
\end{itemize}
					
\textbf{Expected Output:} The test lecture location is updated to the new one

\textbf{Actual Output:} The test lecture location is updated to ITB AB102

\textbf{Results:} Pass

\item \textbf{FRT-LE10}

\textbf{Name:} Administrator Delete Lecture

\textbf{Initial State:} User is logged in as an administrator
\begin{itemize}
\item email: campusconnections@gmail.com
\item password: testing
\end{itemize}

\textbf{Input:} Sample lecture (already in the system) name:
\begin{itemize}
\item code: TEST 1T03
\end{itemize}
					
\textbf{Expected Output:} The lecture is deleted and disappears from the list

\textbf{Actual Output:} The test lecture is deleted and disappears from the list

\textbf{Results:} Pass

\item \textbf{FRT-LE11}

\textbf{Name:} Event Information

\textbf{Initial State:} A sample event exists:
\begin{itemize}
\item name: EXPO
\end{itemize}

\textbf{Input:} User clicks on the sample event
					
\textbf{Expected Output:} All event information is shown in a pop-up window

\textbf{Actual Output:} All event information is shown in a pop-up window:
\begin{itemize}
\item name
\item description
\item location
\item time
\item duration
\item organizer
\item isPublic
\end{itemize}

\textbf{Results:} Pass

\item \textbf{FRT-LE12}

\textbf{Name:} Lecture Information

\textbf{Initial State:} A sample lecture exists:
\begin{itemize}
\item code: SFWRENG 4G06
\end{itemize}

\textbf{Input:} User clicks on the sample lecture
					
\textbf{Expected Output:} All lecture information is shown in a pop-up window

\textbf{Actual Output:} All lecture information is shown in a pop-up window:
\begin{itemize}
\item code
\item name
\item instructor
\item time
\item location
\end{itemize}

\textbf{Results:} Pass

\item \textbf{FRT-LE13}

\textbf{Name:} Lecture Filter by Code

\textbf{Initial State:} Some software engineering lecture exists:
\begin{itemize}
\item SFWRENG 4G06
\item SFWRENG 4E03
\end{itemize}

\textbf{Input:} Filter:
\begin{itemize}
\item FilterString: SFWRENG
\end{itemize}
					
\textbf{Expected Output:} All lectures which do not contain the FilterString in the code are removed from the list

\textbf{Actual Output:} All lectures which do not contain the SFWRENG in the code are removed from the list

\textbf{Results:} Pass

\item \textbf{FRT-LE14}

\textbf{Name:} Event Filter by Name

\textbf{Initial State:} Some job fair event exists:
\begin{itemize}
\item Job Fair: March 4
\item Job Fair: March 10
\end{itemize}

\textbf{Input:} Filter:
\begin{itemize}
\item FilterString: Job Fair
\end{itemize}
					
\textbf{Expected Output:} All events which do not contain the FilterString in the name are removed from the list

\textbf{Actual Output:} All events which do not contain the Job Fair in the name are removed from the list

\textbf{Results:} Pass
\end{enumerate}
\subsection{AR Camera}
This section covers all tests related to functional requirements about AR camera.
\begin{enumerate}
\item \textbf{FRT-AR1}

\textbf{Name:} Successful Building Recognition

\textbf{Initial State:} User is at the front door of JHE

\textbf{Input:} Clear camera view
					
\textbf{Expected Output:} The building is recognized with name and description shown as an AR object

\textbf{Actual Output:} The building is recognized with name and description shown as an AR object

\textbf{Results:} Pass

\item \textbf{FRT-AR2}

\textbf{Name:} Unsuccessful Building Recognition

\textbf{Initial State:} User is out of campus

\textbf{Input:} Clear camera view
					
\textbf{Expected Output:} No AR objects are shown

\textbf{Actual Output:} No AR objects are shown

\textbf{Results:} Pass

\item \textbf{FRT-AR3}

\textbf{Name:} Building Lectures/Events

\textbf{Initial State:} User is in JHE lobby

\textbf{Input:} Clear camera view
					
\textbf{Expected Output:} Event and lecture information separated by room number at the corresponding locations of the building

\textbf{Actual Output:} Event and lecture information separated by room number at the corresponding locations of the building

\textbf{Results:} Pass
\end{enumerate}
\subsection{Map and Location}
This section covers all tests related to functional requirements about the map and location tracking in the system.
\begin{enumerate}
\item \textbf{FRT-MAP1}

\textbf{Name:} User Location

\textbf{Initial State:} User allows the user to use their real-time location

\textbf{Input:} User enters the map page
					
\textbf{Expected Output:} A model representing the user shows up on the map and moves correspondingly when the user moves

\textbf{Actual Output:} A model representing the user shows up on the map and moves correspondingly when the user moves

\textbf{Results:} Pass

\item \textbf{FRT-MAP2}

\textbf{Name:} Friend Locations

\textbf{Initial State:} User has some friends who are willing to share locations:
\begin{itemize}
\item email1: MAP2-1@test.com
\item email2: MAP2-2@test.com
\end{itemize}

\textbf{Input:} User enters the map page
					
\textbf{Expected Output:} Additional models representing friends show up on the map and move correspondingly when friends move

\textbf{Actual Output:} Additional models representing friends show up on the map and move correspondingly when friends move

\textbf{Results:} Pass
\end{enumerate}

\section{Nonfunctional Requirements Evaluation}
The following section outlines the results of non-functional requirements testing. The process and test performed follow the \href{https://github.com/beatlepie/4G06CapstoneProjectTeam2/blob/main/docs/VnVPlan/VnVPlan.pdf}{VnV Plan}. Most of the tests are tested manually while some of them are tested in another way, for instance, load testing is tested with JMeter while some UI requirements are tested by conducting a survey, etc. \textcolor{red}{Some of the tests fail because their related requirements are removed due to the change of the project's scope, these tests will be marked in red.} In general, most of the tests in the plan succeed, indicating that non-functional requirements in the Software Requirements Specification (SRS) document are covered.
\subsection{Look and feel}
This section corresponds to the Look and feel tests in VnV Plan and Look and Feel requirements in SRS.
\begin{enumerate}
\item \textbf{NFRT-LF1}

\textbf{Name:} Survey for feedback on application layout

\textbf{Initial State:} Survey taker is given an account:
\begin{itemize}
\item email: mtest@gmail.com
\item password: mtesting
\end{itemize}

\textbf{Input/Condition:} Usability Survey in section \ref{survey}

\textbf{Expected Output:} Tasks are completed successfully and ``Immediate Visual Response when Clicking'' and ``Appealing Colour Scheme'' questions get average scores that are great than MIN\_SCORE

\textbf{Actual Output:} Tasks are completed successfully and ``Immediate Visual Response when Clicking'' and ``Appealing Colour Scheme'' questions get average scores of 3.8 and 4.8 respectively

\textbf{Result:} Pass

\item \textbf{NFRT-LF2}

\textbf{Name:} Visual inspection with different screen sizes

\textbf{Input/Condition:} User opens the application on the phone with all different screen sizes in the SCREEN\_VIEWPORTS list

\textbf{Expected Output:} For all different pages all visual elements on the screen are within the borders of the screen for all screens in the SCREEN\_VIEWPORTS list

\textbf{Actual Output:} All elements are within the borders of the screen without overlapping for all screen sizes in the SCREEN\_VIEWPORTS list

\textbf{Result:} Pass

\item \textbf{NFRT-LF3}

\textbf{Name:} Visual inspection of color scheme

\textbf{Initial State:} NA
					
\textbf{Input:} User opens the application on the phone

\textbf{Expected Output:} For all different pages the colour scheme is the same

\textbf{Actual Output:} For all different pages the colour of elements and texts is always maroon, gold, black and white

\textbf{Result:} Pass
\end{enumerate}
\subsection{Usability and Humanity}
This section corresponds to the Usability and Humanity tests in VnV Plan and Usability and Humanity requirements in SRS.
\begin{enumerate}
\item \textbf{NFRT-UH1}

\textbf{Name:} Survey for feedback on understandability and easy of use

\textbf{Initial State:} Survey taker is given an account:
\begin{itemize}
\item email: mtest@gmail.com
\item password: mtesting
\end{itemize}

\textbf{Input:} Usability Survey in section \ref{survey}

\textbf{Expected Output:} Tasks are completed successfully and ``No Technical or Software-Specific Language'' question gets an average score that is great than MIN\_SCORE

\textbf{Actual Output:} Tasks are completed successfully and ``No Technical or Software-Specific Language'' question gets an average score of 4

\textbf{Result:} Pass

\item \textbf{NFRT-UH2}

\textbf{Name:} Walkthrough of user guide

\textbf{Input:} GitHub web page (see details in section \ref{static})
					
\textbf{Expected Output:} Convinced the participants that the main features are explained in the GitHub repo

\textbf{Actual Output:} Convinced the participants that the main features are explained in the GitHub repo

\textbf{Result:} Pass

\item \textbf{NFRT-UH3}

\textbf{Name:} Visual inspection of color contrast

\textbf{Initial State:} NA
					
\textbf{Input:} User opens the application checks the color contrast statically

\textbf{Expected Output:} The color contrast is greater than 4.5:1,  the Web Content Accessibility Guidelines' AA standards for accessibility

\textbf{Actual Output:} The color contrasts are listed following:
\begin{itemize}
\item Maroon - White: 10.94:1
\item White - Black: 21:1
\item Gold - Black: 14.97:1
\item Maroon - Gold: 7.8:1
\end{itemize}

\textbf{Result:} Pass
\end{enumerate}
\subsection{Performance}
This section corresponds to the Performance tests in VnV Plan and Performance requirements in SRS.
\begin{enumerate}
\item \textbf{NFRT-P1}

\textbf{Name:} AR camera recognition

\textbf{Initial State:} The user is near or in a target building (JHE)
					
\textbf{Input:} User turns on AR camera
					
\textbf{Expected Output:} Corresponding AR objects appears within RECOGNITION\_TIME

\textbf{Actual Output:} Corresponding AR objects appears within 1 second

\textbf{Result:} Pass

\item \textbf{NFRT-P2}

\textbf{Name:} Real-time location update

\textbf{Initial State:} User allows the application to use device location
					
\textbf{Input:} User turns on the map and walks around on campus
					
\textbf{Expected Output:} The user model on the map is updated within  LOCATION\_UPDATE\_TIME when the user is moving

\textbf{Actual Output:} The user model is updated within 0.5 second if the user is outdoor, and indoor location update time is around 5 second (and sometimes not very accurate)

\textbf{Result:} Pass

\item \textbf{NFRT-P3}

\textbf{Name:} Code Walkthrough For User Personal Data

\textbf{Input:} Source code (see details in section \ref{static})

\textbf{Expected Output:} Successfully convinced the participants the following:
\begin{itemize}
  \item User's personal information does not appear in the database if the user did not grant permission
\end{itemize}

\textbf{Actual Output:} Successfully convinced the participants that the user personal information is collected under permission only

\textbf{Result:} Pass

\item \textbf{NFRT-P4}

\textbf{Name:} Code Walkthrough For Data Transmission Encryption

\textbf{Input:} Source code (see details in section \ref{static})

\textbf{Expected Output:} Successfully convinced the participants the following:
\begin{itemize}
  \item The product only transmits encrypted data from server to user
\end{itemize}

\textbf{Actual Output:} Successfully convinced the participants that the data send from/to the server are encrypted

\textbf{Result:} Pass

\item \textbf{NFRT-P5}

\textbf{Name:} Warning when starting AR camera

\textbf{Initial State:} User allows the application to use camera
					
\textbf{Input:} User turns on the AR camera
					
\textbf{Expected Output:} A warning telling the user to be aware of their surroundings is displayed upon start-up of the camera

\textbf{Actual Output:} A warning telling the user to be aware of their surroundings is displayed upon start-up of the camera

\textbf{Result:} Pass

\item \textbf{NFRT-P6}

\textbf{Name:} Leaving campus warning

\textbf{Initial State:} User opens the map
					
\textbf{Input:} User moves out of the campus
					
\textbf{Expected Output:} A warning message is displayed telling the user that the map is not available out of campus

\textbf{Actual Output:} A warning message is displayed telling the user that the map is not available out of campus and the user is redirected to the menu page once the warning message is closed

\textbf{Result:} Pass

\item \textbf{NFRT-P7}

\textbf{Name:} Special character warning

\textbf{Initial State:} User starts to register
					
\textbf{Input:} User enters nickname with some special character: `; DELETE *'
					
\textbf{Expected Output:} A warning message is displayed telling the user that the special characters are not allowed and stops the user from registering

\textbf{Actual Output:} A warning message is displayed telling the user that the special characters are not allowed and stops the user from registering

\textbf{Result:} Pass

\item \textbf{NFRT-P8}

\textbf{Name:} Email format warning

\textbf{Initial State:} User starts to register
					
\textbf{Input:} User enters a not email string in the email field:
`SELECT * FROM TABLE'
					
\textbf{Expected Output:} A warning message is displayed telling the user that the input is not an email address and stops the user from registering

\textbf{Actual Output:} A warning message is displayed telling the user that the input is not an email address and stops the user from registering

\textbf{Result:} Pass

\item \textbf{NFRT-P9}

\textbf{Name:} AR camera accuracy

\textbf{Initial State:} User is near or in a target building (JHE)
					
\textbf{Input:} User turns on the AR Camera and repeat multiple times
					
\textbf{Expected Output:} AR objects shows up for at least AR\_ACCURACY * number of tests times

\textbf{Actual Output:} AR objects always shows up when walking around JHE lobby

\textbf{Result:} Pass

\item \textbf{\textcolor{red}{NFRT-P10}}

\textbf{Name:} Warning when internet connection is lost

\textbf{Initial State:} User has no internet connection
					
\textbf{Input:} User opens the application
					
\textbf{Expected Output:} There is a pop-up window telling the user the internet is lost

\textbf{Actual Output:} Nothing

\textbf{Result:} \textcolor{red}{Fail}

\textbf{Reason:} Due to a change in the scope of the project, the corresponding requirement is moved out of the scope, therefore this test fails because the feature is not implemented

\item \textbf{NFRT-P11}

\textbf{Name:} Rudimentary functions when the server connection is lost

\textbf{Initial State:} Server is turned down
					
\textbf{Input:} User opens the application
					
\textbf{Expected Output:} The application still works with limited functionalities

\textbf{Actual Output:} The lecture and event pages still work, the single user map still works, the friend system still works

\textbf{Result:} Pass

\item \textbf{NFRT-P12}

\textbf{Name:} Code inspection for server restart

\textbf{Initial State:} NA

\textbf{Input:} Server settings (see details in section \ref{static})
					
\textbf{Expected Output:} Successfully convinced the participants the following:
\begin{itemize}
\item Server attempts to restart when it crashes
\end{itemize}

\textbf{Actual Output:} Successfully convinced the participants that the server is set to restart once it goes down

\textbf{Result:} Pass

\item \textbf{NFRT-P13}

\textbf{Name:} AR camera help button

\textbf{Initial State:} User turns on the AR camera

\textbf{Input:} User clicks the help button
					
\textbf{Expected Output:} A message telling the user things that may affect AR camera appears

\textbf{Actual Output:} A message telling the user things that may affect AR camera appears

\textbf{Result:} Pass

\item \textbf{NFRT-P14}

\textbf{Name:} Load testing for server

\textbf{Initial State:} The server is online and open to connections

\textbf{Input:} Load testing with JMeter (see details in section \ref{load})

\textbf{Expected Output:} The server is able to handle up to MAX\_CAPACITY users connecting to the server simultaneously

\textbf{Actual Output:} The server is able to handle 1000 connections at the same time

\textbf{Result:} Pass

\item \textbf{NFRT-P15}

\textbf{Name:} Code inspection for database capacity

\textbf{Initial State:} NA

\textbf{Input:} Database documentation (see details in section \ref{static})

\textbf{Expected Output:} Successfully convinced the participants the following:
\begin{itemize}
\item The database has enough space to store all the user, lecture and event information
\end{itemize}

\textbf{Actual Output:} Successfully convinced the participants that the current plan has enough space for expected number of users

\textbf{Result:} Pass

\item \textbf{NFRT-P16}

\textbf{Name:} Code walkthrough for adding new building

\textbf{Initial State:} NA

\textbf{Input:} Source code (see details in section \ref{static})

\textbf{Expected Output:} Successfully convinced the participants the following:
\begin{itemize}
\item A new target building can be added without causing the application running any slower
\end{itemize}

\textbf{Actual Output:} Successfully convinced the participants that adding a new building is just like adding a new scene and will not affect the speed of the application

\textbf{Result:} Pass

\item \textbf{NFRT-P17}

\textbf{Name:} Code Peer Evaluation For Longevity

\textbf{Initial State:} NA

\textbf{Input:} Source code (see details in section \ref{static})

\textbf{Expected Output:} Successfully convinced the participants the following:
\begin{itemize} 
  \item The product is able to operate without major malfunctions in release build for at least 1 year
  \item The finalized product will remain compatible with the promised operating systems and devices for at least 3 years
\end{itemize}

\textbf{Actual Output:} Successfully convinced the participants that the product is able to operate without major malfunctions in release build for at least 1 year and the finalized product will remain compatible with the promised operating systems and devices for at least 3 years

\textbf{Result:} Pass
\end{enumerate}
\subsection{Operational and Environmental}
This section corresponds to the Operational and Environmental tests in VnV Plan and Operational and Environmental requirements in SRS.
\begin{enumerate}
\item \textbf{NFRT-OE1}

\textbf{Name:} Visual inspection for application download

\textbf{Initial State:} User has a phone that uses Android 11 or above

\textbf{Input:} User wants to download the application

\textbf{Expected Output:} The product can be downloaded onto the phone from the Google Play Store, or by downloading the APK file directly

\textbf{Actual Output:} The application can be downloaded from an APK file the team releases

\textbf{Result:} Pass
\end{enumerate}
\subsection{Maintainability and Support}
\begin{enumerate}
\item \textbf{NFRT-MS1}

\textbf{Name:} Survey for maintenance time

\textbf{Initial State:} Survey taker is given an account:
\begin{itemize}
\item email: mtest@gmail.com
\item password: mtesting
\end{itemize}

\textbf{Input/Condition:} Usability Survey in section \ref{survey}

\textbf{Expected Output:} Tasks are completed successfully and ``Common periods of usage'' question gets an average score that is great than MIN\_SCORE

\textbf{Actual Output:} Tasks are completed successfully and ``Common periods of usage'' questions get average scores of 3.2

\textbf{Result:} \textcolor{red}{Fail}

\textbf{Reason:} It seems that students may use the application after school

\item \textbf{NFRT-MS2}

\textbf{Name:} Check for feature request

\textbf{Initial State:} A public GitHub repo exists for this application

\textbf{Input:} User goes to the GitHub repo

\textbf{Expected Output:} User can read issues created by the team and also create new issues

\textbf{Actual Output:} User can read issues created by the team and also create new issues

\textbf{Result:} Pass

\item \textbf{NFRT-MS3}

\textbf{Name:} Android version test

\textbf{Initial State:} NA

\textbf{Input:} The application is installed on devices with Android 11 and above version

\textbf{Expected Output:} The application works without any error about compatibility

\textbf{Actual Output:} The application works without any error about compatibility

\textbf{Result:} Pass
\end{enumerate}
\subsection{Security}
\begin{enumerate}
\item \textbf{NFRT-S1}

\textbf{Name:} Access test

\textbf{Initial State:} Three accounts with different accesses is ready:
\begin{itemize}
\item Admin: campusconnections@gmail.com
\item User: mtest@gmail.com
\item Guest gtest@gmail.com
\end{itemize}
  
\textbf{Input:} The user starts the application with the three accounts

\textbf{Expected Output:} At each level of access, the application constrains the possible actions to what is specified in requirement S-A1, S-A2, S-A3.

\textbf{Actual Output:} Possible Actions:
\begin{itemize}
\item Administrator: Everything, include Add/Edit/Delete actitives
\item User: Friend system, Profile system, lecture and event viewing, map system
\item Guest: Public event viewing, single-user map, Profile system
\end{itemize}

\textbf{Result:} Pass
\end{enumerate}
\subsection{Privacy}
\begin{enumerate}
\item \textbf{NFRT-PRV1}

\textbf{Name:} Code inspection for legitimate use of personal data

\textbf{Initial State:} NA

\textbf{Input:} Source code (see details in section \ref{static})

\textbf{Expected Output:} Successfully convinced the participants the following:
\begin{itemize}
\item The usage of a user's personal information by the product abides by the Privacy Act, The Personal Information Protection and Electronic Documents Act, and Canada and Ontario's data protection laws
\end{itemize}

\textbf{Actual Output:} Convinced the participants that the personal data we collect is handled legitimately

\textbf{Result:} Pass

\item \textbf{NFRT-PRV2}

\textbf{Name:} Code inspection for removing unused accounts

\textbf{Initial State:} NA

\textbf{Input:} Source code (see details in section \ref{static})

\textbf{Expected Output:} Successfully convinced the participants the following:
\begin{itemize}
\item Any accounts that are not active for a long time (a semester) will be removed from the system
\end{itemize}

\textbf{Actual Output:} Convinced the participants that the authentication system and database will clean inactive data periodically

\textbf{Result:} Pass
\end{enumerate}
\subsection{Culture}
\begin{enumerate}
\item \textbf{NFRT-CUL1}

\textbf{Name:} Survey for feedback on cultural requirements

\textbf{Initial State:} Survey taker is given an account:
\begin{itemize}
\item email: mtest@gmail.com
\item password: mtesting
\end{itemize}

\textbf{Input/Condition:} Usability Survey in section \ref{survey}

\textbf{Expected Output:} Tasks are completed successfully and ``Cultural Friendliness'' question gets an average score that is great than MIN\_SCORE

\textbf{Actual Output:} Tasks are completed successfully and ``Cultural Friendliness'' questions get average scores of 4.8

\textbf{Result:} Pass
\end{enumerate}
\subsection{Compliance}
\begin{enumerate}
\item \textbf{NFRT-COM1}

\textbf{Name:} Code walkthrough on compliance requirements

\textbf{Initial State:} NA

\textbf{Input:} Source code (see details in section \ref{static})

\textbf{Expected Output:} Successfully convinced the participants the following:
\begin{itemize}
\item The data collected will be handled as per the same legal requirements for the university
\item The application can abide by the guidelines set by university staff
\end{itemize}

\textbf{Actual Output:} Successfully convinced the participants the following:
\begin{itemize}
\item The data collected will be handled as per the same legal requirements for the university
\item The application can abide by the guidelines set by university staff
\end{itemize}

\textbf{Result:} Pass
\end{enumerate}

\section{Usability Survey Result}\label{survey}

\section{Load Test}\label{load}

\section{Non dynamic Tests Result}\label{static}

\section{Comparison to Existing Implementation}	

Not applicable to this project since there is no existing implementation.

\section{Unit Testing}

\section{Changes Due to Testing}

\wss{This section should highlight how feedback from the users and from 
the supervisor (when one exists) shaped the final product.  In particular 
the feedback from the Rev 0 demo to the supervisor (or to potential users) 
should be highlighted.}

\section{Automated Testing}
		
\newpage
\begin{landscape}
\section{Trace to Requirements}

\footnotesize
\begin{longtable}{|l|cccccccccccccccc|}
  \hline
  \textbf{Test IDs} & \multicolumn{14}{c|}{\textbf{Functional Requirement IDs}} \\
  \hline
  ~                 & \textbf{FR1-1} & \textbf{FR2-1} & \textbf{FR2-2} & \textbf{FR2-3} & \textbf{FR2-4} & \textbf{FR2-5} & \textbf{FR2-6} & \textbf{FR2-7} & \textbf{FR3-1} & \textbf{FR3-2} & \textbf{FR3-3} & \textbf{FR3-4} & \textbf{FR3-5} \\
  \hline
  \textbf{FRT-PR1}  & X              & ~              & ~              & ~              & ~              & ~              & ~              & ~              & ~              & ~              & ~              & ~              & ~              \\
  \textbf{FRT-PR2}  & X              & ~              & ~              & ~              & ~              & ~              & ~              & ~              & ~              & ~              & ~              & ~              & ~              \\
  \textbf{FRT-UA1}  & ~              & X              & ~              & ~              & ~              & ~              & ~              & ~              & ~              & ~              & ~              & ~              & ~              \\
  \textbf{FRT-UA2}  & ~              & X              & ~              & ~              & ~              & ~              & ~              & ~              & ~              & ~              & ~              & ~              & ~              \\
  \textbf{FRT-UA3}  & ~              & ~              & ~              & X              & ~              & ~              & ~              & ~              & ~              & ~              & ~              & ~              & ~              \\
  \textbf{FRT-UA4}  & ~              & ~              & ~              & X              & ~              & ~              & ~              & ~              & ~              & ~              & ~              & ~              & ~              \\
  \textbf{FRT-UA5}  & ~              & ~              & X              & ~              & ~              & ~              & ~              & ~              & ~              & ~              & ~              & ~              & ~              \\
  \textbf{FRT-UA6}  & ~              & ~              & ~              & ~              & X              & ~              & ~              & ~              & ~              & ~              & ~              & ~              & ~              \\
  \textbf{FRT-UA7}  & ~              & ~              & ~              & ~              & ~              & X              & ~              & ~              & ~              & ~              & ~              & ~              & ~              \\
  \textbf{FRT-UA8}  & ~              & ~              & ~              & ~              & ~              & ~              & X              & ~              & ~              & ~              & ~              & ~              & ~              \\
  \textbf{FRT-UA9}  & ~              & ~              & ~              & ~              & ~              & ~              & ~              & X              & ~              & ~              & ~              & ~              & ~              \\
  \textbf{FRT-SN1}  & ~              & ~              & ~              & ~              & ~              & ~              & ~              & ~              & X              & ~              & ~              & ~              & ~              \\
  \textbf{FRT-SN2}  & ~              & ~              & ~              & ~              & ~              & ~              & ~              & ~              & X              & ~              & ~              & ~              & ~              \\
  \textbf{FRT-SN3}  & ~              & ~              & ~              & ~              & ~              & ~              & ~              & ~              & X              & ~              & ~              & ~              & ~              \\
  \textbf{FRT-SN4}  & ~              & ~              & ~              & ~              & ~              & ~              & ~              & ~              & ~              & X              & ~              & ~              & ~              \\
  \textbf{FRT-SN5}  & ~              & ~              & ~              & ~              & ~              & ~              & ~              & ~              & ~              & ~              & X              & ~              & ~              \\
  \textbf{FRT-SN6}  & ~              & ~              & ~              & ~              & ~              & ~              & ~              & ~              & ~              & ~              & ~              & X              & ~              \\
  \textbf{FRT-SN7}  & ~              & ~              & ~              & ~              & ~              & ~              & ~              & ~              & ~              & ~              & ~              & ~              & X              \\
  \hline
  \caption{Traceability Between Functional Test Cases and Functional Requirements, FR-1 to FR-3-5} \\
\end{longtable}

\newpage

\begin{longtable}{|l|cccccccccccc|}
  \hline
  \textbf{Test IDs} & \multicolumn{12}{c|}{\textbf{Functional Requirement IDs}} \\
  \hline
  ~                 & \textbf{FR3-6} & \textbf{FR4-1} & \textbf{FR4-2} & \textbf{FR4-3} & \textbf{FR4-4} & \textbf{FR4-5} & \textbf{FR4-6} & \textbf{FR4-7} & \textbf{FR4-8} &  \textbf{FR5-1} & \textbf{FR5-2} & \textbf{FR6-1} \\
  \hline
  \textbf{FRT-SN2}  & X              & ~              & ~              & ~              & ~              & ~              & ~              & ~              & ~              & ~              & ~               & ~              \\
  \textbf{FRT-SN3}  & X              & ~              & ~              & ~              & ~              & ~              & ~              & ~              & ~              & ~              & ~               & ~              \\
  \textbf{FRT-LE1}  & ~              & X              & ~              & ~              & ~              & ~              & ~              & ~              & ~              & ~              & ~               & ~              \\
  \textbf{FRT-LE2}  & ~              & X              & ~              & ~              & ~              & ~              & ~              & ~              & ~              & ~              & ~               & ~              \\
  \textbf{FRT-LE3}  & ~              & ~              & X              & ~              & ~              & ~              & ~              & ~              & ~              & ~              & ~               & ~              \\
  \textbf{FRT-LE4}  & ~              & ~              & X              & ~              & ~              & ~              & ~              & ~              & ~              & ~              & ~               & ~              \\
  \textbf{FRT-LE5}  & ~              & ~              & ~              & X              & ~              & ~              & ~              & ~              & ~              & ~              & ~               & ~              \\
  \textbf{FRT-LE6}  & ~              & ~              & ~              & X              & ~              & ~              & ~              & ~              & ~              & ~              & ~               & ~              \\
  \textbf{FRT-LE7}  & ~              & ~              & ~              & X              & ~              & ~              & ~              & ~              & ~              & ~              & ~               & ~              \\
  \textbf{FRT-LE8}  & ~              & ~              & ~              & ~              & X              & ~              & ~              & ~              & ~              & ~              & ~               & ~              \\
  \textbf{FRT-LE9}  & ~              & ~              & ~              & ~              & X              & ~              & ~              & ~              & ~              & ~              & ~               & ~              \\
  \textbf{FRT-LE10} & ~              & ~              & ~              & ~              & X              & ~              & ~              & ~              & ~              & ~              & ~               & ~              \\
  \textbf{FRT-LE11} & ~              & ~              & ~              & ~              & ~              & X              & ~              & ~              & ~              & ~              & ~               & ~              \\
  \textbf{FRT-LE12} & ~              & ~              & ~              & ~              & ~              & ~              & X              & ~              & ~              & ~              & ~               & ~              \\
  \textbf{FRT-LE13} & ~              & ~              & ~              & ~              & ~              & ~              & ~              & X              & ~              & ~              & ~               & ~              \\
  \textbf{FRT-LE14} & ~              & ~              & ~              & ~              & ~              & ~              & ~              & ~              & X              & ~              & ~               & ~              \\
  \textbf{FRT-AR1}  & ~              & ~              & ~              & ~              & ~              & ~              & ~              & ~              & ~              & X              & ~               & ~              \\
  \textbf{FRT-AR2}  & ~              & ~              & ~              & ~              & ~              & ~              & ~              & ~              & ~              & X              & ~               & ~              \\
  \textbf{FRT-AR3}  & ~              & ~              & ~              & ~              & ~              & ~              & ~              & ~              & ~              & ~              & X               & ~              \\
  \textbf{FRT-MAP1} & ~              & ~              & ~              & ~              & ~              & ~              & ~              & ~              & ~              & ~              & ~               & X              \\
  \textbf{FRT-MAP2} & ~              & ~              & ~              & ~              & ~              & ~              & ~              & ~              & ~              & ~              & ~               & X              \\
  \hline
  \caption{Traceability Between Functional Test Cases and Functional Requirements, FR-3-6 to FR-6} \\
\end{longtable}

\newpage
		
\section{Trace to Modules}		

\newpage
\end{landscape}

\section{Code Coverage Metrics}

\bibliographystyle{plainnat}
\bibliography{../../refs/References}

\newpage{}
\section*{Appendix --- Reflection}

The information in this section will be used to evaluate the team members on the
graduate attribute of Reflection.  Please answer the following question:

\begin{enumerate}
  \item In what ways was the Verification and Validation (VnV) Plan different
  from the activities that were actually conducted for VnV?  If there were
  differences, what changes required the modification in the plan?  Why did
  these changes occur?  Would you be able to anticipate these changes in future
  projects?  If there weren't any differences, how was your team able to clearly
  predict a feasible amount of effort and the right tasks needed to build the
  evidence that demonstrates the required quality?  (It is expected that most
  teams will have had to deviate from their original VnV Plan.)
\end{enumerate}

\end{document}