\documentclass[12pt, titlepage]{article}

\usepackage{booktabs}
\usepackage{tabularx}
\usepackage{float}
\usepackage[normalem]{ulem}
\usepackage{hyperref}
\hypersetup{
    colorlinks,
    citecolor=blue,
    filecolor=black,
    linkcolor=red,
    urlcolor=blue
}
\usepackage[round]{natbib}
\usepackage{longtable}
\usepackage{pdflscape}

\input{../Comments}
%% Common Parts

\newcommand{\progname}{Software Engineering} % PUT YOUR PROGRAM NAME HERE
\newcommand{\authname}{Team \#2, Team Name
\\ Zihao Du 
\\ Matthew Miller
\\ Firas Elayan
\\ Abhiram Neelamraju
\\ Michael Kim} % AUTHOR NAMES                  

\usepackage{hyperref}
    \hypersetup{colorlinks=true, linkcolor=blue, citecolor=blue, filecolor=blue,
                urlcolor=blue, unicode=false}
    \urlstyle{same}
                                


\begin{document}

\title{Project Title: System Verification and Validation Plan for \progname{}} 
\author{\authname}
\date{\today}
	
\maketitle

\pagenumbering{roman}

\section*{Revision History}

\begin{tabularx}{\textwidth}{p{3cm}p{2cm}X}
\toprule {\bf Date} & {\bf Version} & {\bf Notes}\\
\midrule
Oct 29 & 1.0 & Add Functional Requirements Tests\\
Oct 31 & 1.0 & Add Non-Functional Requirement Tests\\
Nov 1 & 1.0 & Add Unit Tests\\
Nov 3 & 1.0 & Add Tables And More Tests\\
Nov 22 & 1.0 & Revision 0: Resolve issues\\
Feb 21 & 1.1 & Revision 1: Introduce new usability testing\\
Feb 25 & 1.1 & Revision 1: Resolve TA Feedback: add TA review in the plan and fix grammar inconsistency\\
Feb 27 & 1.1 & Revision 1: Resolve TA Feedback for functional requirements: make functional tests more specific and add/remove tests according to Revision 1 requirements\\
Mar 2 & 1.1 & Revision 1: Resolve TA Feedback for non-functional requirements: add usability survey content to non-functional tests\\
Mar 3 & 1.1 & Revision 1: Add unit tests according to the design documentation; Add new non-dynamic tests to the table\\
Apr 3 & 1.1 & Revision 1: Changes related to VnV Report feedback\\
\bottomrule
\end{tabularx}

\newpage

\tableofcontents

\listoftables

\newpage

\section{Symbols, Abbreviations, and Acronyms}

\renewcommand{\arraystretch}{1.2}
\begin{tabular}{l l} 
  \toprule		
  \textbf{symbol} & \textbf{description}\\
  \midrule 
  AR & Augmented Reality\\
  \midrule
  Apache JMeter & A load testing tool for performance analysis\\
  \midrule
  APK & Android Package, a file format used by Android\\
  \midrule
  CAPTCHA & Human Verification Test\\
  \midrule
  FEMA & Failure Mode and Effect Analysis\\
  \midrule
  GPS & Global Positioning System\\
  \midrule
  HA & Hazard Analysis\\
  \midrule
  JHE & John Hodgins Engineering Building\\
  \midrule
  Mapbox & A provider of custom online maps for websites and applications\\
  \midrule
  MIS & Module Interface Specification\\
  \midrule
  SRS & Software Requirement Specification\\
  \midrule
  TA & Teaching Assistant\\
  \midrule
  UI & User Interface\\
  \midrule
  URI & Uniform Resource Identifier, identifies an abstract or physical resource\\
  \midrule
  UTF & Unity Test Framework\\
  \midrule
  VnV & Verification and Validation\\
  \midrule
  Vuforia & AR software development kit\\
  \bottomrule
\end{tabular}\\

\newpage

\pagenumbering{arabic}

This document describes the verification and validation plan for CampusConnection.  Within this document, there will be plans to verify the SRS document, the design plan, the VnV plan, the implementation plan, and the software validation plan.  Furthermore, this document will list specific system tests that cover all functional and non-function requirements we mentioned in the SRS, along with a traceability table between test cases and requirements to bind all requirements with tests. 

This document also includes unit tests, but since the design documentation like MIS is not ready yet, we will only cover the most important modules and their corresponding methods tests. The team will iterate on the VnV plan and have a more detailed version with the test code when the modules are specified in the design document.

\section{General Information}

\subsection{Summary}
  
The software we test is a social media application which also uses Augmented Reality features to allow McMaster University students to connect with each other. The software has many different functions to be able to accomplish its goals. These are some of the important functions:

 \begin{itemize}
     \item  Allowing the users to create and manage their account.

     \item  Allowing the users to add, manage and message friends.

     \item  Allowing the users with administrator access to add and manage events and lectures.

     \item  Allowing the users to view and filter lectures and events 
     
     \item Allowing the user to customize their schedule -- pin and unpin events or lectures based on their interest.

     \item  Identifying major campus buildings through the user's location and camera view, and using the information to display relevant lectures and events.

 \end{itemize}

 

\subsection{Objectives}

\subsubsection{Important Qualities}
In order to develop the software to meet its goals and create a complete product, there are many qualities that are important for our project such as:

\begin{itemize}
    \item An easily navigable User Interface adaptable to different screen sizes

    \item Sufficient speed and reliability of the building recognition features

    \item Sufficient speed and reliability in sending and receiving messages

    \item Adequate accuracy in tracking the user's location

    \item Appropriate abidance of security requirements regarding critical information such as location and personal data.

    \item Infrastructure to support a passable scale of total concurrent users and total user data storage
    \end{itemize}


\subsubsection{Out of Scope Objectives}
While we would want to satisfy as many positive qualities as possible, due to limited resources , there will be some objectives that we will be leaving out since they are out of scope such as :

\begin{itemize}
    \item  Project longevity requirements:

    This is because the longevity requirements were written with the assumption that the product will be supported even after the duration of the course for years to come which is something that is not possible to verify or guarantee and is out of the scope. Instead, we could have a code inspection with other developers and the supervisor about these requirements to gain some confidence.

    \item  Testing learning requirements:

    We will be producing material to make it easier to understand and use our product but to test every piece of supporting documentation and tutorial content would require long testing sessions with volunteers which can be out of the scope. Instead, we could use the feedback obtained from the survey to improve our learning-specific content.
    
   \item  Advanced extensibility and maintenance requirements:
   
    This is because the project only aims to support senior software students for now and We are not expected to serve all students by the end of Revision 1.  Maintenance requirements are designed for maintainers to maintain the project well in the future (after Revision 1 if the project can be used in real life), not developers. Therefore they will be out of the scope of this documentation. 
    
\end{itemize}


\subsubsection{Priority of Objectives}
We will be prioritizing reliability in our listed qualities over performance since we want our software to be correct and consistent over fast and unreliable. While we still want to maximize performance where possible, it is difficult to predict exactly how well we will meet our performance criteria since we will be limited by many factors out of our control such as external libraries.

We will also be assuming that any external libraries we depend on (such as Vuforia) will have already been verified by the implementation team due to our lack of control over them. 


\subsection{Relevant Documentation}

Many of the other documents associated with our project are relevant to the Verification and Validation plan such as:

\begin{itemize}
	\item Development Plan:

	The Development Plan contains the roles and responsibilities of each member, which outlines the tester of each components. It also includes the tools and technologies that will be used to facilitate testing

	\href{https://github.com/beatlepie/4G06CapstoneProjectTeam2/blob/docs-VnV-plan/docs/DevelopmentPlan/DevelopmentPlan.pdf}{Development Plan}
    \item Software Requirement Specification (SRS):
          
          The SRS has a lot of components relevant to the VnV plan. The functional requirements are important in identifying the primary functions of the the project while the non functional requirements are critical in identifying the main qualities and objectives of the product. It also contains constraint and assumption information which can help identify out-of-scope objectives.

          \href{https://github.com/beatlepie/4G06CapstoneProjectTeam2/blob/docs-VnV-plan/docs/SRS/SRS.pdf}{Software Requirement Specification (SRS)}
    \item Hazard Analysis (HA):
    
        The HA document contains the Failure Mode and Effect Analysis (FMEA) which is incredibly useful in understanding the possible failures and their consequences. This can help us assign priorities to different failures that can be translated into priorities for different qualities identified earlier in the document. It also contains more detailed safety and security requirements that we can use to verify our project.

        \href{https://github.com/beatlepie/4G06CapstoneProjectTeam2/blob/docs-VnV-plan/docs/HazardAnalysis/HazardAnalysis.pdf}{Hazard Analysis}
\end{itemize}


\section{Plan}


\subsection{Verification and Validation Team}

\begin{table}[H]
  \centering
  \begin{tabular}{|p{0.3\textwidth}|p{0.6\textwidth}|}
  \hline
  \textbf{Team Member} & \textbf{Role in Verification and Validation} \\
  \hline
  Abhiram Neelamraju & \textbf{Lead tester, Unit tester}, Leads the testing process, runs unit tests and generates test reports \\
  \hline
  Michael Kim & \textbf{Usability Tester}, Tests the user interface of the software to assess how intuitive the layout and UI are. \\
  \hline
  Firas Elayan & \textbf{Validation analyst}, Analyses test results against standards and project goals to ensure alignment. \\
  \hline
  Zihao Du & \textbf{Performance tester}, Runs tests and analyses to evaluate the performance of the software system. \\
  \hline
  Matthew Miller & \textbf{Feedback coordinator}, Manages the feedback gathered through reviews conducted by different parties as well as during testing to ensure the feedback is taken into account during development. \\
  \hline
  Waseef Nayeem & \textbf{Issue resolver}, Deal with bugs and issues that arise during the development and validation of the software. \\
  \hline
  \end{tabular}
\end{table}    

\subsection{SRS Verification Plan}

We will be following a few approaches to verify our SRS document, establishing a strong base for the development of our project.

\begin{itemize}
  \item Ad-hoc review by classmates:

      We will be conducting unstructured reviews of our SRS document with our classmates. A couple of different teams of students will go through the document to give us feedback and improvement suggestions around its contents and formatting by creating GitHub issues. This has been done before for Revision 0 and we decided to continue verifying the document for Revision 1.
  
  \item Ad-hoc review by our supervisor:

      Our supervisor, Dr. Yuan will perform a review of our SRS document by reading through it and giving us feedback on it. Dr. Yuan's expertise would provide us with valuable insight, allowing us to improve the quality and correctness of the SRS. Given her experience as well, it would ensure we're properly following set standards.

    \item SRS walkthrough by team:

      A walkthrough of the document will be performed together by the members of our team. We will all go through each section and analyse it, where we'll be able to suggest changes and improvements. Different team members worked on different parts of the document, so this walkthrough will allow for new perspectives on each section.
      
   \item Ad-hoc review by TA:
   
   We will perform a review of our SRS with our TA, Sam. This meeting will be used to discuss how we can resolve Revision 0 feedback and we can also get some more suggestions about what can be improved in Revision 1.

\end{itemize}

\subsection{Design Verification Plan}

\begin{itemize}
  \item Creating mock-ups and prototypes:

      In order to carry out good and efficient evaluations of our design, we will create design mock-ups and prototypes. These will be lo-fi at first and increase in fidelity as the software development progresses. The prototypes will help improve the reviews and feedback of the design by providing a visual (possibly physical) representation of our design, helping reviewers judge the features and quality of the design.
  
  \item Requirements checklist:

      A checklist of our requirements from the SRS document will be made to be used as part of the verification of our design. The checklist will be used during team reviews in order to ensure our design meets the project requirements.
  
  \item Supervisor review:

      As with the SRS document, our designs will receive an ad hoc review by our supervisor, Dr. Yuan. She will ensure our designs meet software standards and adhere to relevant principles.   

  \item Classmates review:

      We will conduct reviews with our classmates to get their input on our designs. This is very valuable for us as our classmates fall under one of our stakeholders, the customers.
      
  \item TA review:

      We will conduct reviews with our TA Same to get his ideas about our designs. He can also help us resolve feedback he gave for Revision 0.

\end{itemize}

\subsection{Verification and Validation Plan Verification Plan}

\begin{itemize}
  \item Team review:

      Our team members will review the VnV plan together to ensure the plan is complete and correct in how/what it verifies and validates. The review will allow us to make sure the VnV plan is line with the goals set for our project.
  
  \item Supervisor review:

      As with the SRS document and our designs, our supervisor, Dr. Yuan, will conduct a review of our VnV plan. Her experience and knowledge of software engineering will allow her to find any shortcomings in our plans and to provide very valuable feedback.

  \item Classmates review:

      Our classmates will conduct their own review of our VnV plan. As they resemble our stakeholders and each have their own experience and skills, their feedback can prove to be very beneficial and can provide a range of perspectives on the effectiveness our plan.
      
  \item TA feedback:

      Our TA, Sam will give us feedback on VnV plan.  This feedback will be useful to improve our documentation in Revision 1.

  \item Mutation testing:

      In order to determine how effective our outlined tests in the VnV plan are at ensuring the correctness of our product, we will employ mutation testing. Mutations will be introduced to the parts of our source code that is covered by the test cases outlined here. These mutations would be very small but should cause the tests to fail. This would allow us to determine how effective our tests are at finding faults and flaws in our product's code.

\end{itemize}

\subsection{Implementation Verification Plan}

      Our implementation verification plan involves making use of the system and unit tests we've outlined in the VnV plan to ensure the correctness and quality of the implementation of our product.\\\\
      We've outlined system tests and unit tests for our requirements. The system tests apply to the application as a whole, placing a focus on verifying the correctness of the application and how well it satisfies the requirements we've outlined in our SRS document. Our unit tests focus on specific modules within our application, aiming at assessing the robustness and accuracy of each module when working individually. All our tests are based on our functional and non-functional requirements to ensure our application adheres to our goals for the project.\\\\
      In addition to the tests outlined in this document, we will also employ some static techniques to verify our implementation. We will be making use of verification tools that are able to identify errors and issues in our source code. We will also be conducting code reviews, where we'll be examining our code (without executing it) to find any problems in relation with our code's adherence to our design and requirements.

\subsection{Automated Testing and Verification Tools}

      We intend to make use of several different tools that will help us in the testing and verification of our product. We have outlined some of these tools in our development plan.\\\\
      Since we are coding our product using C\#, we will be making use of dotnet format for C\# to format our code. Linters are incredibly useful tools that we can use to detect any programming errors and bugs in our source code, as well as any issues with the style. Using this linter would help avoid these problems growing as we build the project, ensuring its correctness.\\\\
      As we are using Unity as our engine to develop our application, our testing framework will be the Unity Test Framework. Unity Test Framework can be used in both Play Mode and Edit mode and there are other UI driven end to end automated tool that we can use if it is manual testing spends too much time.\\\\
      For all tests or linter mentioned above, we will build a continuous integration pipeline in our Github repo so that they can be tested automatically. 

\subsection{Software Validation Plan}

      We will be conductions formal reviews with some of our stakeholders as well as our supervisor, Dr. Yuan. These reviews will ensure that our application meets the requirements as well as the goals of our stakeholders. Dr. Yuan will make sure the product we've build meets different standards.\\\\
      Moreover, we'll use task-based inspection, involving testing different aspects of our app through real-world scenarios. This will aid us in evaluating the functionality of our software. In addition, we will also have students - who represent our stakeholders - test out features of our software, which will allow us to observe our software's performance in the real world. The feedback we get from user testing would be very helpful in ensuring the alignment of our software with the user goals and requirements.    

\section{System Test Description}
\label{sec:system}
	
\subsection{Tests for Functional Requirements}

The following section includes system tests for functional requirements defined in the \href{https://github.com/beatlepie/4G06CapstoneProjectTeam2/blob/main/docs/SRS/SRS.pdf}{SRS} document. In order to cover all the functional requirements, these subsections match exactly the subsections in the SRS document, which cover all the different components of this application. These tests will ensure all requirements are fulfilled and the application works as we expected. Most of the following tests will be run manually while some of them will be automated.

\subsubsection{Pre-Registration Settings}

This section includes all test cases related to what users can do before they start to use the application. This will include the requirement for the consent form and user privacy protection (FR-1-1). In order to create an account and use the application, the user must agree to the terms and conditions in the consent form. The following tests will have random accounts as input and we assume the success of the test on random accounts indicates the success of the test on all accounts.

\begin{enumerate}

\item{\textbf{FRT-PR1}}

\textbf{Name:} Agree To Consent Form

\textbf{Control:} Manual
					
\textbf{Initial State:} The user does not have an account, and they starts to register an account. A consent form appears asking for access to the device and permission to collect user data

\textbf{Input:} The user agrees to all the terms and conditions and clicks `Agree' and continues to complete the registration process
					
\textbf{Output:} A notification shows the registration succeeds and the user is redirected to the login screen

\textbf{Test Case Derivation:} It is a must to request users' consent before collecting their information or using any hardware component
					
\textbf{How test will be performed:} The tester will first clear the app cache and data. Then the tester will run the application and register a new account. Then they will open up the consent form and click on `Agree' button, try to finish the registration and verify the registration is successful and the login screen shows up as well

\textbf{Related Requirement(s):} FR-1-1
					
\item{\textbf{FRT-PR2}}

\textbf{Name:} Disagree To Consent Form

\textbf{Control:} Manual
					
\textbf{Initial State:} The user does not have an account, and they starts to register an account. A consent form appears asking for access to the device and permission to collect user data
					
\textbf{Input:} The user rejects the terms and conditions and clicks `Disagree' and continues to complete the registration process
					
\textbf{Output:} The registration fails and a warning will show up notifying the user that they cannot create an account unless they agree to the consent form

\textbf{Test Case Derivation:} It is a must to request users' consent before collecting their information or using any hardware component

\textbf{How test will be performed:} The tester will first clear the app cache and data. Then the tester will run the application and register a new account. Then they will open up the consent form and click on `Disagree' button, try to finish the registration and verify a warning shows up

\textbf{Related Requirement(s):} FR-1-1

\end{enumerate}

\subsubsection{User Account}

This section includes all test cases related to user accounts. This will include the requirements for account creation (FR-2-1), login (FR-2-3), deletion (FR-2-2) and email verification(FR-2-6), along with user avatar settings (FR-2-5), password management(FR-2-4) and profile management (FR-2-7). User accounts and virtual avatars are necessary for all users who want to use this application. The following tests will have random accounts as input and we assume the success of the test on random accounts indicates the success of the test on all accounts.

\begin{enumerate}

\item{\textbf{FRT-UA1}}

\textbf{Name:} Successful Account Creation

\textbf{Control:} Manual
					
\textbf{Initial State:} The user does not have an account and is not logged in to the application

\textbf{Input:} All information needed to create an account:
\begin{itemize}
\item Email: testUA1@gmail.com
\item password: FRT-UA1
\item nickname: UA1
\end{itemize}
					
\textbf{Output:} An Account with corresponding information is created in the database with the account initialized to INITIAL\_USER\_STATE

\textbf{Test Case Derivation:}  User account operations are the foundation of the application. All functionalities work only with an existing account. Also, it is essential to ensure the account information reflects the input information when creating accounts
					
\textbf{How test will be performed:} The tester will create a test account with all information above and verify that the account can be logged into

\textbf{Related Requirement(s):} FR-2-1

\item{\textbf{FRT-UA2}}

\textbf{Name:} Unsuccessful Account Creation With Existing Email

\textbf{Control:} Manual
					
\textbf{Initial State:} The user does not have an account and is not logged in to the application

\textbf{Input:} All information needed to create an account:
\begin{itemize}
\item Email: qtest@gmail.com (this is an existing test account)
\item password: FRT-UA1
\item nickname: UA1
\end{itemize}
					
\textbf{Output:} Account creation fails with a warning telling the user the email has already been used

\textbf{Test Case Derivation:} Email is the identifier of a user account, which should be unique
					
\textbf{How test will be performed:} The tester will create an account with the information above and verify that the registration fails with a warning

\textbf{Related Requirement(s):} FR-2-1

\item{\textbf{FRT-UA3}}

\textbf{Name:} Successful Account Login

\textbf{Control:} Manual
					
\textbf{Initial State:} The user has an account and is not logged in to the application

\textbf{Input:} All information needed to login:
\begin{itemize}
\item Email: FRT-UA3@test.com (this account exists in the system already)
\item password: FRT-UA3
\end{itemize}
					
\textbf{Output:} User successfully logs into the application and goes to the menu page

\textbf{Test Case Derivation:} User account operations are the foundation of the application. All functionalities work only with an existing account. Also, it is essential to verify that users can log in with only the correct password 
					
\textbf{How test will be performed:} The tester will login with the email and password given above and verifies that the login succeeds and the user is redirected to the menu page

\textbf{Related Requirement(s):} FR-2-3

\item{\textbf{FRT-UA4}}

\textbf{Name:} Unsuccessful Account Login With Wrong Password

\textbf{Control:} Manual
					
\textbf{Initial State:} The user has an account and is not logged in to the application

\textbf{Input:} All information needed to login:
\begin{itemize}
\item Email: FRT-UA3@test.com (this account exists in the system already)
\item password: FRT321 (wrong password)
\end{itemize}
					
\textbf{Output:} Login fails with a warning telling the user the password is wrong

\textbf{Test Case Derivation:} User account operations are the foundation of the application. All functionalities work only with an existing account. Also, it is essential to verify that users can not log in with a wrong password
					
\textbf{How test will be performed:} The tester will login with the email and password given above and verifies that the login fails with a warning

\textbf{Related Requirement(s):} FR-2-3

\item{\textbf{FRT-UA5}}

\textbf{Name:} Account Deletion

\textbf{Control:} Manual
					
\textbf{Initial State:} The user has an account and is logged into the application
\begin{itemize}
\item Email: FRT-UA5@gmail.com (this is an existing test account)
\item password: FRT-UA5
\item nickname: UA5
\end{itemize}

\textbf{Input:} User clicks on the delete account button on the profile page and confirms the deletion
					
\textbf{Output:} The user is redirected to the login page and the account cannot be logged in any more

\textbf{Test Case Derivation:} User account operations are the foundation of the application. When users want to quit, they should be able to delete all related information in the application by deleting their account
					
\textbf{How test will be performed:} The tester will delete the account above and verify the account cannot be logged in any more

\textbf{Related Requirement(s):} FR-2-2

\item{\textbf{FRT-UA6}}

\textbf{Name:} Reset Account Password

\textbf{Control:} Manual
					
\textbf{Initial State:} The user has an account:
\begin{itemize}
\item Email: campusconnections@gmail.com (this is an existing test account)
\item password: qtesting
\end{itemize}

\textbf{Input:} Email address and new password
\begin{itemize}
\item new password: QTesting
\end{itemize}
					
\textbf{Output:} Password is successfully reset

\textbf{Test Case Derivation:} In case the user misplaces or forgets the password, they can still get their account back
					
\textbf{How test will be performed:} The user will click on `Forget Password' button on the login page and enter the email address above. Then the user will follow the instruction on the email sent by the system and enters a new password. After that the user will login with the new password to verify the new password works properly

\textbf{Related Requirement(s):} FR-2-4

\item{\textbf{FRT-UA7}}

\textbf{Name:} Avatar Creation and Modification

\textbf{Control:} Manual
					
\textbf{Initial State:} The user has an account with DEFAULT\_AVATAR

\textbf{Input:} URI represents the new avatar:
\begin{itemize}
\item URI: https://upload.wikimedia.org/wikipedia/commons/2/2f/Google\_2015\_logo.svg
\end{itemize}
					
\textbf{Output:} The user changes the avatar to a Google logo

\textbf{Test Case Derivation:} To help users enjoy social networking when using this application, we should allow them to customize their account, therefore it is necessary to verify they can create an avatar and change it whenever they want
					
\textbf{How test will be performed:} The tester will edit the avatar URI and change it to be the URI above. Then the tester will verify the avatar has been changed to a Google logo

\textbf{Related Requirement(s):} FR-2-5

\item{\textbf{FRT-UA8}}

\textbf{Name:} Email Verification

\textbf{Control:} Manual
					
\textbf{Initial State:} The user has an account whose email has not been verified yet
\begin{itemize}
\item Email: fuz15@mcmaster.ca (this is an existing test account)
\item password: password
\end{itemize}

\textbf{Input:} User clicks on `Verify Email` button on user profile page and follows instructions on the email sent from the system
					
\textbf{Output:} That email above is verified as a valid email address

\textbf{Test Case Derivation:} For students to get access to events and lectures, they must prove the system that they have a valid school email, therefore, email verification is important.
					
\textbf{How test will be performed:} The tester will login with the email and password given above and go to user profile page and click on `Verify Email' button. Then the tester will follow the instructions of the email sent by the system to verify the email. Finally the tester will login and verify that the email verification button is disabled because the email has been verified

\textbf{Related Requirement(s):} FR-2-6

\item{\textbf{FRT-UA9}}

\textbf{Name:} Edit Profile

\textbf{Control:} Manual
					
\textbf{Initial State:} The user has an account 
\begin{itemize}
\item Email: qtest@gmail.com (this is an existing test account)
\item password: qtesting
\item program: Software Engineering
\end{itemize}

\textbf{Input:} New Profile:
\begin{itemize}
\item newProgram: Computer Science
\item newLevel: 4
\end{itemize}
					
\textbf{Output:} The program and level are updated

\textbf{Test Case Derivation:} To help users enjoy social networking when using the application, we should allow them to edit their information whenever they want
					
\textbf{How test will be performed:} The tester will login with the email and password given above and go to user profile page and click on `Edit' button. Then the tester will enter new program and level and save the changes Finally the tester will verify the profile has been updated

\textbf{Related Requirement(s):} FR-2-7
\end{enumerate}

\subsubsection{Social Networking System}

This section includes all test cases related to the social networking system. This will include the requirements for adding (FR-3-1), deleting (FR-3-2), messaging friends of the user (FR-3-3, FR-3-4) and sharing location (FR-3-5). Expanding student social network is the most significant motivation of this application and it is necessary to verify all associated requirements are fully fulfilled. The following tests will have random accounts as input and we assume the success of the test on random accounts indicates the success of the test on all accounts.

\begin{enumerate}
\item{\textbf{FRT-SN1}}

\textbf{Name:} Successful Friend Request

\textbf{Control:} Manual
					
\textbf{Initial State:} The user is logged in with the following account:
\begin{itemize}
\item Semail: FRT-SN1@test.com
\item password: testing
\end{itemize}

\textbf{Input:} A valid email to send the request:
\begin{itemize}
\item Temail: FRT-SN1-F@test.com
\end{itemize}
					
\textbf{Output:} A Request is sent to the target user

\textbf{Test Case Derivation:} Users should be able to make friends on this social media platform by searching for their ID (email) and sending requests
					
\textbf{How test will be performed:} The tester will login and send a friend request from the account given in initial state to the target user. Then the tester will login to the target account and verify that there is a pending friend request

\textbf{Related Requirement(s):} FR-3-1

\item{\textbf{FRT-SN2}}

\textbf{Name:} Friend Request Acceptance

\textbf{Control:} Manual
					
\textbf{Initial State:} A friend request was sent from an account (Semail) to the target account (Temail):
\begin{itemize}
\item Semail: FRT-SN1@test.com
\item Temail: FRT-SN1-F@test.com
\end{itemize}

\textbf{Input:} The request is accepted
					
\textbf{Output:} Two users are added to each other's friend lists

\textbf{Test Case Derivation:} Users should be able to expand networking by accepting friend requests
					
\textbf{How test will be performed:} The tester will first login to target account and accept the request. Then the tester will login to sender account to verify that they are in friend lists of each other

\textbf{Related Requirement(s):} FR-3-1, FR-3-6

\item{\textbf{FRT-SN3}}

\textbf{Name:} Friend Request Rejection

\textbf{Control:} Manual
					
\textbf{Initial State:} A friend request was sent from an account (Semail) to the target account (Temail):
\begin{itemize}
\item Semail: FRT-SN1@test.com
\item Temail: FRT-SN1-F@test.com
\end{itemize}

\textbf{Input:} The request is rejected
					
\textbf{Output:} The request is declined and no friend is added for both accounts

\textbf{Test Case Derivation:} Users should be able to decline friend requests from strangers
					
\textbf{How test will be performed:} The tester will first login to target account and reject the request. Then the tester will login to sender account to verify that they are not in friend lists of each other

\textbf{Related Requirement(s):} FR-3-1, FR-3-6

\item{\textbf{FRT-SN4}}

\textbf{Name:} Friend Deletion

\textbf{Control:} Manual
					
\textbf{Initial State:} A friend (Femail) exist in the friend list of the test account (Temail):
\begin{itemize}
\item Temail: FRT-SN4@test.com
\item Femail: FRT-SN4-F@test.com
\end{itemize}

\textbf{Input:} User deletes the chosen friend
					
\textbf{Output:} The corresponding friend is deleted from the list

\textbf{Test Case Derivation:} Users should be able to remove friends from their list to make space for new friends
					
\textbf{How test will be performed:} The tester will login to the test account (Temail) and delete the friend (Femail) and verify the friend is removed from the list

\textbf{Related Requirement(s):} FR-3-2

\item{\textbf{FRT-SN5}}

\textbf{Name:} Friend Messaging

\textbf{Control:} Manual
					
\textbf{Initial State:} A friend (Femail) exist in the friend list of the test account (Temail):
\begin{itemize}
\item Temail: FRT-SN5@test.com
\item Femail: FRT-SN5-F@test.com
\end{itemize}

\textbf{Input:} Message: `Hello World'
					
\textbf{Output:} The corresponding message is sent to the friend

\textbf{Test Case Derivation:} Chatting with friends is the core functionality of expanding social networks. Friends should be able to send and receive messages from each other
					
\textbf{How test will be performed:} The tester will log in to both the test account and the friend account. Then the tester will open the chat and send the input message from test account and verify that message shows up in the friend account chat

\textbf{Related Requirement(s):} FR-3-3

\item{\textbf{FRT-SN6}}

\textbf{Name:} Friend Sharing Event

\textbf{Control:} Manual
					
\textbf{Initial State:} A friend (Femail) exist in the friend list of the test account (Temail):
\begin{itemize}
\item Temail: FRT-SN6@test.com
\item Femail: FRT-SN6-F@test.com
\end{itemize}

\textbf{Input:} Message that contains event name and follows some specific pattern:
Hey, check this event: \textunderscore E\textunderscore[EXPO]
					
\textbf{Output:} User will be redirect to the event page with that event once they click on the message

\textbf{Test Case Derivation:} Users shall be able to share events they found with friends conveniently so that they can join events together in person.
					
\textbf{How test will be performed:} The tester will log in to both the test account and the friend account. Then the tester will open the chat and send that message from test account and verify that message can be clicked in the friend account chat and it redirects to event page correctly.

\textbf{Related Requirement(s):} FR-3-4

\item{\textbf{FRT-SN7}}

\textbf{Name:} Friend Sharing Lecture

\textbf{Control:} Manual
					
\textbf{Initial State:} A friend (Femail) exist in the friend list of the test account (Temail):
\begin{itemize}
\item Temail: FRT-SN7@test.com
\item Femail: FRT-SN7-F@test.com
\end{itemize}

\textbf{Input:} Message that contains lecture code and follows some specific pattern:
`Hey, are you in this lecture: \textunderscore L\textunderscore[SFRWENG 4G06]'
					
\textbf{Output:} User will be redirect to the lecture page with that lecture once they click on the message

\textbf{Test Case Derivation:} Users shall be able to share lectures they found with friends conveniently so that they can meet more classmate and study together.
					
\textbf{How test will be performed:} The tester will log in to both the test account and the friend account. Then the tester will open the chat and send that message from test account and verify that message can be clicked in the friend account chat and it redirects to lecture page correctly.

\textbf{Related Requirement(s):} FR-3-5
\end{enumerate}

\subsubsection{Lectures and Events}

This section includes all test cases related to events and lectures happening on campus. This will include the requirements for events and lectures themselves, users interacting with these properties, and the power of administration accounts. The test cases ensure that the lectures and events information is accurate and helpful for all users. The following tests will have random lecture/event as input and we assume the success of the test on random event/lecture indicates the success of the test on all of them.

\begin{enumerate}
\item{\textbf{FRT-LE1}}

\textbf{Name:} Save Event

\textbf{Control:} Manual
					
\textbf{Initial State:} A sample event:
\begin{itemize}
\item Name: EXPO
\end{itemize}

\textbf{Input:} On the event page, user clicks on the save button on the pop-up window with details of the sample event
					
\textbf{Output:} The event is saved to the user's event list

\textbf{Test Case Derivation:} The user should be able to interact with events available on campus and customize their own event list. Therefore, they should be able to save events for later
					
\textbf{How test will be performed:} The tester will go to event page and find the sample event, click on the save button on the pop-up window. Then the tester will verify the event shows up in the user's saved event list.

\textbf{Related Requirement(s):} FR-4-1

\item{\textbf{FRT-LE2}}

\textbf{Name:} Unsave Event

\textbf{Control:} Manual
					
\textbf{Initial State:} A sample event that is already been saved:
\begin{itemize}
\item Name: EXPO
\end{itemize}

\textbf{Input:} On the event page, user clicks on the unsave button on the pop-up window with details of the sample event
					
\textbf{Output:} The event is removed from the user's event list

\textbf{Test Case Derivation:} The user should be able to interact with events available on campus and customize their own event list. Therefore, they should be able to unsave events if that is no longer needed
					
\textbf{How test will be performed:} The tester will go to event page and find the sample event, click on the unsave button on the pop-up window. Then the tester will verify the event disappears in the user's saved event list.

\textbf{Related Requirement(s):} FR-4-1

\item{\textbf{FRT-LE3}}

\textbf{Name:} Save Lecture

\textbf{Control:} Manual
					
\textbf{Initial State:} A sample lecture:
\begin{itemize}
\item Code: SFWRENG 4G06
\end{itemize}

\textbf{Input:} On the lecture page, user clicks on the save button on the pop-up window with details of the sample lecture
					
\textbf{Output:} The event is saved to the user's lecture list

\textbf{Test Case Derivation:} The user should be able to interact with lectures available on campus and customize their own lecture list. Therefore, they should be able to save lectures for later
					
\textbf{How test will be performed:} The tester will go to lecture page and find the sample lecture, click on the save button on the pop-up window. Then the tester will verify the event shows up in the user's saved lecture list.

\textbf{Related Requirement(s):} FR-4-2

\item{\textbf{FRT-LE4}}

\textbf{Name:} Unsave Lecture

\textbf{Control:} Manual
					
\textbf{Initial State:} A sample lecture that is already been saved:
\begin{itemize}
\item Code: SFWRENG 4G06
\end{itemize}

\textbf{Input:} On the lecture page, user clicks on the unsave button on the pop-up window with details of the sample lecture
					
\textbf{Output:} The event is removed from the user's lecture list

\textbf{Test Case Derivation:} The user should be able to interact with lectures available on campus and customize their own lecture list. Therefore, they should be able to unsave lectures if that is no longer needed
					
\textbf{How test will be performed:} The tester will go to lecture page and find the sample lecture, click on the unsave button on the pop-up window. Then the tester will verify the event disappears in the user's saved lecture list.

\textbf{Related Requirement(s):} FR-4-2

\item{\textbf{FRT-LE5}}

\textbf{Name:} Administrator Add Event

\textbf{Control:} Manual
					
\textbf{Initial State:} User is logged in as an administrator
\begin{itemize}
\item email: campusconnections@gmail.com
\item password: testing
\end{itemize}

\textbf{Input:} Sample event:
\begin{itemize}
\item name: Test event
\item description: Sample event for system test
\item time: 0
\item duration: 0
\item location: Online
\item isPublic: true
\item organizer: Team 2
\end{itemize}
					
\textbf{Output:} The event is added to the event list

\textbf{Test Case Derivation:} Administrators work as a source of truth in this application, therefore they shall be able to create new events for all users
					
\textbf{How test will be performed:} The tester will log in as an administrator and add the sample event. Then the tester will verify the event is shown in the list with all information

\textbf{Related Requirement(s):} FR-4-3

\item{\textbf{FRT-LE6}}

\textbf{Name:} Administrator Add Event With Existing Name

\textbf{Control:} Manual
					
\textbf{Initial State:} User is logged in as an administrator
\begin{itemize}
\item email: campusconnections@gmail.com
\item password: testing
\end{itemize}

An event with following information exists:
\begin{itemize}
\item name: Test event
\item description: Sample event for system test
\item time: 0
\item duration: 0
\item location: Online
\item isPublic: true
\item organizer: Team 2
\end{itemize}

\textbf{Input:} New sample event:
\begin{itemize}
\item name: Test event
\item description: New Description
\item time: 100
\item duration: 30
\item location: NA
\item isPublic: false
\item organizer: NA
\end{itemize}
					
\textbf{Output:} The sample event information is changed to the one in the input

\textbf{Test Case Derivation:} Administrators know the event name should be unique and it is unlikely that they give a different event the same name, so this happens most likely when they want to update an event (for instance monthly updated club event)
					
\textbf{How test will be performed:} The tester will log in as an administrator and add the sample event. Then the tester will verify the event is still in the list with all information updated

\textbf{Related Requirement(s):} FR-4-3

\item{\textbf{FRT-LE7}}

\textbf{Name:} Administrator Edit Event

\textbf{Control:} Manual
					
\textbf{Initial State:} User is logged in as an administrator
\begin{itemize}
\item email: campusconnections@gmail.com
\item password: testing
\end{itemize}

\textbf{Input:} Sample event name and new location:
\begin{itemize}
\item name: Test event
\item location: ITB AB102
\end{itemize}
					
\textbf{Output:} The test event location is updated to the new one

\textbf{Test Case Derivation:} Administrators work as a source of truth in this application, therefore they shall be able to update event information if that is out of date
					
\textbf{How test will be performed:} The tester will log in as an administrator, find the sample event and update the sample event location. Then the tester will verify the event location shown in the list is updated

\textbf{Related Requirement(s):} FR-4-3

\item{\textbf{FRT-LE8}}

\textbf{Name:} Administrator Delete Event

\textbf{Control:} Manual
					
\textbf{Initial State:} User is logged in as an administrator
\begin{itemize}
\item email: campusconnections@gmail.com
\item password: testing
\end{itemize}

\textbf{Input:} Sample event (already in the system) name:
\begin{itemize}
\item name: Test event
\end{itemize}
					
\textbf{Output:} The event is deleted and disappears from the list

\textbf{Test Case Derivation:} Administrators work as a source of truth in this application, therefore they shall be able to clean old events
					
\textbf{How test will be performed:} The tester will log in as an administrator, find the sample event and click on delete button. Then the tester will verify the event disappears from the list

\textbf{Related Requirement(s):} FR-4-3

\item{\textbf{FRT-LE9}}

\textbf{Name:} Administrator Add Lecture

\textbf{Control:} Manual
					
\textbf{Initial State:} User is logged in as an administrator
\begin{itemize}
\item email: campusconnections@gmail.com
\item password: testing
\end{itemize}

\textbf{Input:} Sample lecture:
\begin{itemize}
\item code: TEST 1T03
\item name: Test lecture
\item time: 12:00 - 13:00, Mon
\item location: Online
\item instructor: NA
\end{itemize}
					
\textbf{Output:} The lecture is added to the lecture list

\textbf{Test Case Derivation:} Administrators work as a source of truth in this application, therefore they shall be able to create new lectures for all users
					
\textbf{How test will be performed:} The tester will log in as an administrator and add the sample lecture. Then the tester will verify the lecture is shown in the list with all information

\textbf{Related Requirement(s):} FR-4-4

\item{\textbf{FRT-LE10}}

\textbf{Name:} Administrator Add Lecture With Existing Code

\textbf{Control:} Manual
					
\textbf{Initial State:} User is logged in as an administrator
\begin{itemize}
\item email: campusconnections@gmail.com
\item password: testing
\end{itemize}

A lecture with following information exists:
\begin{itemize}
\item code: TEST 1T03
\item name: Test lecture
\item time: 12:00 - 13:00, Mon
\item location: Online
\item instructor: NA
\end{itemize}

\textbf{Input:} New sample lecture:
\begin{itemize}
\item code: TEST 1T03
\item name: New Test lecture
\item time: 12:00 - 13:00, Wed
\item location: TBD
\item instructor: TBD
\end{itemize}
					
\textbf{Output:} The sample lecture information is changed to the one in the input

\textbf{Test Case Derivation:} Administrators know that lecture code should be unique and it is very unlikely that they give a different lecture the same code, so this happens most likely when they want to update a lecture (for instance in a new semester the lecture has a new schedule)
					
\textbf{How test will be performed:} The tester will log in as an administrator and add the sample lecture. Then the tester will verify the lecture is still in the list with all information updated

\textbf{Related Requirement(s):} FR-4-4

\item{\textbf{FRT-LE11}}

\textbf{Name:} Administrator Edit Lecture

\textbf{Control:} Manual
					
\textbf{Initial State:} User is logged in as an administrator
\begin{itemize}
\item email: campusconnections@gmail.com
\item password: testing
\end{itemize}

\textbf{Input:} Sample lecture name and new location:
\begin{itemize}
\item code: TEST 1T03
\item location: ITB AB102
\end{itemize}
					
\textbf{Output:} The test lecture location is updated to the new one

\textbf{Test Case Derivation:} Administrators work as a source of truth in this application, therefore they shall be able to update lecture information if that is out of date
					
\textbf{How test will be performed:} The tester will log in as an administrator, find the sample lecture and update the sample lecture location. Then the tester will verify the lecture location shown in the list is updated

\textbf{Related Requirement(s):} FR-4-4

\item{\textbf{FRT-LE12}}

\textbf{Name:} Administrator Delete Lecture

\textbf{Control:} Manual
					
\textbf{Initial State:} User is logged in as an administrator
\begin{itemize}
\item email: campusconnections@gmail.com
\item password: testing
\end{itemize}

\textbf{Input:} Sample lecture (already in the system) name:
\begin{itemize}
\item code: TEST 1T03
\end{itemize}
					
\textbf{Output:} The lecture is deleted and disappears from the list

\textbf{Test Case Derivation:} Administrators work as a source of truth in this application, therefore they shall be able to clean old lectures
					
\textbf{How test will be performed:} The tester will log in as an administrator, find the sample lecture and click on delete button. Then the tester will verify the lecture disappears from the list

\textbf{Related Requirement(s):} FR-4-4

\item{\textbf{FRT-LE13}}

\textbf{Name:} Event Information

\textbf{Control:} Manual
					
\textbf{Initial State:} A sample event exists:
\begin{itemize}
\item name: EXPO
\end{itemize}

\textbf{Input:} User clicks on the sample event
					
\textbf{Output:} All event information is shown in a pop-up window

\textbf{Test Case Derivation:} All events should follow some pattern and have following information:
\begin{itemize}
\item organizer
\item location
\item time
\item duration
\end{itemize}
					
\textbf{How test will be performed:} The tester will find the sample event in the table and click on it. Then the tester will verify all entries listed above are shown in the pop-up window

\textbf{Related Requirement(s):} FR-4-5

\item{\textbf{FRT-LE14}}

\textbf{Name:} Lecture Information

\textbf{Control:} Manual
					
\textbf{Initial State:} A sample lecture exists:
\begin{itemize}
\item code: SFWRENG 4G06
\end{itemize}

\textbf{Input:} User clicks on the sample lecture
					
\textbf{Output:} All lecture information is shown in a pop-up window

\textbf{Test Case Derivation:} All events should follow some pattern and have following information:
\begin{itemize}
\item instructor
\item location
\item time
\item code
\end{itemize}
					
\textbf{How test will be performed:} The tester will find the sample lecture in the table and click on it. Then the tester will verify all entries listed above are shown in the pop-up window

\textbf{Related Requirement(s):} FR-4-6

\item{\textbf{FRT-LE15}}

\textbf{Name:} Lecture Filter by code

\textbf{Control:} Manual
					
\textbf{Initial State:} Some software engineering lecture exists:
\begin{itemize}
\item SFWRENG 4G06
\item SFWRENG 4E03
\end{itemize}

\textbf{Input:} Filter:
\begin{itemize}
\item FilterString: SFWRENG
\end{itemize}
					
\textbf{Output:} All lectures which does not contain the FilterString in the code are removed from the list

\textbf{Test Case Derivation:} In order to find lectures easily, filter by lecture course code is needed if the lecture list is very long
					
\textbf{How test will be performed:} The tester will find go to the lecture page and enter the FilterString and click the search button. Then the tester will verify only software engineering courses are left in the list

\textbf{Related Requirement(s):} FR-4-7

\item{\textbf{FRT-LE16}}

\textbf{Name:} Event Filter by Name

\textbf{Control:} Manual
					
\textbf{Initial State:} Some job fair event exists:
\begin{itemize}
\item Job Fair: March 4
\item Job Fair: March 10
\end{itemize}

\textbf{Input:} Filter:
\begin{itemize}
\item FilterString: Job Fair
\end{itemize}
					
\textbf{Output:} All event which does not contain the FilterString in the name are removed from the list

\textbf{Test Case Derivation:} In order to find event easily, filter by event name is needed if the event list is very long
					
\textbf{How test will be performed:} The tester will find go to the event page and enter the FilterString and click the search button. Then the tester will verify only job fair events are left in the list

\textbf{Related Requirement(s):} FR-4-8
\end{enumerate}

\subsubsection{AR Camera}

This section includes all test cases related to campus navigation with AR technology. This will include the requirements for building recognition (FR-5-1) and demonstrating activities in this building (FR-5-2). Augmented reality provides an immersive user experience and it is the unique selling point of the application. The following tests are about one of the buildings (JHE) only we assume the success of these tests indicates the success of the test on all other target buildings.

\begin{enumerate}
\item{\textbf{FRT-AR1}}

\textbf{Name:} Successful Building Recognition

\textbf{Control:} Manual
					
\textbf{Initial State:} User is at the front door of JHE

\textbf{Input:} Clear camera view
					
\textbf{Output:} The building is recognized with name and description shown as an AR object

\textbf{Test Case Derivation:} Building recognition is the essential functionality for AR technology applied in this application. The building should be recognizable from a certain angle when users look at their camera
					
\textbf{How test will be performed:} The tester will walk to the front door of JHE, open AR camera and look into it, verifying the building is recognized and a name and description is displayed on the screen

\textbf{Related Requirement(s):} FR-5-1

\item{\textbf{FRT-AR2}}

\textbf{Name:} Unsuccessful Building Recognition

\textbf{Control:} Manual
					
\textbf{Initial State:} User is out of campus

\textbf{Input:} Clear camera view
					
\textbf{Output:} No AR objects is shown

\textbf{Test Case Derivation:} Building recognition is the essential functionality for AR technology applied in this application. Any buildings out of campus should not be recognized because they are out of the scope of this project and buildings that are not targets should not be recognized for high accuracy oft AR camera
					
\textbf{How test will be performed:} The tester will walk out of campus, open AR camera and look into it, verifying there is no objects shown on the screen

\textbf{Related Requirement(s):} FR-5-1

\item{\textbf{FRT-AR3}}

\textbf{Name:} Building Lectures/Events

\textbf{Control:} Manual
					
\textbf{Initial State:} User is in JHE lobby

\textbf{Input:} Clear camera view
					
\textbf{Output:} Event and lecture information separated by room number at the corresponding locations of the building

\textbf{Test Case Derivation:} Building recognition is the essential functionality for AR technology applied in this application. The building should be recognizable from a certain angle when users look at their camera
					
\textbf{How test will be performed:} The tester will walk around in JHE lobby, open AR camera and look into it, verifying there are lectures and events information shown at the rooms where they will be held

\textbf{Related Requirement(s):} FR-5-2
\end{enumerate}

\subsubsection{Map and Location}
\begin{enumerate}
\item{\textbf{FRT-MAP1}}

\textbf{Name:} User Location

\textbf{Control:} Manual
					
\textbf{Initial State:} User allows the user to use their real-time location

\textbf{Input:} User enters the map page
					
\textbf{Output:} A model represents the user shows up on the map and moves correspondingly when the user moves

\textbf{Test Case Derivation:} For those who come to the campus for the first time, a map is necessary for them to find the building they want to go to
					
\textbf{How test will be performed:} The tester will go to map page and walk around on campus and verify if the model moves correspondingly on the map

\textbf{Related Requirement(s):} FR-6-1

\item{\textbf{FRT-MAP2}}

\textbf{Name:} Friend Locations

\textbf{Control:} Manual
					
\textbf{Initial State:} User has some friends who are willing to share locations:
\begin{itemize}
\item email1: MAP2-1@test.com
\item email2: MAP2-2@test.com
\end{itemize}

\textbf{Input:} User enters the map page
					
\textbf{Output:} Another models represent friends shows up on the map and moves correspondingly when friends moves

\textbf{Test Case Derivation:} It will be fun to know where your friends are currently on campus and it makes it easier to have in person communications
					
\textbf{How test will be performed:} Testers will login with friends account (email1 and email2), go to map page and walk around on campus and the other tester will verify if their model moves correspondingly on the map

\textbf{Related Requirement(s):} FR-6-1
\end{enumerate}

\subsection{Tests for Nonfunctional Requirements}
\label{sec:nonfunctional}

The following section includes tests for non-functional requirements defined in the \href{https://github.com/beatlepie/4G06CapstoneProjectTeam2/blob/main/docs/SRS/SRS.pdf}{SRS} document. Areas of testing follow the subsections of requirements in the previous document, which mainly include Look and Feel, Usability and Humanity, Performance and Security.\\ 
This section also includes new requirements for the HA, but when they are added to the SRS, they are given new IDs for consistency, please check the HA for more details. \\
Usability requirements will be tested by asking users to do a survey, whose content can be found in section \ref{sec:survey}. Most of the tests here are dynamic and will be done manually or automatically, but there are some tests that need non-dynamic testing like peer code review or code walkthrough.

\subsubsection{Look and Feel}

\begin{enumerate}
\item{\textbf{NFRT-LF1}}

\textbf{Name:} Survey for feedback on application layout

\textbf{Type:} Non-functional, Dynamic, Manual
					
\textbf{Initial State:} Survey taker is given an account:
\begin{itemize}
\item email: mtest@gmail.com
\item password: mtesting
\end{itemize}

\textbf{Input/Condition:} User follows the instructions and attempts to finish tasks in the survey 
					
\textbf{Output/Result:} The users should be able to complete tasks without any help and the average individual score is over MIN\_SCORE in the``Immediate Visual Response when Clicking'' and ``Appealing Colour Scheme'' categories of the survey
					
\textbf{How test will be performed:} A survey will be given to at least
SURVEY\_SAMPLE\_SIZE users where they will give feedback to different elements of the interface. All of the survey takers are McMaster students

\textbf{Related Requirement(s):} LF-A1, LF-S2

\item{\textbf{NFRT-LF2}}

\textbf{Name:} Visual inspection with different screen sizes

\textbf{Type:} Non-functional, Dynamic, Manual
					
\textbf{Initial State:} NA
					
\textbf{Input/Condition:} User opens the application on the phone with all different screen sizes in the SCREEN\_VIEWPORTS list

\textbf{Output/Result:} For all different pages all visual elements on the screen are within the borders of the screen for all screens in the SCREEN\_VIEWPORTS list

\textbf{How test will be performed:} The application will be opened on all screens in the SCREEN\_VIEWPORTS list.\
The testers will visually inspect each page and each screen to make sure that all visual elements on the screen are within its borders

\textbf{Related Requirement(s):} LF-A2

\item{\textbf{NFRT-LF3}}

\textbf{Name:} Visual inspection of color scheme

\textbf{Type:} Non-functional, Dynamic, Manual
					
\textbf{Initial State:} NA
					
\textbf{Input/Condition:} User opens the application on the phone
					
\textbf{Output/Result:} For all different pages the colour scheme is the same

\textbf{How test will be performed:} The tester will visually inspect each page to make sure that the colour scheme on each page is the same

\textbf{Related Requirement(s):} LF-S1
\end{enumerate}

\subsubsection{Usability and Humanity}

\begin{enumerate}
\item{\textbf{NFRT-UH1}}

\textbf{Name:} Survey for feedback on understandability and easy of use

\textbf{Type:} Non-functional, Dynamic, Manual
					
\textbf{Initial State:} Survey taker is given an account:
\begin{itemize}
\item email: mtest@gmail.com
\item password: mtesting
\end{itemize}

\textbf{Input/Condition:} User follows the instructions and attempts to finish tasks in the survey 
					
\textbf{Output/Result:} The user can finish tasks without help and the average individual score is over MIN\_SCORE in the ``No Technical or Software-Specific Language'' category of the survey
					
\textbf{How test will be performed:} A survey will be given to at least
SURVEY\_SAMPLE\_SIZE users where they will give feedback to different elements of the interface. All of the survey takers are McMaster students

\textbf{Related Requirement(s):} UH-EOU1, UH-UP1

\item{\textbf{NFRT-UH2}}

\textbf{Name:} Walkthrough of user guide 

\textbf{Type:} Non-functional, Static, Manual
					
\textbf{Input/Condition:} GitHub web page
					
\textbf{Output/Result:} Convinced the participants that the main features are explained in the GitHub repo

\textbf{How test will be performed:} The tester will use a guest GitHub account and navigate to the GitHub page and verify there exists explanation of application features

\textbf{Related Requirement(s):} UH-L1

\item{\textbf{NFRT-UH3}}

\textbf{Name:} Visual inspection of color contrast 

\textbf{Type:} Non-functional, Manual
					
\textbf{Initial State:} NA
					
\textbf{Input/Condition:} User opens the application checks the color contrast statically
					
\textbf{Output/Result:} The color contrast is greater than 4.5:1,  the Web Content Accessibility Guidelines' AA standards for accessibility

\textbf{How test will be performed:} The tester will statically check if the color contrast is enough for items to be easily visible. (Contrast between background color and text)

\textbf{Related Requirement(s):} UH-A1
\end{enumerate}

\subsubsection{Performance}

\begin{enumerate}
\item{\textbf{NFRT-P1}}

\textbf{Name:} AR camera recognition

\textbf{Type:} Non-functional, Dynamic, Manual
					
\textbf{Initial State:} The user is near a target building (JHE)
					
\textbf{Input/Condition:} User turns on AR camera
					
\textbf{Output/Result:} Corresponding AR objects appears quickly

\textbf{How test will be performed:} The tester will turn on the AR camera and look at JHE and verify that the recognition time ls less than RECOGNITION\_TIME

\textbf{Related Requirement(s):} P-SL1

\item{\textbf{NFRT-P2}}

\textbf{Name:} Real-time location update

\textbf{Type:} Non-functional, Dynamic, Manual
					
\textbf{Initial State:} User allows the application to use device location
					
\textbf{Input/Condition:} User turns on the map and walks around on campus
					
\textbf{Output/Result:} The user model on the map is updated when the user is moving

\textbf{How test will be performed:} The tester will open the map and walk around, and verify that the model on the map is updated within LOCATION\_UPDATE\_TIME

\textbf{Related Requirement(s):} P-SL2

\item{\textbf{NFRT-P3}}

\textbf{Name:} AR camera accuracy

\textbf{Type:} Non-functional, Dynamic, Manual
					
\textbf{Initial State:} User is near a target building (JHE)
					
\textbf{Input/Condition:} User turns on the AR Camera
					
\textbf{Output/Result:} AR objects shows up

\textbf{How test will be performed:} The tester will try the camera from different angles and verify that the success rate is at lease AR\_ACCURACY

\textbf{Related Requirement(s):} P-PA1

\item{\textbf{NFRT-P4}}

\textbf{Name:} Warning when internet connection is lost

\textbf{Type:} Non-functional, Dynamic, Manual
					
\textbf{Initial State:} User has no internet connection
					
\textbf{Input/Condition:} User opens the application
					
\textbf{Output/Result:} There is a pop-up window telling the user the internet is lost

\textbf{How test will be performed:} The tester will open the application with offline mode and visually check the warning message appears

\textbf{Related Requirement(s):} P-RF1

\item{\textbf{NFRT-P5}}

\textbf{Name:} Rudimentary functions when the server connection is lost

\textbf{Type:} Non-functional, Dynamic, Manual
					
\textbf{Initial State:} Server is turned down
					
\textbf{Input/Condition:} User opens the application
					
\textbf{Output/Result:} The application still works with limited functionalities

\textbf{How test will be performed:} The tester will open the application without an available server and check the map can still show the user location, the saved events and lectures and other user profile information can be viewed and changed

\textbf{Related Requirement(s):} P-RF2

\item{\textbf{NFRT-P6}}

\textbf{Name:} Code inspection for server restart

\textbf{Type:} Non-functional, Static, Manual
					
\textbf{Input/Condition:} Server settings
					
\textbf{Output/Result:} Successfully convinced the stakeholders and users the following:
\begin{itemize}
\item Server attempts to restart when it crashes
\end{itemize}

\textbf{How test will be performed:} A code inspection will be held with superior's attendance to show that the server is set to restart upon crash

\textbf{Related Requirement(s):} P-RF3

\item{\textbf{NFRT-P7}}

\textbf{Name:} AR camera help button

\textbf{Type:} Non-functional, Dynamic, Manual
					
\textbf{Initial State:} User turns on the AR camera

\textbf{Input/Condition:} User clicks the help button
					
\textbf{Output/Result:} A message telling the user things that may affect AR camera appears

\textbf{How test will be performed:} The tester will turn on the AR camera and click on the help button and visually check if the message shows up

\textbf{Related Requirement(s):} P-RF4

\item{\textbf{NFRT-P8}}

\textbf{Name:} Load testing for server

\textbf{Type:} Non-functional, Dynamic, Automatic
					
\textbf{Initial State:} The server is online and ready to connect to

\textbf{Input/Condition:} Load testing tool simulates there are MAX\_CAPACITY users simultaneously accessing the server

\textbf{Output/Result:} The server is able to handle all the users connecting to the server at once

\textbf{How test will be performed:} The tester will use Apache JMeter for the load test simulation

\textbf{Related Requirement(s):} P-SE1

\item{\textbf{NFRT-P9}}

\textbf{Name:} Code inspection for database capacity

\textbf{Type:} Non-functional, Static, Automatic

\textbf{Input/Condition:} Database documentation

\textbf{Output/Result:} Successfully convinced the participants the following:
\begin{itemize}
\item The database has enough space to store all the user, lecture and event information
\end{itemize}

\textbf{How test will be performed:} A code inspection will be held with superior's attendance to show that the Firebase database official document states there will be enough capacity with the current plan

\textbf{Related Requirement(s):} P-SE2

\item{\textbf{NFRT-P10}}

\textbf{Name:} Code walkthrough for adding new building

\textbf{Type:} Non-functional, Static, Automatic

\textbf{Input/Condition:} Source code

\textbf{Output/Result:} Successfully convinced the participants the following:
\begin{itemize}
\item A new target building can be added without causing the application running any slower
\end{itemize}

\textbf{How test will be performed:} A code walkthrough will be held with the superior's attendance to show that an extra target building will not affect the performance of the application

\textbf{Related Requirement(s):} P-SE3

\item{\textbf{NFRT-P11}}

\textbf{Name:} Code Peer Evaluation For Longevity

\textbf{Type:} Non-functional, Static, Manual

\textbf{Input/Condition:} Source code

\textbf{Output/Result:} Successfully convinced the participants the following:
\begin{itemize} 
  \item The product is able to operate without major malfunctions in release build for at least 1 year
  \item The finalized product will remain compatible with the promised operating systems and devices for at least 3 years
\end{itemize}

\textbf{How test will be performed:} The tester will hold a peer evaluation session for the other developers to inspect the source code and make sure it can operate and remain compatible with promised operating systems for a long time

\textbf{Related Requirement(s):} P-L1, P-L2
\end{enumerate}

\subsubsection{Operational and Environmental}

\begin{enumerate}
\item{\textbf{NFRT-OE1}}

\textbf{Name:} Visual inspection for application download

\textbf{Type:} Non-functional, Manual

\textbf{Initial State:} User has a phone that uses Android 11 or above

\textbf{Input/Condition:} User wants to download the application

\textbf{Output/Result:} The product can be downloaded onto the phone from the Google Play Store, or by downloading the APK file directly

\textbf{How test will be performed:} The testers will check the Google Play Store for the product, or find an APK file for the product.\
The testers will then download and install the product onto the phone from the Google Play Store or the APK file.

\textbf{Related Requirement(s):} OE-EPE1, OE-P1
\end{enumerate}

\subsubsection{Maintainability and Support}

\begin{enumerate}
\item{\textbf{NFRT-MS1}}

\textbf{Name:} Survey for maintenance time

\textbf{Type:} Non-functional, Static, Manual

\textbf{Initial State:} Survey taker is given an account:
\begin{itemize}
\item email: mtest@gmail.com
\item password: mtesting
\end{itemize}

\textbf{Input/Condition:} User answers the question about ``Common periods of usage''

\textbf{Output/Result:} The average individual score is over MIN\_SCORE for this question

\textbf{How test will be performed:} A survey will be given to at least SURVEY\_SAMPLE\_SIZE users where they will tell if they are most likely to use the application at school time only

\textbf{Related Requirement(s):} MS-M1

\item{\textbf{NFRT-MS2}}

\textbf{Name:} Check for feature request

\textbf{Type:} Non-functional, Dynamic, Manual
					
\textbf{Initial State:} A public GitHub repo exists for this application

\textbf{Input/Condition:} User goes to the GitHub repo

\textbf{Output/Result:} User can read issues created by the team and also create new issues

\textbf{How test will be performed:} The tester will verify the team's feature issues are open to view and new issues can be raised for feature request or bug report

\textbf{Related Requirement(s):} MS-S1, MS-S2

\item{\textbf{NFRT-MS3}}

\textbf{Name:} Android version test

\textbf{Type:} Non-functional, Dynamic, Manual
					
\textbf{Initial State:} NA

\textbf{Input/Condition:} The application is installed on devices with Android 11 and above version

\textbf{Output/Result:} The application works without any error about compatibility

\textbf{How test will be performed:} The tester will install the application on different Android version (11 and above) and verify the application can start without errors, and the core features (Map, friend chatting and AR camera) have the same performance over different Android versions.

\textbf{Related Requirement(s):} MS-A1
\end{enumerate}

\subsubsection{Security}

\begin{enumerate}
\item{\textbf{NFRT-S1}}

\textbf{Name:} Access test
  
\textbf{Type:} Non-functional, Dynamic, Manual
            
\textbf{Initial State:} Three accounts with different accesses is ready:
\begin{itemize}
\item Admin: campusconnections@gmail.com
\item User: mtest@gmail.com
\item Guest gtest@gmail.com
\end{itemize}
  
\textbf{Input/Condition:} The user starts the application with the three accounts
            
\textbf{Output/Result:} At each level of access, the application constrains the possible actions to what is specified in requirement S-A1, S-A2, S-A3.
  
\textbf{How test will be performed:}
\begin{enumerate}
  \item The tester will begin at the first level of access, which should only allow the user to view public events,  account recovery page, and single single-user map
  \item The tester will then test with the second level of access which should allow the user to do everything except edit, adding, or deleting activities
  \item The tester will test that actions only possible at the third level of access (adding/deleting/editing events, pull logs not accessible to users) are not possible to do at this level
\end{enumerate}

\textbf{Related Requirement(s):} S-A1, S-A2, S-A3

\item{\textbf{NFRT-S2}}

\textbf{Name:} Special character warning

\textbf{Type:} Non-functional, Dynamic, Manual
					
\textbf{Initial State:} User starts to register
					
\textbf{Input/Condition:} User enters nickname with some special character: `; DELETE *' (See all SPECIAL\_CHAR in the appendix)
					
\textbf{Output/Result:} A warning message is displayed telling the user that the special characters are not allowed and stops the user from registering

\textbf{How test will be performed:} The tester will start to register and enter the nickname above and visually inspect if the warning message appears

\textbf{Related Requirement(s):} P-SC5

\item{\textbf{NFRT-S3}}

\textbf{Name:} Email format warning

\textbf{Type:} Non-functional, Dynamic, Manual
					
\textbf{Initial State:} User starts to register
					
\textbf{Input/Condition:} User enters a not email string in the email field:
`SELECT * FROM TABLE'
					
\textbf{Output/Result:} A warning message is displayed telling the user that the input is not an email address and stops the user from registering

\textbf{How test will be performed:} The tester will start to register and enters a email `SELECT * FROM TABLE' and visually inspect if the warning message appears

\textbf{Related Requirement(s):} P-SC6
\end{enumerate}

\subsubsection{Privacy}
\begin{enumerate}
\item{\textbf{NFRT-PVC1}}

\textbf{Name:} Code inspection for legitimate use of personal data

\textbf{Type:} Non-functional, Static, Manual

\textbf{Input/Condition:} Source code

\textbf{Output/Result:} Successfully convinced the participants the following:
\begin{itemize}
\item The usage of a user's personal information by the product abides by the Privacy Act, The Personal Information Protection and Electronic Documents Act, and Canada and Ontario's data protection laws
\end{itemize}

\textbf{How test will be performed:} A code inspection will be held with supervisor's attendance to show the usage of personal data is legitimate

\textbf{Related Requirement(s):} S-P1

\item{\textbf{NFRT-PVC2}}

\textbf{Name:} Code inspection for removing unused accounts

\textbf{Type:} Non-functional, Static, Manual

\textbf{Input/Condition:} Source code

\textbf{Output/Result:} Successfully convinced the participants the following:
\begin{itemize}
\item Any accounts that are not active for a long time (a semester) will be removed from the system
\end{itemize}

\textbf{How test will be performed:} A code inspection will be held with supervisor's attendance to show the system is set to clean inactive users periodically

\textbf{Related Requirement(s):} S-P2

\item{\textbf{NFRT-PVC3}}

\textbf{Name:} Code Walkthrough For User Personal Data

\textbf{Type:} Non-functional, Static, Manual

\textbf{Input/Condition:} Source code

\textbf{Output/Result:} Successfully convinced the participants the following:
\begin{itemize}
  \item User's personal information does not appear in the database if the user did not grant permission
\end{itemize}

\textbf{How test will be performed:} A code walkthough will be held by the developers to show that database storing is only possible when the user grants permission

\textbf{Related Requirement(s):} P-SC1

\item{\textbf{NFRT-PVC4}}

\textbf{Name:} Code Walkthrough For Data Transmission Encryption

\textbf{Type:} Non-functional, Static, Manual

\textbf{Input/Condition:} Source code

\textbf{Output/Result:} Successfully convinced the participants the following:
\begin{itemize}
  \item The product only transmits encrypted data from server to user
\end{itemize}

\textbf{How test will be performed:}  A code walkthough will be held by the developers to show that the application transmit encrypted data only

\textbf{Related Requirement(s):} P-SC2

\item{\textbf{NFRT-PVC5}}

\textbf{Name:} Leaving campus warning

\textbf{Type:} Non-functional, Dynamic, Manual
					
\textbf{Initial State:} User opens the map
					
\textbf{Input/Condition:} User moves out of the campus
					
\textbf{Output/Result:} A warning message is displayed telling the user that the map is not available out of campus

\textbf{How test will be performed:} The tester will open the map and walk out of campus and visually inspect if the warning message appears

\textbf{Related Requirement(s):} P-SC4
\end{enumerate}

\subsubsection{User Safety}
\begin{enumerate}
\item{\textbf{NFRT-US1}}

\textbf{Name:} Warning when starting AR camera

\textbf{Type:} Non-functional, Dynamic, Manual
					
\textbf{Initial State:} User allows the application to use camera
					
\textbf{Input/Condition:} User turns on the AR camera
					
\textbf{Output/Result:} A warning telling the user to be aware of their surroundings is displayed upon start-up of the camera

\textbf{How test will be performed:} The testers will open the AR camera on one device and visually inspect if the message is shown

\textbf{Related Requirement(s):} P-SC3
\end{enumerate}

\subsubsection{Cultural}
\begin{enumerate}
\item{\textbf{NFRT-CUL1}}

\textbf{Name:} Survey for feedback on cultural requirements

\textbf{Type:} Non-functional, Dynamic, Manual

\textbf{Initial State:} Use has an account:
\begin{itemize}
\item email: mtest@gmail.com
\item password: mtesting
\end{itemize}

\textbf{Input/Condition:} User follows the instructions and attempts to finish tasks in the survey

\textbf{Output/Result:} The average individual score is over MIN\_SCORE in the "Cultural Friendliness" category of the Usability Survey

\textbf{How test will be performed:} A survey will be given to at least\
SURVEY\_SAMPLE\_SIZE users where they will give feedback to different\
elements of the interface. All of the survey takers are McMaster students

\textbf{Related Requirement(s):} CUL-C1
\end{enumerate}

\subsubsection{Compliance}
\begin{enumerate}
\item{\textbf{NFRT-COM1}}

\textbf{Name:} Code walkthrough on compliance requirements

\textbf{Type:} Non-functional, Static, Manual

\textbf{Input/Condition:} Source code

\textbf{Output/Result:} Successfully convinced the participants the following:
\begin{itemize}
\item The data collected will be handled as per the same legal requirements for the university
\item The application can abide by the guidelines set by university staff
\end{itemize}

\textbf{How test will be performed:} A code walkthrough will be held with the supervisor's attendance to show that the project fulfils the compliance requirements

\textbf{Related Requirement(s):} COM-SC1, COM-SC2
\end{enumerate}

\newpage
\begin{landscape}
\subsection{Traceability Between Test Cases and Requirements}
	\footnotesize
	\begin{longtable}{|l|cccccccccccccccc|}
	\caption{Traceability Between Functional Test Cases and Functional Requirements, FR-1 to FR-3-5}                                                                                                                                                                                                                           \\
		\hline
		\textbf{Test IDs}   & \multicolumn{14}{c|}{\textbf{Functional Requirement IDs}}                                                                                                                                                                                                                 \\
		\hline
		~                   & \textbf{FR1-1}  & \textbf{FR2-1} & \textbf{FR2-2} & \textbf{FR2-3} & \textbf{FR2-4} & \textbf{FR2-5} & \textbf{FR2-6} & \textbf{FR2-7} & \textbf{FR3-1} & \textbf{FR3-2} & \textbf{FR3-3} & \textbf{FR3-4} & \textbf{FR3-5} \\
		\hline
		\textbf{FRT-PR1} & X                                                         & ~             & ~             & ~             & ~             & ~             & ~             & ~             & ~             & ~             & ~             & ~             & ~             \\
		\textbf{FRT-PR2} & X                                                         & ~             & ~             & ~             & ~             & ~             & ~             & ~             & ~             & ~             & ~             & ~             & ~             \\
		\textbf{FRT-UA1} & ~                                                         & X             & ~             & ~             & ~             & ~             & ~             & ~             & ~             & ~             & ~             & ~             & ~             \\
		\textbf{FRT-UA2} & ~                                                         & X             & ~ & ~             & ~             & ~             & ~             & ~             & ~             & ~             & ~             & ~             & ~             \\
		\textbf{FRT-UA3} & ~                                                         & ~             & ~             & X             & ~             & ~             & ~             & ~             & ~             & ~             & ~             & ~             & ~             \\
		\textbf{FRT-UA4} & ~                                                         & ~             & ~             & X             & ~             & ~             & ~             & ~             & ~             & ~             & ~             & ~             & ~              \\
		\textbf{FRT-UA5} & ~                                                         & ~             & X             & ~             & ~             & ~             & ~             & ~             & ~             & ~             & ~             & ~             & ~             \\
		\textbf{FRT-UA6} & ~                                                         & ~             & ~             & ~             & X             & ~             & ~             & ~             & ~             & ~             & ~             & ~             & ~              \\
		\textbf{FRT-UA7} & ~                                                         & ~             & ~             & ~             & ~             & X             & ~             & ~             & ~             & ~             & ~             & ~             & ~             \\
		\textbf{FRT-UA8} & ~                                                         & ~             & ~             & ~             & ~             & ~             & X             & ~             & ~             & ~             & ~             & ~             & ~             & ~             \\
		\textbf{FRT-UA9} & ~                                                         & ~             & ~             & ~             & ~             & ~             & ~             & X             & ~             & ~             & ~             & ~             & ~             \\
		\textbf{FRT-SN1} & ~                                                         & ~             & ~             & ~             & ~             & ~             & ~             & ~             & X             & ~              & ~             & ~             & ~             \\
		\textbf{FRT-SN2} & ~                                                         & ~             & ~             & ~             & ~             & ~             & ~             & ~             & X             & ~             & ~             & ~             & ~             \\
		\textbf{FRT-SN3} & ~                                                         & ~             & ~             & ~             & ~             & ~             & ~             & ~             & X             & ~             & ~             & ~             & ~             \\
		\textbf{FRT-SN4} & ~                                                         & ~             & ~             & ~             & ~             & ~             & ~             & ~             & ~             & X             & ~             & ~             & ~             \\
		\textbf{FRT-SN5} & ~                                                         & ~             & ~             & ~             & ~             & ~             & ~             & ~             & ~             & ~             & X             & ~             & ~             \\
		\textbf{FRT-SN6} & ~                                                         & ~             & ~             & ~             & ~             & ~             & ~             & ~             & ~             & ~             & ~             & X             & ~             \\
		\textbf{FRT-SN7} & ~                                                         & ~             & ~             & ~             & ~             & ~             & ~             & ~             & ~             & ~             & ~             & ~             & X             \\
		\hline
	\end{longtable}

	\newpage
	\begin{longtable}{|l|cccccccccccc|}
		\caption{Functional Test Case and Requirement Traceability Cont.}                                                                                                                                                                                                  \\
		\hline
		\textbf{Test IDs}   & \multicolumn{12}{c|}{\textbf{Functional Requirement IDs}}                                                                                                                                                                                                                 \\
		\hline
		~                   & \textbf{FR3-6}  & \textbf{FR4-1}  & \textbf{FR4-2} & \textbf{FR-4-3} & \textbf{FR4-4} & \textbf{FR4-5} & \textbf{FR4-6} & \textbf{FR4-7} & \textbf{FR4-8} &  \textbf{FR5-1} & \textbf{FR5-2} & \textbf{FR6-1} \\
		\hline
		\textbf{FRT-SN2} & X                                                         & ~             & ~             & ~             & ~             & ~             & ~             & ~             & ~             & ~             & ~ &~\\
		\textbf{FRT-SN3} & X                                                         & ~             & ~             & ~             & ~             & ~             & ~             & ~             & ~             & ~             & ~ &~\\
		\textbf{FRT-LE1} & ~                                                         & X             & ~             & ~             & ~             & ~             & ~             & ~             & ~             & ~             & ~ &~\\
		\textbf{FRT-LE2} & ~                                                         & X             & ~             & ~             & ~             & ~             & ~             & ~             & ~             & ~             & ~ &~\\
		\textbf{FRT-LE3} & ~                                                         & ~             & X             & ~             & ~             & ~             & ~             & ~             & ~             & ~             & ~ &~\\
		\textbf{FRT-LE4} & ~                                                         & ~             & X             & ~             & ~             & ~             & ~             & ~             & ~             & ~             & ~ &~\\
		\textbf{FRT-LE5} & ~                                                         & ~             & ~             & X             & ~             & ~             & ~             & ~             & ~             & ~             & ~ &~\\
		\textbf{FRT-LE6} & ~                                                         & ~             & ~             & X             & ~             & ~             & ~             & ~             & ~             & ~             & ~ &~\\
		\textbf{FRT-LE7} & ~                                                         & ~             & ~             & X             & ~             & ~             & ~             & ~             & ~             & ~             & ~ &~\\
		\textbf{FRT-LE8} & ~                                                         & ~             & ~             & X             & ~             & ~             & ~             & ~             & ~             & ~             & ~ &~\\
		\textbf{FRT-LE9} & ~                                                         & ~             & ~             & ~             & X             & ~             & ~             & ~             & ~             & ~             & ~ &~\\
		\textbf{FRT-LE10} & ~                                                         & ~             & ~             & ~             & X             & ~             & ~             & ~             & ~             & ~             & ~ &~\\
		\textbf{FRT-LE11} & ~                                                         & ~             & ~             & ~             & X             & ~             & ~             & ~             & ~             & ~             & ~ &~\\
		\textbf{FRT-LE12} & ~                                                         & ~             & ~             & ~             & X             & ~             & ~             & ~             & ~             & ~             & ~ &~\\
		\textbf{FRT-LE13} & ~                                                         & ~             & ~             & ~             & ~             & X             & ~             & ~             & ~             & ~             & ~ &~\\
		\textbf{FRT-LE14} & ~                                                         & ~             & ~             & ~             & ~             & ~             & X             & ~             & ~             & ~             & ~ &~\\
		\textbf{FRT-LE15} & ~                                                         & ~             & ~             & ~             & ~             & ~             & ~             & X             & ~             & ~             & ~ &~\\
		\textbf{FRT-LE16} & ~                                                         & ~             & ~             & ~             & ~             & ~             & ~             & ~             & X             & ~             & ~ &~\\
		\textbf{FRT-AR1} & ~                                                         & ~             & ~             & ~             & ~             & ~             & ~             & ~             & ~             & X             & ~ &~\\
		\textbf{FRT-AR2} & ~                                                         & ~             & ~             & ~             & ~             & ~             & ~             & ~             & ~             & X             & ~ &~\\
		\textbf{FRT-AR3} & ~                                                         & ~             & ~             & ~             & ~             & ~             & ~             & ~             & ~             & ~             & X &~\\
		\textbf{FRT-MAP1} & ~                                                         & ~             & ~             & ~             & ~             & ~             & ~             & ~             & ~             & ~             & ~ &X\\
		\textbf{FRT-MAP2} & ~                                                         & ~             & ~             & ~             & ~             & ~             & ~             & ~             & ~             & ~             & ~ &X\\
		\hline
	\end{longtable}

	\newpage
	
		\begin{longtable}{|l|cccccccc|}
	\caption{Traceability Between Nonfunctional Test Cases and Nonfunctional Requirements, LF-A1 to UH-A1}                                                                                                                                                                                                                           \\
		\hline
		\textbf{Test IDs}   & \multicolumn{8}{c|}{\textbf{Nonfunctional Requirement IDs}}                                                                                                                                                                                                                 \\
		\hline
		~                   & \textbf{LF-A1}  & \textbf{LF-A2} & \textbf{LF-S1} & \textbf{LF-S2} & \textbf{UH-EOU1} & \textbf{UH-L1} & \textbf{UH-UP1} & \textbf{UH-A1} \\
		\hline
		\textbf{NFRT-LF1} & X                                                         & ~             & ~             & X             & ~             & ~             & ~             & ~\\
		\textbf{NFRT-LF2} & ~                                                         & X             & ~             & ~             & ~             & ~             & ~             & ~\\
		\textbf{NFRT-LF3} & ~                                                         & ~             & X             & ~             & ~             & ~             & ~             & ~\\
		\textbf{NFRT-UH1} & ~                                                         & ~             & ~             & ~             & X             & ~             & X             & ~\\
		\textbf{NFRT-UH2} & ~                                                         & ~             & ~             & ~             & ~             & X             & ~             & ~\\
		\textbf{NFRT-UH3} & ~                                                         & ~             & ~             & ~             & ~             & ~             & ~             & X\\
		\hline
	\end{longtable}

	\newpage
	
		\begin{longtable}{|l|ccccccccccccc|}
	\caption{Traceability Between Nonfunctional Test Cases and Nonfunctional Requirements, P-SL1 to P-RF4}                                                                                                                                                                                                                           \\
		\hline
		\textbf{Test IDs}   & \multicolumn{13}{c|}{\textbf{Nonfunctional Requirement IDs}}                                                                                                                                                                                                                 \\
		\hline
		~                   & \textbf{P-SL1}  & \textbf{P-SL2} & \textbf{P-SC1} & \textbf{P-SC2} & \textbf{P-SC3} & \textbf{P-SC4} & \textbf{P-SC5} & \textbf{P-SC6} & \textbf{P-PA1} & \textbf{P-RF1} & \textbf{P-RF2} & \textbf{P-RF3} & \textbf{P-RF4} \\
		\hline
		\textbf{NFRT-P1} & X                                                         & ~             & ~             & ~             & ~             & ~             & ~             & ~             & ~             & ~             & ~             & ~             & ~\\
		\textbf{NFRT-P2} & ~                                                         & X             & ~             & ~             & ~             & ~             & ~             & ~             & ~             & ~             & ~             & ~             & ~ \\
		\textbf{NFRT-\sout{P3} \textcolor{red}{PVC3}} & ~                                                         & ~             & X             & ~             & ~             & ~             & ~             & ~             & ~             & ~             & ~ & ~ & ~ \\
		\textbf{NFRT-\sout{P4} \textcolor{red}{PVC4}} & ~                                                         & ~             & ~             & X             & ~             & ~             & ~             & ~             & ~             & ~             & ~ & ~ & ~ \\
		\textbf{NFRT-\sout{P5} \textcolor{red}{US1}} & ~                                                         & ~             & ~             & ~             & X             & ~             & ~             & ~             & ~             & ~             & ~ & ~ & ~ \\
		\textbf{NFRT-\sout{P6} \textcolor{red}{PVC5}} & ~                                                         & ~             & ~             & ~             & ~             & X             & ~             & ~             & ~             & ~             & ~ & ~ & ~ \\
		\textbf{NFRT-\sout{P7} \textcolor{red}{S2}} & ~                                                         & ~             & ~             & ~             & ~             & ~             & X             & ~             & ~             & ~             & ~ & ~ & ~ \\
		\textbf{NFRT-\sout{P8} \textcolor{red}{S3}} & ~                                                         & ~             & ~             & ~             & ~             & ~             & ~             & X             & ~             & ~             & ~ & ~ & ~ \\
		\textbf{NFRT-P3} & ~                                                         & ~             & ~             & ~             & ~             & ~             & ~             & ~             & X             & ~             & ~ & ~ & ~ \\
		\textbf{NFRT-P4} & ~                                                         & ~             & ~             & ~             & ~             & ~             & ~             & ~             & ~             & X             & ~ & ~ & ~ \\
		\textbf{NFRT-P5} & ~                                                         & ~             & ~             & ~             & ~             & ~             & ~             & ~             & ~             & ~             & X & ~ & ~ \\
		\textbf{NFRT-P6} & ~                                                         & ~             & ~             & ~             & ~             & ~             & ~             & ~             & ~             & ~             & ~ & X & ~ \\
		\textbf{NFRT-P7} & ~                                                         & ~             & ~             & ~             & ~             & ~             & ~             & ~             & ~             & ~             & ~ & ~ & X \\
		\hline
	\end{longtable}

	\newpage
	
			\begin{longtable}{|l|ccccccccccc|}
	\caption{Traceability Between Nonfunctional Test Cases and Nonfunctional Requirements, P-SE1 to MS-A1}                                                                                                                                                                                                                           \\
		\hline
		\textbf{Test IDs}   & \multicolumn{11}{c|}{\textbf{Nonfunctional Requirement IDs}}                                                                                                                                                                                                                 \\
		\hline
		~                   & \textbf{P-SE1} & \textbf{P-SE2} & \textbf{P-SE3} & \textbf{P-L1} & \textbf{P-L2} & \textbf{OE-EPE1} & \textbf{OE-P1} & \textbf{MS-M1} & \textbf{MS-S1} & \textbf{MS-S2} & \textbf{MS-A1}  \\
		\hline
		\textbf{NFRT-P8} & X                                                         & ~             & ~             & ~             & ~             & ~             & ~             & ~             & ~             & ~ &~\\
		\textbf{NFRT-P9} & ~                                                         & X             & ~             & ~             & ~             & ~             & ~             & ~             & ~             & ~ &~\\
		\textbf{NFRT-P10} & ~                                                         & ~             & X             & ~             & ~             & ~             & ~             & ~             & ~             & ~ &~\\
		\textbf{NFRT-P11} & ~                                                         & ~             & ~             & X             & X             & ~             & ~             & ~             & ~             & ~ &~\\
		\textbf{NFRT-OE1} & ~                                                         & ~             & ~             & ~             & ~             & X             & X             & ~             & ~             & ~ &~\\
		\textbf{NFRT-MS1} & ~                                                         & ~             & ~             & ~             & ~             & ~             & ~             & X             & ~             & ~ &~\\
		\textbf{NFRT-MS2} & ~                                                         & ~             & ~             & ~             & ~             & ~             & ~             & ~             & X             &X  &~\\
		\textbf{NFRT-MS3} & ~                                                         & ~             & ~             & ~             & ~             & ~             & ~             & ~             & ~             &~  &X\\
		\hline
	\end{longtable}
	
	\newpage
	
			\begin{longtable}{|l|cccccccc|}
	\caption{Traceability Between Nonfunctional Test Cases and Nonfunctional Requirements, S-A1 to COMP-SC1}                                                                                                                                                                                                                           \\
		\hline
		\textbf{Test IDs}   & \multicolumn{8}{c|}{\textbf{Nonfunctional Requirement IDs}}                                                                                                                                                                                                                 \\
		\hline
		~                   & \textbf{S-A1} & \textbf{S-A2} & \textbf{S-A3} & \textbf{S-P1} & \textbf{S-P2} & \textbf{CUL-C1} & \textbf{COMP-L1} & \textbf{COMP-SC1}\\
		\hline
		\textbf{NFRT-S1} & X                                                         & X             & X             & ~             & ~             & ~             & ~             & ~\\
		\textbf{NFRT-PVC1} & ~                                                         & ~             & ~             & X             & ~             & ~             & ~             & ~\\
		\textbf{NFRT-PVC2} & ~                                                         & ~             & ~             & ~             & X             & ~             & ~             & ~\\
		\textbf{NFRT-CUL1} & ~                                                         & ~             & ~             & ~             & ~             & X             & ~             & ~\\
		\textbf{NFRT-COM1} & ~                                                         & ~             & ~             & ~             & ~             & ~             & X             & X\\
		\hline
	\end{longtable}

	\newpage
\end{landscape}

\section{Unit Test Description}
\label{sec:unit}

This section includes all unit tests for functional and non-functional requirements. Though most of the requirements will be tested manually, unit testing will still be utilized for some basic functionalities of some low level components. According to the module hierarchy in \href{https://github.com/beatlepie/4G06CapstoneProjectTeam2/blob/main/docs/Design/SoftArchitecture/MG.pdf}{MG}, the following modules will be tested with unit tests:
\begin{itemize}
	\item \textbf{M1} User
	\item \textbf{M2} Event
	\item \textbf{M3} Lecture
	\item \textbf{M4} Pagination and Filter
	\item \textbf{M5} Authentication
	\item \textbf{M6} Database
\end{itemize}

\subsection{Unit Testing Scope}

All modules defined above are within the testing scope.  The scope of unit testing was limited to the components that could be automatically tested. Third-party library modules like Vuforia and Mapbox are out of the scope of unit testing and we will trust they are bug-free. Also, high-level modules in the hierarchy are usually responsible for rendering UI and reacting to user actions (e.g. user profile module, lecture list module) therefore they are out of the scope of unit testing as well (in fact they are not quite testable with Unity test framework). Instead, they are tested with manual testing.

\subsection{Tests for Functional Requirements}

The following sections details unit tests for functional requirements. It is an essential aspect of testing as it verifies whether the modules are behaving
correctly given the requirements in the SRS.

\subsubsection{User Module}

This section will contain all unit tests for the module ``User''. The tests are chosen based on common user flows and cover the most important methods of this module. The getter and setter methods are part of the unit test as well, but they will not be included in the section because they are trivial.

\begin{enumerate}

\item{FRT-M1-1}

\textbf{Name:} Successful default user creation

\textbf{Type:} Functional, Dynamic, Automatic, Unit
					
\textbf{Initial State:} NA
					
\textbf{Input:} User identifier, email: unitTest@test.com
					
\textbf{Output:} A user is successfully created with following attributes:
\begin{itemize}
\item email: unitTest@test.com
\item nickname: `'
\item photoURI: null
\item program: `'
\item level: 0
\item lectures: empty list of string
\item event: empty list of string
\item friends: empty list of string
\item friendInvitation: empty list of string
\end{itemize}

\textbf{Test Case Derivation:} Verify that a new user can be created by just specifying the email and the rest attributes will be filled automatically

\textbf{How test will be performed:} Create a test case in UTF that creates a new user with all information provided, verify the user is created correctly

\item{FRT-M1-2}

\textbf{Name:} Successful new user creation

\textbf{Type:} Functional, Dynamic, Automatic, Unit
					
\textbf{Initial State:} NA
					
\textbf{Input:} User:
\begin{itemize}
\item email: unitTest@test.com
\item nickname: `Mr. Unit Test'
\item photoURI: http://example/com/photo.jpg
\item program: `software engineering'
\item level: 1
\item lectures: \{`Lecture 1', `Lecture 2'\}
\item event: \{`Event 1', `Event 2'\}
\item friends: \{`unitTest1@test.com', `unitTest2@test.com'\}
\item friendInvitation: \{`unitTest3@test.com'\}
\end{itemize}
					
\textbf{Output:} A user is successfully created with attributes above


\textbf{Test Case Derivation:} Verify that a new user can be created with all attributes specified

\textbf{How test will be performed:} Create a test case in UTF that creates a new user with all information provided, verify the user is created correctly
\end{enumerate}

\subsubsection{Event Module}

This section will contain all unit tests for the module ``Event''. The tests are chosen based on common user flows and cover the most important methods of this module. The getter and setter methods are part of the unit test as well, but they will not be included in the section because they are trivial.

\begin{enumerate}
\item{FRT-M2-1}

\textbf{Name:} Successful default event creation

\textbf{Type:} Functional, Dynamic, Automatic, Unit
					
\textbf{Initial State:} NA
					
\textbf{Input:} Event identifier, name: unit test event
					
\textbf{Output:} An event is successfully created with following attributes:
\begin{itemize}
\item name: unit test event
\item description: PLACEHOLDER
\item organizer: PLACEHOLDER
\item duration: DEFAULT\_DURATION
\item time: DEFAULT\_TIME
\item isPulbic: false
\item location: PLACEHOLDER
\end{itemize}

\textbf{Test Case Derivation:} Verify that a new event can be created by just specifying the name and the rest attributes will be filled automatically

\textbf{How test will be performed:} Create a test case in UTF that creates a new event with all information provided, verify the event is created correctly

\item{FRT-M2-2}

\textbf{Name:} Successful new event creation

\textbf{Type:} Functional, Dynamic, Automatic, Unit
					
\textbf{Initial State:} NA
					
\textbf{Input:} Event:
\begin{itemize}
\item name: unit test event2
\item description: A unit test sample event
\item organizer: Team 2
\item duration: 30
\item time: 1708922625
\item isPulbic: true
\item location: Online
\end{itemize}
					
\textbf{Output:} An event is successfully created with attributes above

\textbf{Test Case Derivation:} Verify that a new event can be created with all attributes specified

\textbf{How test will be performed:} Create a test case in UTF that creates a new event with all information provided, verify the event is created correctly
\end{enumerate}

\subsubsection{Lecture Module}

This section will contain all unit tests for the module ``Lecture''. The tests are chosen based on common user flows and cover the most important methods of this module. The getter and setter methods are part of the unit test as well, but they will not be included in the section because they are trivial.

\begin{enumerate}
\item{FRT-M3-1}

\textbf{Name:} Successful default lecture creation

\textbf{Type:} Functional, Dynamic, Automatic, Unit
					
\textbf{Initial State:} NA
					
\textbf{Input:} Lecture identifier, code: UNIT TEST3
					
\textbf{Output:} A lecture is successfully created with following attributes:
\begin{itemize}
\item code: UNIT TEST3
\item name: PLACEHOLDER
\item instructor: PLACEHOLDER
\item time: PLACEHOLDER
\item location: PLACEHOLDER
\end{itemize}

\textbf{Test Case Derivation:} Verify that a new lecture can be created by just specifying the code and the rest attributes will be filled automatically

\textbf{How test will be performed:} Create a test case in UTF that creates a new lecture with all information provided, verify the lecture is created correctly

\item{FRT-M3-2}

\textbf{Name:} Successful new lecture creation

\textbf{Type:} Functional, Dynamic, Automatic, Unit
					
\textbf{Initial State:} NA
					
\textbf{Input:} Lecture:
\begin{itemize}
\item code: UNIT TEST3
\item name: Unit test Lecture
\item instructor: Team 2
\item time: 12:30 - 13:30, Mon
\item location: Online
\end{itemize}
					
\textbf{Output:} A lecture is successfully created with attributes above

\textbf{Test Case Derivation:} Verify that a new lecture can be created with all attributes specified

\textbf{How test will be performed:} Create a test case in UTF that creates a new lecture with all information provided, verify the lecture is created correctly
\end{enumerate}

\subsubsection{Pagination and Filter Module}

This section will contain all unit tests for the module ``Pagination and Filter''. The tests are chosen based on common user flows and cover the most important methods of this module. The getter and setter methods are part of the unit test as well, but they will not be included in the section because they are trivial. The Pagination module here has a type T = Pizza, which has two string attributes: toppings and size.

\begin{enumerate}
\item{FRT-M4-1}

\textbf{Name:} Successful pagination creation

\textbf{Type:} Functional, Dynamic, Automatic, Unit
					
\textbf{Initial State:} NA
					
\textbf{Input:} List of Pizza:
\begin{itemize}
\item medium, cheese
\item large, veggie
\end{itemize}
pageCount = 2, filterBy and filterString are null
					
\textbf{Output:} A pagination class is created with following attributes:
\begin{itemize}
\item entryList: List of pizza from input
\item filteredList: same as entryList
\item maxPage: 1
\item currentPage: 1
\item filterBy: null
\item filterString: null
\end{itemize}

\textbf{Test Case Derivation:} Verify that a new pagination object can be created by just specifying a list of T, page count, filter content and filter type

\textbf{How test will be performed:} Create a test case in UTF that creates a new pagination with all information provided, verify the pagination is created correctly with all the attributes expected

\item{FRT-M4-2}

\textbf{Name:} Successful pagination creation with filter 

\textbf{Type:} Functional, Dynamic, Automatic, Unit
					
\textbf{Initial State:} NA
					
\textbf{Input:} List of Pizza:
\begin{itemize}
\item medium, cheese
\item large, veggie
\item medium, pepperoni
\end{itemize}
pageCount = 2, filterBy = `size' and filterString = `medium'
					
\textbf{Output:} A pagination class is created with following attributes:
\begin{itemize}
\item entryList: List of pizza from input
\item filteredList: List of medium pizza only
\item maxPage: 1
\item currentPage: 1
\item filterBy: `size'
\item filterString: `medium'
\end{itemize}

\textbf{Test Case Derivation:} Verify that a new pagination object can be created with a filter

\textbf{How test will be performed:} Create a test case in UTF that creates a new pagination with all information provided, verify the pagination with filter is created correctly with all the attributes expected

\item{FRT-M4-3}

textbf{Name:} Successful pagination addition 

\textbf{Type:} Functional, Dynamic, Automatic, Unit
					
\textbf{Initial State:} List of Pizza:
\begin{itemize}
\item medium, cheese
\item large, veggie
\end{itemize}
pageCount = 2, filterBy and filterString = null
					
\textbf{Input:} New pizza added to the list:
\begin{itemize}
\item medium, pepperoni
\end{itemize}
					
\textbf{Output:} A pagination class is created with following attributes:
\begin{itemize}
\item entryList: List of all three pizza
\item filteredList: same as entryList
\item maxPage: 2
\item currentPage: 1
\item filterBy: null
\item filterString: null
\end{itemize}

\textbf{Test Case Derivation:} Verify that a new object of type T can be added to a pagination object

\textbf{How test will be performed:} Create a test case in UTF that creates a new pagination with all information provided and adds a new object to the class, verify the pagination attributes are updated correctly

\item{FRT-M4-4}

textbf{Name:} Successful pagination deletion 

\textbf{Type:} Functional, Dynamic, Automatic, Unit
					
\textbf{Initial State:} List of Pizza:
\begin{itemize}
\item medium, cheese
\item large, veggie
\item medium, pepperoni
\end{itemize}
pageCount = 2, filterBy and filterString = null
					
\textbf{Input:} New pizza to be deleted from the list:
\begin{itemize}
\item medium, pepperoni
\end{itemize}
					
\textbf{Output:} A pagination class is created with following attributes:
\begin{itemize}
\item entryList: List of two pizza (cheese and veggie)
\item filteredList: same as entryList
\item maxPage: 1
\item currentPage: 1
\item filterBy: null
\item filterString: null
\end{itemize}

\textbf{Test Case Derivation:} Verify that a new object of type T can be removed from a pagination object

\textbf{How test will be performed:} Create a test case in UTF that creates a new pagination with all information provided and deletes an object from the class, verify the pagination attributes are updated correctly

\item{FRT-M4-5}

textbf{Name:} Next page

\textbf{Type:} Functional, Dynamic, Automatic, Unit
					
\textbf{Initial State:} List of Pizza:
\begin{itemize}
\item medium, cheese
\item large, veggie
\item medium, pepperoni
\end{itemize}
pageCount = 2, filterBy and filterString = null
					
\textbf{Input:} Call next page method twice
					
\textbf{Output:} Current page is updated only once:
\begin{itemize}
\item currentPage: 2
\end{itemize}

\textbf{Test Case Derivation:} Verify that the current page increases by one when next page method is called (if the page is not last page)

\textbf{How test will be performed:} Create a test case in UTF that creates a new pagination with all information provided and calls next page twice verify the pagination current page is updated correctly

\item{FRT-M4-6}

textbf{Name:} Previous page

\textbf{Type:} Functional, Dynamic, Automatic, Unit
					
\textbf{Initial State:} List of Pizza:
\begin{itemize}
\item medium, cheese
\item large, veggie
\item medium, pepperoni
\end{itemize}
pageCount = 2, filterBy and filterString = null\\
Call next page method first
					
\textbf{Input:} Call previous page method twice
					
\textbf{Output:} Current page is updated only once:
\begin{itemize}
\item currentPage: 1
\end{itemize}

\textbf{Test Case Derivation:} Verify that the current page reduces by one when previous page method is called (if the page is not first page)

\textbf{How test will be performed:} Create a test case in UTF that creates a new pagination with all information provided and calls previous page twice verify the pagination current page is updated correctly

\item{FRT-M4-7}

textbf{Name:} Last page

\textbf{Type:} Functional, Dynamic, Automatic, Unit
					
\textbf{Initial State:} List of Pizza:
\begin{itemize}
\item medium, cheese
\item large, veggie
\item medium, pepperoni
\end{itemize}
pageCount = 2, filterBy and filterString = null
					
\textbf{Input:} Call last page method
					
\textbf{Output:} Current page is updated:
\begin{itemize}
\item currentPage: 2
\end{itemize}

\textbf{Test Case Derivation:} Verify that the current page is updated to be max page when last page method is called

\textbf{How test will be performed:} Create a test case in UTF that creates a new pagination with all information provided and calls last page method, verify the pagination current page is updated correctly

\item{FRT-M4-8}

textbf{Name:} First page

\textbf{Type:} Functional, Dynamic, Automatic, Unit
					
\textbf{Initial State:} List of Pizza:
\begin{itemize}
\item medium, cheese
\item large, veggie
\item medium, pepperoni
\end{itemize}
pageCount = 2, filterBy and filterString = null\\
Call next page method first
					
\textbf{Input:} Call first page method
					
\textbf{Output:} Current page is updated:
\begin{itemize}
\item currentPage: 1
\end{itemize}

\textbf{Test Case Derivation:} Verify that the current page is updated to be 1 when first page method is called

\textbf{How test will be performed:} Create a test case in UTF that creates a new pagination with all information provided and calls first page method, verify the pagination current page is updated correctly
\end{enumerate}

\subsubsection{Authentication Module}

This section will contain all unit tests for the module ``Authentication''.  This module wraps some methods from third-party library Firebase Authentication, and these methods will be mocked in the unit tests.

\begin{enumerate}
\item{FRT-M5-1}

\textbf{Name:} Get current user email

\textbf{Type:} Functional, Dynamic, Automatic, Unit
					
\textbf{Initial State:} Current user email: `unitTest@test.com'
					
\textbf{Input:} NA
					
\textbf{Output:} The email in the initial state

\textbf{Test Case Derivation:} Verify that the current user is available in the system

\textbf{How test will be performed:} Create a test case in UTF that creates an authentication module and calls the get current user email method with a mock of Firebase Authentication current user

\item{FRT-M5-2}

\textbf{Name:} Is email verified

\textbf{Type:} Functional, Dynamic, Automatic, Unit
					
\textbf{Initial State:} Current user is email verified: true
					
\textbf{Input:} NA
					
\textbf{Output:} The boolean value in initial state

\textbf{Test Case Derivation:} Verify that the application can check email verification state

\textbf{How test will be performed:} Create a test case in UTF that creates an authentication module and calls the get current email verified method with a mock of Firebase Authentication email verification value

\item{FRT-M5-3}

\textbf{Name:} Verify email

\textbf{Type:} Functional, Dynamic, Automatic, Unit
					
\textbf{Initial State:} Current user email: `unitTest@test.com'
					
\textbf{Input:} Call verify email method
					
\textbf{Output:} An email is sent to the email address to verify the email

\textbf{Test Case Derivation:} Verify that the application can send verification email if requested

\textbf{How test will be performed:} Create a test case in UTF that creates an authentication module and calls the verify email method with a mock of Firebase Authentication email verifying method
\end{enumerate}

\subsubsection{Database Module}

This section will contain all unit tests for the module ``Database''.  This module wraps some methods from third-party library Firebase Database, and these methods will be mocked in the unit tests.

\begin{enumerate}
\item{FRT-M6-1}

\textbf{Name:} Get value

\textbf{Type:} Functional, Dynamic, Automatic, Unit
					
\textbf{Initial State:} NA
					
\textbf{Input:} A valid path to a test node: `/events/public/EXPO/name'
					
\textbf{Output:} The corresponding value of the node: EXPO

\textbf{Test Case Derivation:} Verify that the module can read from the database

\textbf{How test will be performed:} Create a test case in UTF that creates a database module and calls the read method with a mock of Firebase database read method

\item{FRT-M6-2}

\textbf{Name:} Set value

\textbf{Type:} Functional, Dynamic, Automatic, Unit
					
\textbf{Initial State:} NA
					
\textbf{Input:} A valid path to a test node and a new value: 
\begin{itemize}
\item path: `/events/public/EXPO/description'
\item value: `EXPO, yes!'
\end{itemize}

\textbf{Output:} The corresponding value changes to the new value

\textbf{Test Case Derivation:} Verify that the module can edit the database

\textbf{How test will be performed:} Create a test case in UTF that creates a database module and calls the set method with a mock of Firebase database set method

\item{FRT-M6-3}

\textbf{Name:} Delete data

\textbf{Type:} Functional, Dynamic, Automatic, Unit
					
\textbf{Initial State:} NA
					
\textbf{Input:} A valid path to a test node : `/events/public/EXPO'

\textbf{Output:} The corresponding event is removed

\textbf{Test Case Derivation:} Verify that the module can delete data from the database

\textbf{How test will be performed:} Create a test case in UTF that creates a database module and calls the delete method with a mock of Firebase database delete method
\end{enumerate}

\subsection{Tests for Nonfunctional Requirements}

This application does not have non-functional requirements to be tested with unit testing.  

\subsection{Traceability Between Test Cases and Modules}
\newpage
\begin{longtable}{|l|cccccc|}
	\caption{Traceability Between Test Cases and Modules}                                                                                             \\
	\hline
	\textbf{Test IDs}  & \multicolumn{6}{c|}{\textbf{Module IDs}}                                                                                     \\
	\hline
	~                  & \textbf{M1}                              & \textbf{M2} & \textbf{M3} & \textbf{M4} & \textbf{M5} & \textbf{M6} \\
	\hline
	\textbf{FRT-M1-1}  & X                                        & ~           & ~           & ~           & ~ & ~ \\
	\textbf{FRT-M1-2}  & X                                        & ~           & ~           & ~           & ~ & ~ \\
	\textbf{FRT-M2-1}  & ~                                        & X           & ~           & ~           & ~  & ~ \\
	\textbf{FRT-M2-2}  & ~                                        & X           & ~           & ~           & ~   & ~\\
	\textbf{FRT-M3-1}  & ~                                        & ~           & X           & ~           & ~   & ~\\
	\textbf{FRT-M3-2}  & ~                                        & ~           & X           & ~           & ~  & ~ \\
    \textbf{FRT-M4-1}  & ~                                        & ~           & ~           & X           & ~   & ~\\
	\textbf{FRT-M4-2}  & ~                                        & ~           & ~           & X           & ~   & ~\\
	\textbf{FRT-M4-3}  & ~                                        & ~           & ~           & X           & ~   & ~\\
	\textbf{FRT-M4-4}  & ~                                        & ~           & ~           & X           & ~   & ~\\
	\textbf{FRT-M4-5}  & ~                                        & ~           & ~           & X           & ~   & ~\\
	\textbf{FRT-M4-6}  & ~                                        & ~           & ~           & X           & ~   & ~\\
	\textbf{FRT-M4-7}  & ~                                        & ~           & ~           & X           & ~   & ~\\
	\textbf{FRT-M4-8}  & ~                                        & ~           & ~           & X           & ~   & ~\\
	\textbf{FRT-M5-1}  & ~                                        & ~           & ~           & ~           & X  & ~ \\
	\textbf{FRT-M5-2}  & ~                                        & ~           & ~           & ~           & X  & ~ \\
	\textbf{FRT-M5-4}  & ~                                        & ~           & ~           & ~           & X  & ~ \\
	\textbf{FRT-M6-1}  & ~                                        & ~           & ~           & ~           & ~  & X \\
	\textbf{FRT-M6-2}  & ~                                        & ~           & ~           & ~           & ~  & X \\
	\textbf{FRT-M6-3}  & ~                                        & ~           & ~           & ~           & ~  & X \\
	\hline
\end{longtable}

\section{Nondynamic Test Plan}

This section summaries all the nondynamic test the team will conduct in section \ref{sec:system} system testing and section \ref{sec:unit} unit testing. Most of the tests will be in the form of code review,  document review or code walkthroughs.
\begin{table}[H]
\caption{Nondynamic Test Plan}
\begin{tabular}{|p{0.3\linewidth} | p{0.5\linewidth}| p{0.2\linewidth} |}
\hline
\multicolumn{1}{|l}{\bfseries Test Method} & \multicolumn{1}{|l|}{\bfseries Description} & \multicolumn{1}{l|}{\bfseries Related Requirement(s)}\\
\hline
Code Inspection for user data & It will be held when the implementation is done with the attendance of the supervisor. It will convince people that the data collection is totally legitimate and happens only with the user's permission & P-SC1, S-P1, COM-L1 \\
\hline
Database Walkthrough & A code walkthrough focusing on database connection and its storage. It will be held when the database implementation is done and convince people that the database is capable of storing all user data and will clean unused data periodically & P-SE2, S-P2\\
\hline
Server \& Data Transmission Walkthrough & A code walkthrough focusing on the server structure. It will be held when the server implementation is done and convince people that the server encrypts data before transmitting and is robust against unexpected crash & P-SC2, P-RF3 \\
\hline
Longevity Peer Evaluation & A peer evaluation session to convince people that the produce can operate for a long time. It will be held when the implementation is done by another developers' team. Though it does not 100\% fulfill the requirements, it builds some confidence & P-L1, P-L2\\
\hline
Code Walkthrough with Primary Reviewers & A code walkthrough focusing on Unity scripts and AR implementation. It will be held after Revision 0 is done and convince our primary reviewers that the a new target building can be added easily with our AR feature & P-SE3\\
\hline
\end{tabular}
\end{table}
\begin{table}[H]
\caption{Nondynamic Test Plan Cont}
\begin{tabular}{|p{0.3\linewidth} | p{0.5\linewidth}| p{0.2\linewidth} |}
\hline
\multicolumn{1}{|l}{\bfseries Test Method} & \multicolumn{1}{|l|}{\bfseries Description} & \multicolumn{1}{l|}{\bfseries Related Requirement(s)}\\
\hline
GitHub Walkthrough with Reviewers & A walkthrough happens when the team has meetings with reviewers. During the walkthrough, the other team confirms that they can read issues the team created and also write new issues. Also, they will confirm there is a clear instruction for users & UH-L1, MS-S1, MS-S2\\
\hline
Document Review & The peer evaluation happens after the deadline for all important documents. During the document review, the other team points out problems and improves document quality & All\\
\hline
Weekly Code Review & The developer will have weekly code reviews with the team for better code quality and formatting & All\\
\hline
TA Review and Feedback & The developer will have meetings with the TA for every deliverable to get some feedback & All\\
\hline
\end{tabular}
\end{table}

\newpage

\section{Appendix}

This is where you can place additional information.

\subsection{Symbolic Parameters}

The definition of the test cases will call for SYMBOLIC\_CONSTANTS.
Their values are defined in this section for easy maintenance.

\begin{table}[H]
\caption{\bf Symbolic Parameter Table}
\begin{tabular}{|p{0.4\linewidth} | p{0.3\linewidth}| p{0.3\linewidth} |}
\hline
\multicolumn{1}{|l}{\bfseries Symbolic Parameter} & \multicolumn{1}{|l|}{\bfseries Description} & \multicolumn{1}{l|}{\bfseries Value}\\
\hline
INITIAL\_USER\_STATE & The default user state with all default user information & All entries empty except email, nickname and password \\
\hline
DEFAULT\_AVATAR & The default virtual avatar & An unisex avatar with default settings \\
\hline
MIN\_SCORE & The passing grade for a category in the survey & 3.5/5\\
\hline
MAX\_CAPACITY & The max number of simultaneous users & 100\\
\hline
SURVEY\_SAMPLE\_SIZE & Sample size of the usability survey & 5\\
\hline
SCREEN\_VIEWPORTS & List of all popular mobile screen sizes & 360X740, 390X844, 820X1180\\
\hline
RECOGNITION\_TIME & Max time the system takes to recognize image & 3 second\\
\hline
LOCATION\_UPDATE\_TIME & Max time the system takes for location to update & 10 second\\
\hline
AR\_ACCURACY & AR camera target recognition success rate & 80\%\\
\hline
PLACEHOLDER & Default value of attributes of type string for data structures in unit testing & NA\\
\hline
DEFAULT\_TIME & Default unix time for events & 1708922625\\
\hline
DEFAULT\_DURATION & Default duration (in minutes) for events & 30\\
\hline
SPECIAL\_CHAR & Suspicious characters that can be used for SQL injection & \textbackslash $\vert$ ! \# \$ \% \& / ( ) = ?  @ £ § € \{ \} . - ; ' ,  \textlangle \textrangle \\
\hline
\end{tabular}
\end{table}

\subsection{Usability Survey}
\label{sec:survey}
Usability testing holds particular significance, especially for a social networking application such as this one. We intend to conduct a usability survey among a select group of students to gather feedback. Initially, we will start with a sample size of 5 to 10 participants. If we consistently receive repetitive suggestions, signalling that we've addressed the majority of issues, we will maintain this sample size. However, if new issues arise or further insights are needed, we will consider conducting another survey with a larger sample size in the future.\\
The survey will be composed of following three session and take about 30 minutes to complete:
\subsubsection{Tasks}
\textbf{The participant will be completing the following tasks. They will use the project as a guest or user or administrator. Each role will have unique tasks and the role is assigned randomly at the start of a test. A task fails if the tester asks for help or spends more than 1 minute to complete a single subtask. This section is to test if the UI and layout is user-friendly and easy-to-understand.}
\begin{enumerate}
\item Lecture
\begin{enumerate}
\item Direct from menu page to lecture page
\item Search for a lecture the tester asked for
\item Check the location of the lecture
\item Pin and then unpin the lecture
\end{enumerate}
\item Event
\begin{enumerate}
\item Direct from map page to event page
\item Search for an event the tester asked for
\item Check the location of the event
\item Pin and then unpin the event
\end{enumerate}
\item Activity Management
\begin{enumerate}
\item Direct from menu page to event/lecture page (tester decide which page they will test)
\item Add a new activity, where the tester will provide details
\end{enumerate}
\item Map
\begin{enumerate}
\item Direct from menu page to map page
\item Zoom in and center the map to show the current location
\end{enumerate}
\item AR Camera
\begin{enumerate}
\item Direct from menu page to AR page
\item Walk around and check if there exist AR object or watch the recording if the participant is not in a target building
\end{enumerate}
\item User Profile
\begin{enumerate}
\item Direct from map page to user profile page
\item Check the program and level of the user
\item Edit any part of the profiles
\item Check the lectures and events that are pinned
\end{enumerate}
\item Friend and Chat
\begin{enumerate}
\item Direct from menu page to friend page
\item \textit{option 1:} Send a request to tester's account
\item \textit{option 2:} Accept a request from tester
\item Open a chat with tester and send a message
\end{enumerate}
\end{enumerate}
\textbf{Reminder: A participant will not do all of the tasks. They will be given a random role at the start of the test and the tasks they will have depend on the role. Administrator will perform task 3, 4, 5, 6, 7; User will perform task 1, 4, 5, 6, 7; Guest will perform task 2, 4, 6.}
\subsubsection{Rating}
\textbf{This section covers tests for some non-functional requirements. The participant will answer the following agree/disagree questions on a rating scale of 5. Here is a breakdown:}
\begin{itemize}
\item 1: Strongly Disagree
\item 2: Disagree
\item 3: Neutral
\item 4: Agree
\item 5: Strongly Agree
\end{itemize}
\begin{enumerate}
	\item \textbf{Immediate Visual Response when Clicking:}  The application provides visual feedback when using different features or switching between pages.
	\item \textbf{Appealing Colour Scheme:} The colour scheme chosen for the interface is similar to McMaster official application.
	\item \textbf{No Technical or Software-specific Language:} The product does not use technical/software-specific language unless necessary.
	\item \textbf{Common periods of usage:} You will be using the application mostly during school time (from 8:30 AM to 6:30 PM).
	\item \textbf{Cultural Friendliness:} The application does not have any words or symbols that are offensive to you.\\\\
\textcolor{red}{\textbf{If given more time, the following questions will be added when conducting a new usability survey}}
	\item \textbf{Customer Satisfaction:} This application will be useful for my daily life and I'd like to recommend it to my friends.
	\item \textbf{Loading Speed:} This application loads/updates what I need in a reasonable time.
\end{enumerate}
\subsubsection{Open-ended Questions}
\textbf{In this section we invite participants to provide their thoughts on the pros and cons of the project. They should feel free to share any feedback or insights.}
\begin{itemize}
\item What do you think is the most difficult to use feature and why?
\item Which feature will you use the most in your daily live and why?
\end{itemize}
\newpage{}
\section*{Appendix --- Reflection}

The information in this section will be used to evaluate the team members on the
graduate attribute of Lifelong Learning.  Please answer the following questions:

\begin{enumerate}
  \item What knowledge and skills will the team collectively need to acquire to
  successfully complete the verification and validation of your project?
  Examples of possible knowledge and skills include dynamic testing knowledge,
  static testing knowledge, specific tool usage etc.  You should look to
  identify at least one item for each team member.
  \item For each of the knowledge areas and skills identified in the previous
  question, what are at least two approaches to acquiring the knowledge or
  mastering the skill?  Of the identified approaches, which will each team
  member pursue, and why did they make this choice?
\end{enumerate}

\begin{itemize}
	\item Static Testing Knowledge - Matthew Miller
	
	One skill that the team will need to complete the verification and validation\
	of the project is static testing knowledge. One way to master this skill is by\
	practicing through code reviews, which involve reading through the code to check\
	for errors. This can be done with our own code or another group's code. Another\
	approach to acquiring this knowledge is by watching tutorials on how to do static\
	testing to help further understanding of the procedure.

	I plan to acquire knowledge of static testing to gain a better understanding of\
	the procedure, as well as putting that knowledge into practice. I chose these\
	approaches because I find that I learn best by applying knowledge in real-world\
	settings.
	
	\item Stress Testing, Apache JMeter - Zihao Du
	
	One of the skill the team needs to acquire for testing purpose is stress testing since the application has a max capacity of users and the team need to check if the server is robust enough to fulfil the requirement. One approach to master this skill is to read their official user manual and tutorial -- as an Apache project, it has a detailed tutorial on the website for beginners. Another approach to acquire the knowledge is to watch some crach course videos about how to use JMeter for stress test like \href{https://www.youtube.com/watch?v=mXGcBvWYl-U}{this}. 
	
	I decide to learn this skill by reading the user manual on their website. It is not because JMeter has a detailed tutorial with step by step screenshot which is very easy to understand, but the official documents can give me a better understanding of the tool itself in general since we need it for both server and database load testing.

\end{itemize}

\begin{itemize}
  \item Automated and Manual Dynamic Testing Knowledge - Michael Kim
  
  For automated dynamic testing knowledge, learning about tools that show code coverage\
  will allow me to identify any aspects of the product that require further testing.\
  I am interested in this as I had many instances where issues previously caught\
  reoccured after a different fix because I had not automated the test cases and\
  neglected them. This will help me set up automated tests anyone can run before\
  pushing to GitHub to check that their code did not break a previously functional\
  segment of the product. Another method of aquiring this skill would code review\
  and documentation to understand what purpose each of the functions serve.\
  This will help me write comprehensive tests and cover all known use cases of the\
  code. I am interested in this method as the product may have other maintainers\
  join in the future, and having these documentations and comments will help in their\
  understanding of the product and continuing the proper maintenance.

  Another aspect of automated dynamic testing that I must learn is how to set up the automated tests and run them all
  For manual dynamic testing knowledge, I will learn about proper usage of Burp Suite,\
  Android Studio Logging tools, Valgrind, and other monitoring and logging applications\
  to help verify the activities of our product. I chose to learn more about this because\
  my experiences in Co-Op as Security Specialist Intern piqued by interest in software testing\
  and verification of security requirements. Through this experience, I have tested applications\
  to see whether they adhere to security requirements imposed by the non-functional requirements.\
  There are various tutorials that can be found on YouTube and extensive documentation provided by\
  the community that uses these tools. This will help significantly in understanding the specific uses\
  of these tools and allow me to test rapidly when a new prototype or update is built. Another approach\
  in learning this technique would be going through bug reports and reproduction steps outlined for other\
  applications. From my Co-Op experience, I had to go through some of these reports so that our products\
  would not experience the same failures other companies have. I am interested in learning more about this\
  process and see what other issues were found and how they were fixed. This will help in grasping the proper\
  procedure of manual dynamic testing and potential failures. There are detailed reports and research on existing\
  products that has experienced significant failures, such as the 2022 Rogers Communications outage. This will also\
  help in prioritization of issues so that critical fixes and failures are patched first.
\end{itemize}

\begin{itemize}
	\item Testing Tools: Unity Test Framework - Waseef Nayeem
	
	One of the specific technologies that will be used for verification and validation is the Unity Test Framework.\
	Unity provides a fully-featured solution for unit testing and also provides tools for automating more complex tests.\
	There is extensive documentation on the UTF on Unity's website as well as several videos and examples that can be used to learn.\
	Another way to learn and master is through experimentation. I could create some basic test assemblies and unit test and modify the inputs or scale up to more complex testable cases.

	Since my role in the project is focused on Unity development I will research and learn how to use the Unity Test Framework.\
	I will accomplish this by quickly learning the basics of UTF and then experimenting early with test cases. I indicated in my previous reflection that\
	software testing is an area I would like to improve in. This will allow me to develop better testing methodologies and improve the quality of future software projects.

\end{itemize}

\begin{itemize}

	\item Object detection domain knowledge - Abhiram Neelamraju \\
	
	Knowledge in the field of object detection is something that could be of great use to the developers. It is something that could help the rest of the team along with developers figure out what is within and what is beyond the scope of the project.
	We can use this information to better plan our features and come up with a realistic expectation for our product. For example, we can research into Vuforia and what its good at and what it might have problems with. \\

	There are many ways to obtain this domain knowledge regarding object detection. There are several free courses available on websites such as coursera, udemy etc. covering relevant content. There are also youtube tutorials that cover similar content especially specific to Vuforia. 
	My personal choice of learning method would be youtube. There are many youtube videos covering basic principles of object detection along with end-to-end project examples implementing Vuforia. This is a great efficient tool to learn from for free.

\end{itemize}

\begin{itemize}

  \item Knowledge of AR Standards - Firas Elayan \\

  It is essential that the team acquires knowledge of existing AR standards, as it will be needed to effectively verify and validate our application. These standards tell us about the quality that our product must have. They can also act as guidelines for creating a user interface that works well and is appealing to users. Knowledge of these standards and applying them to the verification and validation of our software will ensure it follows the best practices and performs at its highest level.
  
  One way to learn this knowledge is to watch tutorial videos. Our supervisor has recommended some videos for beginners, and another way to acquire this domain knowledge is to read documentations of third party library, for example user manual of Unity 3D. I want to learn these standards by watching videos, these crash course videos can help me know the standards quickly and apply what I learned to ensure the quality of the application is the best it could be.\\\\

\end{itemize}

\end{document}