\documentclass[12pt, titlepage]{article}

\usepackage{booktabs}
\usepackage{tabularx}
\usepackage{hyperref}
\hypersetup{
    colorlinks,
    citecolor=blue,
    filecolor=black,
    linkcolor=red,
    urlcolor=blue
}
\usepackage[round]{natbib}
\usepackage{longtable}

\input{../Comments}
%% Common Parts

\newcommand{\progname}{Software Engineering} % PUT YOUR PROGRAM NAME HERE
\newcommand{\authname}{Team \#2, Team Name
\\ Zihao Du 
\\ Matthew Miller
\\ Firas Elayan
\\ Abhiram Neelamraju
\\ Michael Kim} % AUTHOR NAMES                  

\usepackage{hyperref}
    \hypersetup{colorlinks=true, linkcolor=blue, citecolor=blue, filecolor=blue,
                urlcolor=blue, unicode=false}
    \urlstyle{same}
                                


\begin{document}

\title{Project Title: System Verification and Validation Plan for \progname{}} 
\author{\authname}
\date{\today}
	
\maketitle

\pagenumbering{roman}

\section*{Revision History}

\begin{tabularx}{\textwidth}{p{3cm}p{2cm}X}
\toprule {\bf Date} & {\bf Version} & {\bf Notes}\\
\midrule
Oct 29 & 1.0 & Add Functional Requirements Tests\\
Oct 31 & 1.0 & Add Non-Functional Requirement Tests\\
Nov 1 & 1.0 & Add Unit Tests\\
\bottomrule
\end{tabularx}

\newpage

\tableofcontents

\listoftables

\newpage

\section{Symbols, Abbreviations, and Acronyms}

\renewcommand{\arraystretch}{1.2}
\begin{tabular}{l l} 
  \toprule		
  \textbf{symbol} & \textbf{description}\\
  \midrule 
  AR & Augmented Reality\\
  \midrule
  SRS & Software Requirement Specification\\
  \midrule
  CAPTCHA & Human Verification Test\\
  \midrule
  MIS & Module Interface Specification\\
  \midrule
  Vuforia & AR software development kit\\
  \midrule
  UTF & Unity Test Framework\\
  \midrule
  GPS & Global Positioning System\\
  \bottomrule
\end{tabular}\\

\newpage

\pagenumbering{arabic}

This document ... \wss{provide an introductory blurb and roadmap of the
  Verification and Validation plan}

\section{General Information}

\subsection{Summary}
  
The software being tested is a social media application which also uses Augmented Reality features to allow McMaster University students to connect with each other. The software has many different functions to be able to accomplish its goals. These are some of the important functions:

 \begin{itemize}
     \item  Allowing the users to create and manage their account.

     \item  Allowing the users to add, manage and message friends.

     \item  Allowing the users with administrator access to add and manage events.

     \item  Allowing the users to view ongoing events and pin and unpin them based on interest.

     \item  Identifying major campus buildings through the user's location and camera view, and using the information to display relevant lectures, events and their timings.

 \end{itemize}

 

\subsection{Objectives}

In order to develop the software to meet its goals and create a complete product, there are many qualities that are important for our project such as:

\begin{itemize}

    \item  Correctness of the software in achieving its designated functions

    \item An easily navigable User Interface adaptable to different screen sizes

    \item Sufficient speed and reliability of the building recognition features

    \item Sufficient speed and reliability in sending and receiving messages

    \item Adequate accuracy in tracking the user's location

    \item Appropriate abidance of security requirements regarding critical information such as location and personal data.

    \item Infrastructure to support a passable scale of total concurrent users and total user data storage

    \end{itemize}



While we would want to satisfy as many positive qualities as possible, due to limited resources , there will be some objectives that we will be leaving out since they are out of scope such as :

\begin{itemize}
    \item  Project longevity requirements:

    This is because the longevity requirements were written with the assumption that the product will be supported even after the duration of the course for years to come which is something that is not possible to verify or guarantee and is out of the scope.

    \item  Testing learning requirements:

    We will be producing material to make it easier to understand and use our product but to test every piece of supporting documentation and tutorial content would require long testing sessions with volunteers which can be out of the scope. Instead, since we will be testing our user interface, we could use the feedback obtained there to improve our learning specific content as well.
    
\end{itemize}



We will be prioritizing reliability in our listed qualities over performance since we want our software to be correct and consistent over fast and unreliable. While we still want to maximize performance where possible, it is difficult to predict exactly how well we will meet our performance criteria since we will be limited by many factors out of our control such as external libraries.

We will also be assuming that any external libraries we depend on (such as Vuforia) will have already been verified by the implementation team due to our lack of control over them. 


\subsection{Relevant Documentation}





Many of the other documents associated with our project are relevant to the Verification and Validation plan such as:

\begin{itemize}
    \item Software Requirement Specification (SRS):
          
          The SRS has a lot of components relevant to the VnV plan. The functional requirements are important in identifying the primary functions of the the project while the non functional requirements are critical in identifying the main qualities and objectives of the product. It also contains constraint and assumption information which can help identify out-of-scope objectives.

          \citet{SRS}
    \item Hazard Analysis (HA):
    
        The HA document contains the Failure Mode and Effect Analysis (FMEA) which is incredibly useful in understanding the possible failures and their consequences. This can help us assign priorities to different failures that can be translated into priorities for different qualities identified earlier in the document. It also contains more detailed safety and security requirements that we can use to verify our project.

        \citet{HA}
\end{itemize}


\section{Plan}

\wss{Introduce this section.   You can provide a roadmap of the sections to
  come.}

\subsection{Verification and Validation Team}

\wss{Your teammates.  Maybe your supervisor.
  You should do more than list names.  You should say what each person's role is
  for the project's verification.  A table is a good way to summarize this information.}

\subsection{SRS Verification Plan}

\wss{List any approaches you intend to use for SRS verification.  This may include
  ad hoc feedback from reviewers, like your classmates, or you may plan for 
  something more rigorous/systematic.}

\wss{Maybe create an SRS checklist?}

\subsection{Design Verification Plan}

\wss{Plans for design verification}

\wss{The review will include reviews by your classmates}

\wss{Create a checklists?}

\subsection{Verification and Validation Plan Verification Plan}

\wss{The verification and validation plan is an artifact that should also be
verified.  Techniques for this include review and mutation testing.}

\wss{The review will include reviews by your classmates}

\wss{Create a checklists?}

\subsection{Implementation Verification Plan}

\wss{You should at least point to the tests listed in this document and the unit
  testing plan.}

\wss{In this section you would also give any details of any plans for static
  verification of the implementation.  Potential techniques include code
  walkthroughs, code inspection, static analyzers, etc.}

\subsection{Automated Testing and Verification Tools}

\wss{What tools are you using for automated testing.  Likely a unit testing
  framework and maybe a profiling tool, like ValGrind.  Other possible tools
  include a static analyzer, make, continuous integration tools, test coverage
  tools, etc.  Explain your plans for summarizing code coverage metrics.
  Linters are another important class of tools.  For the programming language
  you select, you should look at the available linters.  There may also be tools
  that verify that coding standards have been respected, like flake9 for
  Python.}

\wss{If you have already done this in the development plan, you can point to
that document.}

\wss{The details of this section will likely evolve as you get closer to the
  implementation.}

\subsection{Software Validation Plan}

\wss{If there is any external data that can be used for validation, you should
  point to it here.  If there are no plans for validation, you should state that
  here.}

\wss{You might want to use review sessions with the stakeholder to check that
the requirements document captures the right requirements.  Maybe task based
inspection?}

\wss{For those capstone teams with an external supervisor, the Rev 0 demo should 
be used as an opportunity to validate the requirements.  You should plan on 
demonstrating your project to your supervisor shortly after the scheduled Rev 0 demo.  
The feedback from your supervisor will be very useful for improving your project.}

\wss{For teams without an external supervisor, user testing can serve the same purpose 
as a Rev 0 demo for the supervisor.}

\wss{This section might reference back to the SRS verification section.}

\section{System Test Description}
	
\subsection{Tests for Functional Requirements}

The following section includes system tests for functional requirements defined in the \href{https://github.com/beatlepie/4G06CapstoneProjectTeam2/blob/main/docs/SRS-Volere/SRS.pdf}{SRS} document. In order to cover all the functional requirements, these subsections match exactly the subsections in the SRS document, which cover all the different components of this application. These tests will ensure all requirements are fulfilled and the application works as we expected. Most of the following tests will be run manually while some of them will be automated.

\subsubsection{Pre-Game Settings}

This section includes all test cases related to what users can do before they start to use the application. This will include the requirement for the consent form and user privacy protection (FR-1-1). In order to plan the game, the user must agree to the terms and conditions in the consent form.

\begin{enumerate}

\item{\textbf{FRT-PG1}}

\textbf{Name:} Agree To Consent Form

\textbf{Control:} Manual
					
\textbf{Initial State:} The user is not logged in to the application, and all features in the application are not accessible to the user. A consent form appears asking for access to the device and permission to collect user data

\textbf{Input:} The user agrees to all the terms and conditions and clicks 'Agree'
					
\textbf{Output:} The user is redirected to the login screen

\textbf{Test Case Derivation:} It is a must to request users' consent before collecting their information or using any hardware component
					
\textbf{How test will be performed:} The tester will first clear the app cache and data. Then the tester will run the application, accept the consent form and verify login screen shows up

\textbf{Related Requirement(s):} FR-1-1
					
\item{\textbf{FRT-PG2}}

\textbf{Name:} Disagree To Consent Form

\textbf{Control:} Manual
					
\textbf{Initial State:} The user is not logged in to the application, and all features in the application are not accessible to the user. A consent form appears asking for access to the device and permission to collect user data
					
\textbf{Input:} The user rejects the terms and conditions and clicks 'Disagree'
					
\textbf{Output:} The user is not redirected to the login screen

\textbf{Test Case Derivation:} It is a must to request users' consent before collecting their information or using any hardware component

\textbf{How test will be performed:} The tester will first clear the app cache and data. Then the tester will run the application, reject the consent form and verify login screen does not show up

\textbf{Related Requirement(s):} FR-1-1

\end{enumerate}

\subsubsection{User Account}

This section includes all test cases related to user accounts. This will include the requirements for account creation (FR-2-1), login (FR-2-3), deletion (FR-2-2) and email verification(FR-2-8), along with user avatar settings (FR-2-6, FR-2-7). User accounts and virtual avatars are necessary for all users who want to use this application.

\begin{enumerate}

\item{\textbf{FRT-UA1}}

\textbf{Name:} Successful Account Creation

\textbf{Control:} Manual
					
\textbf{Initial State:} The user does not have an account and is not logged in to the application

\textbf{Input:} All information needed to create an account
					
\textbf{Output:} An Account with corresponding information is created in the database with the account initialized to INITIAL\_USER\_STATE

\textbf{Test Case Derivation:}  User account operations are the foundation of the application. All functionalities work only with an existing account. Also, it is essential to ensure the account information reflects the input information when creating accounts
					
\textbf{How test will be performed:} The tester will create an account with all information and verify that the account can be logged into

\textbf{Related Requirement(s):} FR-2-1

\item{\textbf{FRT-UA2}}

\textbf{Name:} Unsuccessful Account Creation

\textbf{Control:} Manual
					
\textbf{Initial State:} The user does not have an account and is not logged in to the application

\textbf{Input:} All information needed to create an account, including an existing username
					
\textbf{Output:} Account creation fails with an error message telling the user the username already exists

\textbf{Test Case Derivation:} Username is the identifier of a user account, which should be unique
					
\textbf{How test will be performed:} The tester will create an account with an existing username and verify that the creation fails

\textbf{Related Requirement(s):} FR-2-1

\item{\textbf{FRT-UA3}}

\textbf{Name:} Successful Account Login

\textbf{Control:} Automated
					
\textbf{Initial State:} The user has an account and is not logged in to the application

\textbf{Input:} Username and valid password
					
\textbf{Output:} User successfully logs into the application

\textbf{Test Case Derivation:} User account operations are the foundation of the application. All functionalities work only with an existing account. Also, it is essential to verify that users can log in with only the correct password 
					
\textbf{How test will be performed:} The tester will create an automated test that inputs a valid password and verifies that the login is completed

\textbf{Related Requirement(s):} FR-2-3

\item{\textbf{FRT-UA4}}

\textbf{Name:} Unsuccessful Account Login

\textbf{Control:} Automated
					
\textbf{Initial State:} The user has an account and is not logged in to the application

\textbf{Input:} Username and wrong password
					
\textbf{Output:} Login fails with an error message telling the user the password is wrong

\textbf{Test Case Derivation:} User account operations are the foundation of the application. All functionalities work only with an existing account. Also, it is essential to verify that users can log in with only the correct password
					
\textbf{How test will be performed:} The tester will create an automated test that inputs an invalid password and verifies that a corresponding error is returned

\textbf{Related Requirement(s):} FR-2-3

\item{\textbf{FRT-UA5}}

\textbf{Name:} Account Deletion

\textbf{Control:} Manual
					
\textbf{Initial State:} The user has an account and is logged into the application

\textbf{Input:} Corresponding user request
					
\textbf{Output:} Data pertaining to the given username is deleted

\textbf{Test Case Derivation:} User account operations are the foundation of the application. When users want to quit, they should be able to delete all related information in the application by deleting their account
					
\textbf{How test will be performed:} The tester will delete the test account and verify the account does not exist anymore in-game

\textbf{Related Requirement(s):} FR-2-2

\item{\textbf{FRT-UA6}}

\textbf{Name:} Reset Account Password

\textbf{Control:} Automated
					
\textbf{Initial State:} The user has an account

\textbf{Input:} New password and answers to security questions for password recovery
					
\textbf{Output:} Password is successfully reset

\textbf{Test Case Derivation:} In case the user misplaces or forgets the password, they can still get their account back
					
\textbf{How test will be performed:} The tester will first reset the password. Then the tester will try to log in with the new password and verify the new password works properly

\textbf{Related Requirement(s):} FR-2-4

\item{\textbf{FRT-UA7}}

\textbf{Name:} Human Verification Test

\textbf{Control:} Manual
					
\textbf{Initial State:} The user does not have an account and is not logged into the application

\textbf{Input:} All information needed to create an account
					
\textbf{Output:} The tester passes the test and creates an account successfully

\textbf{Test Case Derivation:} To prevent malicious automated systems from creating bot accounts
					
\textbf{How test will be performed:} The tester will verify the CAPTCHA test appears and functions well when creating an account

\textbf{Related Requirement(s):} FR-2-5

\item{\textbf{FRT-UA8}}

\textbf{Name:} Avatar Creation and Modification

\textbf{Control:} Manual
					
\textbf{Initial State:} The user has an account with INITIAL\_AVATAR

\textbf{Input:} Avatar and corresponding user request
					
\textbf{Output:} User creates an avatar and changes it

\textbf{Test Case Derivation:} To help users enjoy social networking when using this application, users shall be able to personalize their account, therefore it is necessary to verify they can create an avatar and change it whenever they want
					
\textbf{How test will be performed:} The tester will create an avatar from the initial one and modify it. Then the tester will verify the changes are visible for all friends from another test account

\textbf{Related Requirement(s):} FR-2-6, FR-2-7

\item{\textbf{FRT-UA9}}

\textbf{Name:} Email verification

\textbf{Control:} Manual
					
\textbf{Initial State:} The user adds McMaster email to user profile

\textbf{Input:} User verifying after receving the verification email
					
\textbf{Output:} User email is verified and all access is granted for the account

\textbf{Test Case Derivation:} The user has to be identified as a McMaster student to have enough access to all functionalities (Event and Lecture information). So it is essential to test the email verification functionality
					
\textbf{How test will be performed:} The tester will add a test McMaster email to the test account and verify the email can be verified and all functionalities work for the account after email verification

\textbf{Related Requirement(s):} FR-2-8
\end{enumerate}

\subsubsection{Social Networking System}

This section includes all test cases related to the social networking system. This will include the requirements for adding (FR-3-1), deleting (FR-3-2), messaging friends of the user (FR-3-3, FR-3-4) and sharing location (FR-3-5). Expanding student social network is the most significant motivation of this application and it is necessary to verify all associated requirements are fully fulfilled.

\begin{enumerate}
\item{\textbf{FRT-SN1}}

\textbf{Name:} Successful Friend Request

\textbf{Control:} Manual
					
\textbf{Initial State:} The user is logged in

\textbf{Input:} A valid username and corresponding user request
					
\textbf{Output:} A Request is sent to the given user

\textbf{Test Case Derivation:} Users should be able to make friends on this social media platform by searching for names and sending requests
					
\textbf{How test will be performed:} The tester will first send a friend request to a test account. Then the tester will verify that test account receives a friend request

\textbf{Related Requirement(s):} FR-3-1

\item{\textbf{FRT-SN2}}

\textbf{Name:} Friend Request Acceptance

\textbf{Control:} Manual
					
\textbf{Initial State:} A friend request was sent

\textbf{Input:} User accepts the request
					
\textbf{Output:} Two users are added to each other's friend lists

\textbf{Test Case Derivation:} Users should be able to expand networking by accepting friend requests
					
\textbf{How test will be performed:} The tester will first accept the request. Then the tester will verify that the two accounts become friends of each other

\textbf{Related Requirement(s):} FR-3-1

\item{\textbf{FRT-SN3}}

\textbf{Name:} Friend Request Rejection

\textbf{Control:} Manual
					
\textbf{Initial State:} A friend request was sent

\textbf{Input:} User rejects the request
					
\textbf{Output:} The request is declined and no friend is added to the list

\textbf{Test Case Derivation:} Users should be able to decline friend requests from strangers
					
\textbf{How test will be performed:} The tester will first reject the request. Then the tester will verify that no friendship relation is generated between the two accounts

\textbf{Related Requirement(s):} FR-3-1

\item{\textbf{FRT-SN4}}

\textbf{Name:} Friend Deletion

\textbf{Control:} Manual
					
\textbf{Initial State:} A friend exists in the friend list

\textbf{Input:} User deletes the chosen friend
					
\textbf{Output:} The corresponding friend is deleted from the list

\textbf{Test Case Derivation:} Users should be able to remove friends from their list to make space for new friends
					
\textbf{How test will be performed:} The tester will delete a test friend account and verify the friend is removed from the list

\textbf{Related Requirement(s):} FR-3-2

\item{\textbf{FRT-SN5}}

\textbf{Name:} Friend Messaging

\textbf{Control:} Manual
					
\textbf{Initial State:} A friend exists in the friend list

\textbf{Input:} Audio/text message to send
					
\textbf{Output:} The corresponding message is sent to the friend

\textbf{Test Case Derivation:} Chatting with friends is the core functionality of expanding social networks. Friends should be able to send and receive messages from each other
					
\textbf{How test will be performed:} The tester will send an audio and text message to another test friend account. Then the tester will verify the other account received the correct message

\textbf{Related Requirement(s):} FR-3-3, FR-3-4

\item{\textbf{FRT-SN6}}

\textbf{Name:} Friend Location Sharing

\textbf{Control:} Manual
					
\textbf{Initial State:} A friend exists in the friend list

\textbf{Input:} User Location
					
\textbf{Output:} User location is visible to the corresponding friend

\textbf{Test Case Derivation:} Users shall be able to share their current location with friends to meet in person. It is essential to verify the location is accurate and up-to-date
					
\textbf{How test will be performed:} The tester will share the location with a friend, and then verify the location is visible and accurate (accuracy of GPS\_ACCURACY) from the other test account

\textbf{Related Requirement(s):} FR-3-5
\end{enumerate}

\subsubsection{AR Campus}

This section includes all test cases related to campus navigation with AR technology. This will include the requirements for building recognition (FR-4-1) and demonstration building information (FR-4-2, FR-4-3). Augmented reality provides an immersive user experience and it is the unique selling point of the application.

\begin{enumerate}
\item{\textbf{FRT-AR1}}

\textbf{Name:} Successful Building Recognition

\textbf{Control:} Manual
					
\textbf{Initial State:} User looks at a building on campus

\textbf{Input:} Clear Camera view
					
\textbf{Output:} The building is recognized and its name is given

\textbf{Test Case Derivation:} Building recognition is the essential functionality for AR technology applied in this application. The building should be recognizable from a certain angle when users look at their camera
					
\textbf{How test will be performed:} The tester will walk around a building on campus and look into the camera, verifying the building is recognized and a list of events/lectures is displayed on the screen

\textbf{Related Requirement(s):} FR-4-1, FR-4-2, FR-4-3

\item{\textbf{FRT-AR2}}

\textbf{Name:} Unsuccessful Building Recognition

\textbf{Control:} Manual
					
\textbf{Initial State:} User looks at a building off-campus

\textbf{Input:} Camera view
					
\textbf{Output:} The building is not recognized correctly

\textbf{Test Case Derivation:} Building recognition is the essential functionality for AR technology applied in this application. The building should be recognizable from a certain angle when users look at their camera
					
\textbf{How test will be performed:} The tester will walk around a building that is not on campus and look into the camera, verifying the building is not recognized or the system recognizing it as a school building

\textbf{Related Requirement(s):} FR-4-1, FR-4-2, FR-4-3
\end{enumerate}

\subsubsection{Lectures and Events}

This section includes all test cases related to events and lectures happening on campus. This will include the requirements for events and lectures themselves, users interacting with these properties, and the power of administration accounts (All FR-5 requirements in the SRS document). The test cases ensure that the lectures and events information is accurate and helpful for all users.

\begin{enumerate}
\item{\textbf{FRT-LE1}}

\textbf{Name:} Interest/Disinterest Event

\textbf{Control:} Manual
					
\textbf{Initial State:} An event exists in a specific building

\textbf{Input:} Corresponding user request
					
\textbf{Output:} The event with all necessary information is added/removed from the user's event list

\textbf{Test Case Derivation:} The user should be able to interact with events available on campus and share their activities with friends. Therefore, they should be able to switch between 'interest' and 'disinterest' states for all events 
					
\textbf{How test will be performed:} The tester will pin and unpin the event and verify the event shows up and disappears from the user's interested event list. Then the tester will verify the event has all associated information including club/department, location and time

\textbf{Related Requirement(s):} FR-5-1, FR-5-2, FR-5-7

\item{\textbf{FRT-LE2}}

\textbf{Name:} Pin/unpin Lectures

\textbf{Control:} Manual
					
\textbf{Initial State:} A lecture with all necessary information exists in a specific building

\textbf{Input:} Corresponding user request
					
\textbf{Output:} The lecture is added/removed from the user's lecture list

\textbf{Test Case Derivation:} The user should be able to interact with lectures shown in the app and share their schedule with friends. Therefore, they should be able to switch between 'pinned' and 'unpinned' states for all lectures 
					
\textbf{How test will be performed:} The tester will pin and unpin the lecture and verify the lecture shows up and disappears from the user's pinned lecture list. Then the tester will verify the event has all associated information including instructor, location and time

\textbf{Related Requirement(s):} FR-5-3, FR-5-4, FR-5-8

\item{\textbf{FRT-LE3}}

\textbf{Name:} Administrator Add Events

\textbf{Control:} Manual
					
\textbf{Initial State:} User is logged in as an admin

\textbf{Input:} Sample event with all necessary information
					
\textbf{Output:} The event is posted on the building and is available for all users to pin

\textbf{Test Case Derivation:} Administrators work as a source of truth in this application, therefore they shall be able to create new events for all users
					
\textbf{How test will be performed:} The tester will log in as an administrator and add the sample event. Then the tester will verify the event shown in the corresponding building is available for all users

\textbf{Related Requirement(s):} FR-5-5

\item{\textbf{FRT-LE4}}

\textbf{Name:} Administrator Change Events

\textbf{Control:} Manual
					
\textbf{Initial State:} User is logged in as an admin

\textbf{Input:} New event information and corresponding user request
					
\textbf{Output:} The posted event information is updated

\textbf{Test Case Derivation:} Administrators work as a source of truth in this application, therefore they shall be able to update event information to get users noticed
					
\textbf{How test will be performed:} The tester will log in as an administrator and update the sample event. Then the tester will verify the event shown in the corresponding building is updated for all users

\textbf{Related Requirement(s):} FR-5-5

\item{\textbf{FRT-LE5}}

\textbf{Name:} Administrator Delete Events

\textbf{Control:} Manual
					
\textbf{Initial State:} User is logged in as an admin

\textbf{Input:} Existing event and corresponding user request
					
\textbf{Output:} The event is deleted

\textbf{Test Case Derivation:} Administrators work as a source of truth in this application, therefore they shall be able to clean out-of-date event
					
\textbf{How test will be performed:} The tester will log in as an administrator and delete the sample event. Then the tester will verify the event shown in the corresponding building is deleted for all users 

\textbf{Related Requirement(s):} FR-5-5

\item{\textbf{FRT-LE6}}

\textbf{Name:} Administrator Add Lectures

\textbf{Control:} Manual
					
\textbf{Initial State:} User is logged in as an admin

\textbf{Input:} Sample lecture with all necessary information
					
\textbf{Output:} The lecture is posted on the building and is available for all users to pin

\textbf{Test Case Derivation:} Administrators work as a source of truth in this application, therefore they shall be able to create new lectures for all users when new semesters come
					
\textbf{How test will be performed:} The tester will log in as an administrator and add the sample lecture. Then the tester will verify the lecture shown in the corresponding building is available for all users 

\textbf{Related Requirement(s):} FR-5-6

\item{\textbf{FRT-LE7}}

\textbf{Name:} Administrator Change Lecture

\textbf{Control:} Manual
					
\textbf{Initial State:} User is logged in as an admin

\textbf{Input:} New lecture information and corresponding user request
					
\textbf{Output:} The posted lecture information is updated

\textbf{Test Case Derivation:} Administrators work as a source of truth in this application, therefore they shall be able to update lecture information to get users noticed
					
\textbf{How test will be performed:} The tester will log in as an administrator and update the sample lecture. Then the tester will verify the lecture shown in the corresponding building is updated for all users

\textbf{Related Requirement(s):} FR-5-6

\item{\textbf{FRT-LE8}}

\textbf{Name:} Administrator Delete Lectures

\textbf{Control:} Manual
					
\textbf{Initial State:} User is logged in as an admin

\textbf{Input:} Existing lecture and corresponding user request
					
\textbf{Output:} The lecture is deleted

\textbf{Test Case Derivation:} Administrators work as a source of truth in this application, therefore they shall be able to clean out-of-date lectures after each semester
					
\textbf{How test will be performed:} The tester will log in as an administrator and delete the sample lecture. Then the tester will verify the lecture shown in the corresponding building is deleted for all users

\textbf{Related Requirement(s):} FR-5-6
\end{enumerate}

\subsection{Tests for Nonfunctional Requirements}
\label{sec:nonfunctional}

The following section includes tests for non-functional requirements defined in the \href{https://github.com/beatlepie/4G06CapstoneProjectTeam2/blob/main/docs/SRS-Volere/SRS.pdf}{SRS} document. Areas of testing follow the subsections of requirements in the previous document, which mainly include Look and Feel, Usability and Humanity, Performance and Security.\\ Usability requirements will be tested by asking users to do a survey, whose content can be found in section \ref{sec:survey}. Most of the tests here are dynamic and will be done manually or automatically, but there are some tests that need non-dynamic testing like peer code review or code walkthrough.

\subsubsection{Look and Feel}

\begin{enumerate}
\item{\textbf{NFRT-LF1}}

\textbf{Name:} Survey for feedback on application layout

\textbf{Type:} Non-functional, Dynamic, Manual
					
\textbf{Initial State:} User has an account

\textbf{Input/Condition:} User is logged into the homepage and pokes around the application
					
\textbf{Output/Result:} Get feedback and verify layout is user-friendly. This test is a pass if average individual score is over MIN\_SCORE in the "User-friendly Layout",\
``Intuitive icons'', ``Immediate Visual Response when Clicking'' and ``Appealing Colour Scheme'' categories of the User Experience Survey
					
\textbf{How test will be performed:} A survey will be given to at least
SURVEY\_SAMPLE\_SIZE users where they will give feedback to different elements of the interface. All of the survey takers are selected randomly from McMaster University

\textbf{Related Requirement(s):} LF-A1, LF-A2, LF-S2

\item{\textbf{NFRT-LF2}}

\textbf{Name:} Visual inspection of interface

\textbf{Type:} Non-functional, Dynamic, Manual
					
\textbf{Initial State:} User has the application downloaded on their phone
					
\textbf{Input/Condition:} User has the application open on their phone
					
\textbf{Output/Result:} The test passes if, for all different pages:
\begin{itemize}
  \item The colour scheme is the same
  \item All visual elements on the screen are within the borders of the screen for all screens in the SCREEN\_VIEWPORTS list
\end{itemize}

\textbf{How test will be performed:} The application will be opened on all screens in the SCREEN\_VIEWPORTS list.\
The testers will visually inspect each page and each screen to make sure that:
\begin{itemize}
  \item All visual elements on the screen are within its borders
  \item The colour scheme on each page is the same
\end{itemize}

\textbf{Related Requirement(s):} LF-A3, LF-S1

\end{enumerate}

\subsubsection{Usability and Humanity}

\begin{enumerate}
\item{\textbf{NFRT-UH1}}

\textbf{Name:} Survey for feedback on application layout

\textbf{Type:} Non-functional, Dynamic, Manual
					
\textbf{Initial State:} User has an account

\textbf{Input/Condition:} User is logged into the homepage and pokes around the application
					
\textbf{Output/Result:} Get feedback and verify layout is user-friendly. This test is a pass if average individual score is over MIN\_SCORE in the ``Easy to Navigate'',\
``Helpful Tutorial'' and ``No Technical or Software-Specific Language'' categories of the User Experience Survey
					
\textbf{How test will be performed:} A survey will be given to at least
SURVEY\_SAMPLE\_SIZE users where they will give feedback to different elements of the interface. All of the survey takers are selected randomly from McMaster University

\textbf{Related Requirement(s):} UH-EOU1, UH-L2, UH-UP1

\item{\textbf{NFRT-UH2}}

\textbf{Name:} Visual inspection of interface

\textbf{Type:} Non-functional, Dynamic, Manual
					
\textbf{Initial State:} User has the application downloaded on their phone
					
\textbf{Input/Condition:} User has the application open on their phone
					
\textbf{Output/Result:} The test passes if:
\begin{itemize}
  \item The name, time, and location of all pinned events and lectures are displayed on a user's personal page
  \item The user is able to update their USER\_PROFILE and avatar
  \item A tutorial explaining the features of the application appears on first launch, and is available upon request from the user
\end{itemize}

\textbf{How test will be performed:} The testers will open the application on one device and visually inspect the application to check that the above conditions are satisfied.

\textbf{Related Requirement(s):} UH-PI1, UH-PI2, UH-L1

\item{\textbf{NFRT-UH3}}

\textbf{Name:} Unit Test for Colour Contrast

\textbf{Type:} Non-functional, Static, Automatic
					
\textbf{Initial State:} There exists a method that calculates the luminance of colours on the screen\
and calculates the contrast between the brightest and darkest colours.

\textbf{Input/Condition:} All colours on each page are given as input

\textbf{Output/Result:} The colour contrast for each page is calculated to be at least 4.5:1

\textbf{How test will be performed:} There will be a unit test that takes all the colours on each page as an input,\
calculates the contrast, and checks if that contrast is at least 4.5:1

\textbf{Related Requirement(s):} UH-A1

\end{enumerate}

\subsubsection{Performance}

\begin{enumerate}
\item{\textbf{NFRT-P1}}

\textbf{Name:} Visual inspection of interface for messages

\textbf{Type:} Non-functional, Dynamic, Manual
					
\textbf{Initial State:} User has the application downloaded on their phone
					
\textbf{Input/Condition:} User has the application open on their phone
					
\textbf{Output/Result:} The test passes if:
\begin{itemize}
  \item A notification in the application appears when the network strength is below ~75dBm and/or the server connection is poor
  \item An error message stating that there is no internet connection is displayed when the application fails to connect to the internet.
  \item A message telling the user to be aware of their surroundings is displayed upon startup.
  \item When using a device that is incompatible with AR, a warning message is displayed relaying that information to the user.
  \item When the device is not connected to the internet, the product is able to provide rudimentary functionalities using its offline components.
\end{itemize}

\textbf{How test will be performed:} The testers will open the application on one device and visually inspect the application to check that the above conditions are satisfied.

\textbf{Related Requirement(s):} P-RF1, P-RF3, P-RF4, P-RF5, P-RF6

\item{\textbf{NFRT-P2}}

\textbf{Name:} Speed and Latency Tests

\textbf{Type:} Non-functional, Dynamic, Automatic
					
\textbf{Initial State:} The product is launched on two separate devices

\textbf{Input/Condition:} The product is logged in to one account on one device, and another account on the other. Both accounts are friends

\textbf{Output/Result:}
\begin{itemize}
  \item The popup response following successful image recognitions appears within RECOGNITION\_TIME
  \item A user receives a message within MESSAGE\_TIME after being sent by another user
  \item A user's updated location must appear on their friends' device within LOCATION\_UPDATE\_TIME of the location change
  \item A user receives a message within MESSAGE\_TIME after being sent by another user when there are issues collecting location information
\end{itemize}

\textbf{How test will be performed:} The code has a built-in timer to verify the timing of the above

\textbf{Related Requirement(s):} P-SL1, P-SL2, P-SL3, P-RF2

\item{\textbf{NFRT-P3}}

\textbf{Name:} Code Inspection

\textbf{Type:} Non-functional, Static, Manual
					
\textbf{Initial State:} The code for all the related requirements exists

\textbf{Input/Condition:} The testers have the code open

\textbf{Output/Result:}
\begin{itemize}
  \item User's personal information does not appear in the database if the user did not grant permission
  \item The usage of a user's persona information by the product abides by the Privacy Act, The Personal Information Protection and Electronic Documents Act, and Canada and Ontario's data protection laws
  \item The product does not execute any code that involves the transmission of information outside of the product
  \item Information for an event added to the database is still present at least 3 months after its addition
  \item The product architecture allows for the addition of new buildings without considerable drawbacks in performance
  \item The product is able to operate without major malfunctions in release build for at least 1 year
  \item The finalized product will remain compatible with the promised operating systems and devices for at least 3 years
\end{itemize}

\textbf{How test will be performed:} The code will be inspected by developers and supervisors to check for the above

\textbf{Related Requirement(s):} P-SC1, P-SC3, P-SC4, P-C2, P-SE3, P-L1, P-L2

\item{\textbf{NFRT-P4}}

\textbf{Name:} Building Identification

\textbf{Type:} Non-functional, Dynamic, Manual
					
\textbf{Initial State:} The user has the product launched on their device

\textbf{Input/Condition:} The product is scanning a building on campus

\textbf{Output/Result:} The product identifies a building with a success rate of at least 80 percent

\textbf{How test will be performed:} Testers will scan buildings with the product multiple times and calculate the percentage of successful scans

\textbf{Related Requirement(s):} P-PA1

\item{\textbf{NFRT-P5}}

\textbf{Name:} Load Testing for Server Capacity

\textbf{Type:} Non-functional, Dynamic, Automatic
					
\textbf{Initial State:} The server and database are online and ready to connect to

\textbf{Input/Condition:} There are MAX\_CAPACITY users simultaneously accessing the server. There is information for MAX\_USER\_LOAD users stored in the database.

\textbf{Output/Result:} The server is able to handle all the users connecting to the server at once. The database can handle all the data that it contains

\textbf{How test will be performed:} Testers use a J-meter for the test

\textbf{Related Requirement(s):} P-C1, P-SE1, P-SE2

\end{enumerate}

\subsubsection{Operational and Environmental}

\begin{enumerate}
\item{\textbf{NFRT-OE1}}

\textbf{Name:} Visual inspection of device

\textbf{Type:} Non-functional, Manual

\textbf{Initial State:} User has a phone that uses Android 11 or above

\textbf{Input/Condition:} User's phone is turned on and unlocked

\textbf{Output/Result:} The test passes if:
\begin{itemize}
  \item The product can be downloaded onto the phone from the Google Play Store, or by downloading the APK file directly
\end{itemize}

\textbf{How test will be performed:} The testers will check the Google Play Store for the product, or find an APK file for the product.\
The testers will then download and install the product onto the phone from the Google Play Store or the APK file.

\textbf{Related Requirement(s):} OE-P1

\end{enumerate}

\subsubsection{Maintainability and Support}

\quad \textbf{N/A}

\subsubsection{Security}

\begin{enumerate}
\item{\textbf{NFRT-S1}}

\textbf{Name:} Access Test
  
\textbf{Type:} Non-functional, Dynamic, Manual
            
\textbf{Initial State:} The product is downloaded onto a supported device
  
\textbf{Input/Condition:} The product is launched
            
\textbf{Output/Result:} At each level of access, the product constrains the possible actions to what is specified in requirement S-A1.
  
\textbf{How test will be performed:}
\begin{enumerate}
  \item The tester will download the product to a supported device and launch the product.
  \item The product will begin at the first level of access, which should only allow the user to view the login, account creation, and account recovery pages.
  \item The tester will log in or create an account to move to the second level of access (McMaster student/faculty member).
  \item The tester will test that actions only possible at the third level of access (adding/deleting/editing events, pull logs not accessible to users) are not possible to do at this level.
  \item The tester will log out of the account and log in as an administrator, moving to the third level of access.
  \item The tester will test that actions only possible at this level of access (adding/deleting/editing events, pull logs not accessible to users) are possible to do.
\end{enumerate}

\textbf{Related Requirement(s):} S-A1

\item{\textbf{NFRT-S2}}

\textbf{Name:} Unit Testing

\textbf{Type:} Non-functional, Static, Automatic
					
\textbf{Initial State:} The code for all the related requirements exists

\textbf{Input/Condition:} The unit tests for all related requirements exist. These unit tests check that the code satisfies the fit crieria for their associated requirement(s).

\textbf{Output/Result:} The following tests pass:
\begin{itemize}
  \item The product prevents users from creating a user account using student/faculty-exclusive information tied to an already existing account
  \item All data collected must be encrypted both on disk in the server and on transit by an industry standard encryption algorithm
  \item The product erases all data for a specific user if that user requests to do so
  \item Accounts that are inactive for a certain period of time are deleted to prevent unnecessary data from being held
\end{itemize}

\textbf{How test will be performed:} Unit tests will be written to test for the above

\textbf{Related Requirement(s):} S-IG1, S-P1, S-P2, S-P4, S-P5

\item{\textbf{NFRT-S3}}

\textbf{Name:} Visual inspection of device

\textbf{Type:} Non-functional, Dynamic, Manual

\textbf{Initial State:} User has the product downloaded and launched on their device

\textbf{Input/Condition:} The user chooses to create an account

\textbf{Output/Result:} The product asks the user whether or not they agree with the privacy terms

\textbf{How test will be performed:} The testers will launch the product and begin to create a new account

\textbf{Related Requirement(s):} S-P3

\end{enumerate}

\subsubsection{Cultural}

\begin{enumerate}
\item{\textbf{NFRT-CUL1}}

\textbf{Name:} Visual inspection of device

\textbf{Type:} Non-functional, Dynamic, Manual

\textbf{Initial State:} User has the product downloaded on their device

\textbf{Input/Condition:} The user launches the app

\textbf{Output/Result:} The product does not display any language or symbols deemed offensive by any marginalized, reigious, or ethnic groups

\textbf{How test will be performed:} The testers will launch the product and visit each page to check for offensive langauge or symbols

\textbf{Related Requirement(s):} CUL-C1, CUL-C2

\end{enumerate}

\subsubsection{Compliance}

\quad \textbf{N/A}

\subsection{Traceability Between Test Cases and Requirements}

\wss{Provide a table that shows which test cases are supporting which
  requirements.}

\section{Unit Test Description}

This section includes all unit tests for functional and non-functional requirements. Though most of the requirements will be tested manually, unit testing will still be utilized for some basic functionalities of all components. Since the team has not completed a detailed design document, this section cannot refer to all modules in \href{https://github.com/beatlepie/4G06CapstoneProjectTeam2/blob/main/docs/Design/SoftDetailedDes/MIS.pdf}{MIS}. Instead it will refer to a list of core modules to be implemented:
\begin{itemize}
	\item \textbf{M1} User
	\item \textbf{M2} Friend
	\item \textbf{M3} Building
	\item \textbf{M4} Event
	\item \textbf{M5} Lecture
\end{itemize}
It is the first draft of the implementation design, and a more detailed version of unit test will be added once the MIS is completed.

\subsection{Unit Testing Scope}

All modules defined above are within the testing scope.  The scope of unit testing was limited to the components that could be automatically tested. The AR module is out of the scope since it is largely composed of a third-parity library, Vuforia. It is a sophisticated AR software development kit, so we assumed that Vuforia’s logic is correct and therefore shall not be tested.

\subsection{Tests for Functional Requirements}

The following sections details unit tests for functional requirements. It is an essential aspect of testing as it verifies whether the modules are behaving
correctly given the requirements in the SRS.

\subsubsection{User Module}

This section will contain all unit tests for module ``User''. The tests are chosen based on common user flows and cover the most important methods of this module.

\begin{enumerate}

\item{FRT-M1-1}

\textbf{Name:} Password Setting

\textbf{Type:} Functional, Dynamic, Automatic, Unit
					
\textbf{Initial State:} There is an old password for an account
					
\textbf{Input:} New password
					
\textbf{Output:} The old password is replaced by the new one

\textbf{Test Case Derivation:} Verify the password reset functionality works well

\textbf{How test will be performed:} Create a test case in UTF that sets a new password for a user and verify the password has been changed

\item{FRT-M1-2}

\textbf{Name:} Search For Valid Username

\textbf{Type:} Functional, Dynamic, Automatic, Unit
					
\textbf{Initial State:} There are some valid users during the test
					
\textbf{Input:} One of the valid usernames
					
\textbf{Output:} The user is found from the list by its username

\textbf{Test Case Derivation:} Verify that user can search for a user by username

\textbf{How test will be performed:} Create a test case in UTF that searches for a existing username and verify the corresponding user is found

\item{FRT-M1-3}

\textbf{Name:} Search For Invalid Username

\textbf{Type:} Functional, Dynamic, Automatic, Unit
					
\textbf{Initial State:} There are some valid users during the test
					
\textbf{Input:} Invalid usernames
					
\textbf{Output:} No user is found

\textbf{Test Case Derivation:} Verify that user can search for a user by username

\textbf{How test will be performed:} Create a test case in UTF that searches for a non-existing username and verify no user is found

\item{FRT-M1-4}

\textbf{Name:} Get Location

\textbf{Type:} Functional, Dynamic, Automatic, Unit
					
\textbf{Initial State:} User agrees to share location
					
\textbf{Input:} User request
					
\textbf{Output:} Geographic position of the device with accuracy of GPS\_ACCURACY

\textbf{Test Case Derivation:} Verify that user can share their location accurately

\textbf{How test will be performed:} Create a test case in UTF that searches for a non-existing username and verify no user is found
\end{enumerate}

\subsubsection{Friend Module}

This section will contain all unit tests for module ``Friend''. The tests are chosen based on common user flows and cover the most important methods of this module.

\begin{enumerate}
\item{FRT-M2-1}

\textbf{Name:} Send new message

\textbf{Type:} Functional, Dynamic, Automatic, Unit
					
\textbf{Initial State:} There are two accounts being friends
					
\textbf{Input:} A message sent friend to the friend
					
\textbf{Output:} The Message is added to chat history

\textbf{Test Case Derivation:} Verify that messages can be sent and stored between friends

\textbf{How test will be performed:} Create a test case in UTF that messages a friend and verify the message is added to chat history

\item{FRT-M2-2}

\textbf{Name:} Receive new message

\textbf{Type:} Functional, Dynamic, Automatic, Unit
					
\textbf{Initial State:} There are two accounts being friends
					
\textbf{Input:} A message sent from the friend
					
\textbf{Output:} The Message is added to chat history

\textbf{Test Case Derivation:} Verify that messages can be sent and stored between friends

\textbf{How test will be performed:} Create a test case in UTF that receives a message from a friend and verify the message is added to chat history
\end{enumerate}

\subsubsection{Building Module}

This section will contain all unit tests for module ``Building''. The tests are chosen based on common user flows and cover the most important methods of this module.

\begin{enumerate}
\item{FRT-M3-1}

\textbf{Name:} List Events

\textbf{Type:} Functional, Dynamic, Automatic, Unit
					
\textbf{Initial State:} There are some events for this target building
					
\textbf{Input:} User request
					
\textbf{Output:} All events in the building are listed

\textbf{Test Case Derivation:} Verify that all events of the building are shown with nothing missing

\textbf{How test will be performed:} Create a test case in UTF where a user shares the location, verify the location information retrieved matches the expected GPS coordinates

\item{FRT-M3-2}

\textbf{Name:} List Lectures

\textbf{Type:} Functional, Dynamic, Automatic, Unit
					
\textbf{Initial State:} There are some lectures for this target building
					
\textbf{Input:} User request
					
\textbf{Output:} All lectures in the building are listed

\textbf{Test Case Derivation:} Verify that all lectures of the building are shown with nothing missing

\textbf{How test will be performed:} Create a test case in UTF that has a building with a list of lectures, verify all lectures are displayed once lecture information is requested

\item{FRT-M3-3}

\textbf{Name:} Check User Location in Bounds

\textbf{Type:} Functional, Dynamic, Automatic, Unit

\textbf{Initial State:} Building has a bounding box outlined by GPS coordinates

\textbf{Input:} User location
					
\textbf{Output:} True if the user location is within the bounding box, False otherwise

\textbf{Test Case Derivation:} Verify that the GPS building detection algorithm is correct

\textbf{How test will be performed:} Create a test case in UTF that has a building object with a bounding box and a user object with a dummy location, verify that the algorithm identifies cases where the user is within the bounding box

\end{enumerate}

\subsubsection{Event Module}

This section will contain all unit tests for module ``Event''. The tests are chosen based on common user flows and cover the most important methods of this module.

\begin{enumerate}
\item{FRT-M4-1}

\textbf{Name:} Successful New Event Creation

\textbf{Type:} Functional, Dynamic, Automatic, Unit
					
\textbf{Initial State:} User is logged in as admin
					
\textbf{Input:} Event information including location, time and club/department
					
\textbf{Output:} Event is successfully added to the system

\textbf{Test Case Derivation:} Verify that new event can be added by an administrator only if all information is provided

\textbf{How test will be performed:} Create a test case in UTF that creates a new event with all information provided, verify the event is added to the event list

\item{FRT-M4-2}

\textbf{Name:} Unsuccessful New Event Creation

\textbf{Type:} Functional, Dynamic, Automatic, Unit
					
\textbf{Initial State:} User is logged in as admin
					
\textbf{Input:} Event information with one of the following missing: location, time or club/department
					
\textbf{Output:} Event cannot be added to the system, an error pops up saying information missing

\textbf{Test Case Derivation:} Verify that new event can be added by an administrator only if all information is provided

\textbf{How test will be performed:} Create a test case in UTF that creates a new event with some necessary information missing, verify the event fails to be added to the event list
\end{enumerate}

\subsubsection{Lecture Module}

This section will contain all unit tests for module ``Lecture''. The tests are chosen based on common user flows and cover the most important methods of this module.

\begin{enumerate}
\item{FRT-M5-1}

\textbf{Name:} Successful New Lecture Creation

\textbf{Type:} Functional, Dynamic, Automatic, Unit
					
\textbf{Initial State:} User is logged in as admin
					
\textbf{Input:} Lecture information including location, time and instructor
					
\textbf{Output:} Lecture is successfully added to the system

\textbf{Test Case Derivation:} Verify that new lecture can be added by an administrator only if all information is provided

\textbf{How test will be performed:} Create a test case in UTF that creates a new lecture with all information provided, verify the lecture is added to the lecture list

\item{FRT-M5-2}

\textbf{Name:} Unsuccessful New Lecture Creation

\textbf{Type:} Functional, Dynamic, Automatic, Unit
					
\textbf{Initial State:} User is logged in as admin
					
\textbf{Input:} Lecture information with one of the following missing: location, time or instructor
					
\textbf{Output:} Lecture cannot be added to the system, an error pops up saying information missing

\textbf{Test Case Derivation:} Verify that new lecture can be added by an administrator only if all information is provided

\textbf{How test will be performed:} Create a test case in UTF that creates a new lecture with some necessary information missing, verify the lecture fails to be added to the lecture list
\end{enumerate}

\subsection{Tests for Nonfunctional Requirements}

This application does not have non-functional requirements to be tested with unit testing. All non-functional requirements will be verified through system tests in section \ref{sec:nonfunctional}. Some of the access and robustness requirements will be tested along with other functional requirement in the section above.

\subsection{Traceability Between Test Cases and Modules}

\begin{longtable}{|l|ccccc|}
	\caption{Traceability Between Test Cases and Modules}                                                                                             \\
	\hline
	\textbf{Test IDs}  & \multicolumn{5}{c|}{\textbf{Module IDs}}                                                                                     \\
	\hline
	~                  & \textbf{M1}                              & \textbf{M2} & \textbf{M3} & \textbf{M4} & \textbf{M5} \\
	\hline
	\textbf{FRT-M1-1}  & X                                        & ~           & ~           & ~           & ~ \\
	\textbf{FRT-M1-2}  & X                                        & ~           & ~           & ~           & ~ \\
	\textbf{FRT-M1-3}  & X                                        & ~           & ~           & ~           & ~ \\
	\textbf{FRT-M1-4}  & X                                        & ~           & ~           & ~           & ~ \\
	\textbf{FRT-M2-1}  & ~                                        & X           & ~           & ~           & ~ \\
	\textbf{FRT-M2-2}  & ~                                        & X           & ~           & ~           & ~ \\
	\textbf{FRT-M3-1}  & ~                                        & ~           & X           & ~           & ~ \\
	\textbf{FRT-M3-2}  & ~                                        & ~           & X           & ~           & ~ \\
	\textbf{FRT-M3-3}  & ~                                        & ~           & X           & ~           & ~ \\
  \textbf{FRT-M4-1}  & ~                                        & ~           & ~           & X           & ~ \\
	\textbf{FRT-M4-2}  & ~                                        & ~           & ~           & X           & ~ \\
	\textbf{FRT-M5-1}  & ~                                        & ~           & ~           & ~           & X \\
	\textbf{FRT-M5-2}  & ~                                        & ~           & ~           & ~           & X \\
	\hline
\end{longtable}
				
\bibliographystyle{plainnat}

\bibliography{../../refs/References}

\newpage

\section{Appendix}

This is where you can place additional information.

\subsection{Symbolic Parameters}

The definition of the test cases will call for SYMBOLIC\_CONSTANTS.
Their values are defined in this section for easy maintenance.

\begin{table}[h]
\caption{\bf Symbolic Parameter Table}
\begin{tabular}{|p{0.4\linewidth} | p{0.3\linewidth}| p{0.3\linewidth} |}
\hline
\multicolumn{1}{|l}{\bfseries Symbolic Parameter} & \multicolumn{1}{|l|}{\bfseries Description} & \multicolumn{1}{l|}{\bfseries Value}\\
\hline
INITIAL\_USER\_STATE & The default user state with all default user information & All entries empty except username and password \\
\hline
INITIAL\_AVATAR & The default virtual avatar & An unisex virtual avatar with default settings \\
\hline
MIN\_SCORE & The passing grade for a category in the survey & 7/10\\
\hline
SURVEY\_SAMPLE\_SIZE & Size of the user experience survey & 50\\
\hline
GPS\_ACCURACY & Accuracy of location sharing & 25 meter\\
\hline
SCREEN\_VIEWPORTS & List of all popular mobile screen sizes & 360X740, 390X844, 820X1180\\
\hline
USER\_PROFILE & All information about the user & username, password, gender, age, email, area of study\\
\hline
\end{tabular}
\end{table}

\subsection{Usability Survey Questions?}
\label{sec:survey}

\wss{This is a section that would be appropriate for some projects.}

\newpage{}
\section*{Appendix --- Reflection}

The information in this section will be used to evaluate the team members on the
graduate attribute of Lifelong Learning.  Please answer the following questions:

\newpage{}
\section*{Appendix --- Reflection}

\wss{This section is not required for CAS 741}

The information in this section will be used to evaluate the team members on the
graduate attribute of Lifelong Learning.  Please answer the following questions:

\begin{enumerate}
  \item What knowledge and skills will the team collectively need to acquire to
  successfully complete the verification and validation of your project?
  Examples of possible knowledge and skills include dynamic testing knowledge,
  static testing knowledge, specific tool usage etc.  You should look to
  identify at least one item for each team member.
  \item For each of the knowledge areas and skills identified in the previous
  question, what are at least two approaches to acquiring the knowledge or
  mastering the skill?  Of the identified approaches, which will each team
  member pursue, and why did they make this choice?
\end{enumerate}

\end{document}