\documentclass[12pt, titlepage]{article}

\usepackage{booktabs}
\usepackage{tabularx}
\usepackage{hyperref}
\hypersetup{
    colorlinks,
    citecolor=blue,
    filecolor=black,
    linkcolor=red,
    urlcolor=blue
}
\usepackage[round]{natbib}

\input{../Comments}
%% Common Parts

\newcommand{\progname}{Software Engineering} % PUT YOUR PROGRAM NAME HERE
\newcommand{\authname}{Team \#2, Team Name
\\ Zihao Du 
\\ Matthew Miller
\\ Firas Elayan
\\ Abhiram Neelamraju
\\ Michael Kim} % AUTHOR NAMES                  

\usepackage{hyperref}
    \hypersetup{colorlinks=true, linkcolor=blue, citecolor=blue, filecolor=blue,
                urlcolor=blue, unicode=false}
    \urlstyle{same}
                                


\begin{document}

\title{Project Title: System Verification and Validation Plan for \progname{}} 
\author{\authname}
\date{\today}
	
\maketitle

\pagenumbering{roman}

\section*{Revision History}

\begin{tabularx}{\textwidth}{p{3cm}p{2cm}X}
\toprule {\bf Date} & {\bf Version} & {\bf Notes}\\
\midrule
Oct 29 & 1.0 & Add Functional Requirements Tests\\
Date 2 & 1.1 & Notes\\
\bottomrule
\end{tabularx}

~\\
\wss{The intention of the VnV plan is to increase confidence in the software.
However, this does not mean listing every verification and validation technique
that has ever been devised.  The VnV plan should also be a \textbf{feasible}
plan. Execution of the plan should be possible with the time and team available.
If the full plan cannot be completed during the time available, it can either be
modified to ``fake it'', or a better solution is to add a section describing
what work has been completed and what work is still planned for the future.}

\wss{The VnV plan is typically started after the requirements stage, but before
the design stage.  This means that the sections related to unit testing cannot
initially be completed.  The sections will be filled in after the design stage
is complete.  the final version of the VnV plan should have all sections filled
in.}

\newpage

\tableofcontents

\listoftables
\wss{Remove this section if it isn't needed}

\listoffigures
\wss{Remove this section if it isn't needed}

\newpage

\section{Symbols, Abbreviations, and Acronyms}

\renewcommand{\arraystretch}{1.2}
\begin{tabular}{l l} 
  \toprule		
  \textbf{symbol} & \textbf{description}\\
  \midrule 
  T & Test\\
  \bottomrule
\end{tabular}\\

\wss{symbols, abbreviations, or acronyms --- you can simply reference the SRS
  \citep{SRS} tables, if appropriate}

\wss{Remove this section if it isn't needed}

\newpage

\pagenumbering{arabic}

This document ... \wss{provide an introductory blurb and roadmap of the
  Verification and Validation plan}

\section{General Information}

\subsection{Summary}

\wss{Say what software is being tested.  Give its name and a brief overview of
  its general functions.}

\subsection{Objectives}

\wss{State what is intended to be accomplished.  The objective will be around
  the qualities that are most important for your project.  You might have
  something like: ``build confidence in the software correctness,''
  ``demonstrate adequate usability.'' etc.  You won't list all of the qualities,
  just those that are most important.}

\wss{You should also list the objectives that are out of scope.  You don't have 
the resources to do everything, so what will you be leaving out.  For instance, 
if you are not going to verify the quality of usability, state this.  It is also 
worthwhile to justify why the objectives are left out.}

\wss{The objectives are important because they highlight that you are aware of 
limitations in your resources for verification and validation.  You can't do everything, 
so what are you going to prioritize?  As an example, if your system depends on an 
external library, you can explicitly state that you will assume that external library 
has already been verified by its implementation team.}

\subsection{Relevant Documentation}

\wss{Reference relevant documentation.  This will definitely include your SRS
  and your other project documents (design documents, like MG, MIS, etc).  You
  can include these even before they are written, since by the time the project
  is done, they will be written.}

\citet{SRS}

\wss{Don't just list the other documents.  You should explain why they are relevant and 
how they relate to your VnV efforts.}

\section{Plan}

\wss{Introduce this section.   You can provide a roadmap of the sections to
  come.}

\subsection{Verification and Validation Team}

\wss{Your teammates.  Maybe your supervisor.
  You should do more than list names.  You should say what each person's role is
  for the project's verification.  A table is a good way to summarize this information.}

\subsection{SRS Verification Plan}

\wss{List any approaches you intend to use for SRS verification.  This may include
  ad hoc feedback from reviewers, like your classmates, or you may plan for 
  something more rigorous/systematic.}

\wss{Maybe create an SRS checklist?}

\subsection{Design Verification Plan}

\wss{Plans for design verification}

\wss{The review will include reviews by your classmates}

\wss{Create a checklists?}

\subsection{Verification and Validation Plan Verification Plan}

\wss{The verification and validation plan is an artifact that should also be
verified.  Techniques for this include review and mutation testing.}

\wss{The review will include reviews by your classmates}

\wss{Create a checklists?}

\subsection{Implementation Verification Plan}

\wss{You should at least point to the tests listed in this document and the unit
  testing plan.}

\wss{In this section you would also give any details of any plans for static
  verification of the implementation.  Potential techniques include code
  walkthroughs, code inspection, static analyzers, etc.}

\subsection{Automated Testing and Verification Tools}

\wss{What tools are you using for automated testing.  Likely a unit testing
  framework and maybe a profiling tool, like ValGrind.  Other possible tools
  include a static analyzer, make, continuous integration tools, test coverage
  tools, etc.  Explain your plans for summarizing code coverage metrics.
  Linters are another important class of tools.  For the programming language
  you select, you should look at the available linters.  There may also be tools
  that verify that coding standards have been respected, like flake9 for
  Python.}

\wss{If you have already done this in the development plan, you can point to
that document.}

\wss{The details of this section will likely evolve as you get closer to the
  implementation.}

\subsection{Software Validation Plan}

\wss{If there is any external data that can be used for validation, you should
  point to it here.  If there are no plans for validation, you should state that
  here.}

\wss{You might want to use review sessions with the stakeholder to check that
the requirements document captures the right requirements.  Maybe task based
inspection?}

\wss{For those capstone teams with an external supervisor, the Rev 0 demo should 
be used as an opportunity to validate the requirements.  You should plan on 
demonstrating your project to your supervisor shortly after the scheduled Rev 0 demo.  
The feedback from your supervisor will be very useful for improving your project.}

\wss{For teams without an external supervisor, user testing can serve the same purpose 
as a Rev 0 demo for the supervisor.}

\wss{This section might reference back to the SRS verification section.}

\section{System Test Description}
	
\subsection{Tests for Functional Requirements}

The following section includes system tests for functional requirements defined in the \href{https://github.com/beatlepie/4G06CapstoneProjectTeam2/blob/main/docs/SRS-Volere/SRS.pdf}{SRS} document. In order to cover all the functional requirements, these subsections match exactly the subsections in the SRS document, which cover all the different components of this application. These tests will ensure all requirements are fulfilled and the application works as we expected. Most of the following tests will be run manually while some of them will be automated.

\subsubsection{Pre-Game Settings}

This section includes all test cases related to what users can do before they start to use the application. This will include the requirement for the consent form and user privacy protection (FR-1-1). In order to plan the game, the user must agree to the terms and conditions in the consent form.

\begin{enumerate}

\item{\textbf{FRT-PG1}}

\textbf{Name:} Agree To Consent Form

\textbf{Control:} Manual
					
\textbf{Initial State:} The user is not logged in to the application, and all features in the application are not accessible to the user. A consent form appears asking for access to the device and permission to collect user data

\textbf{Input:} The user agrees to all the terms and conditions and clicks 'Agree'
					
\textbf{Output:} The user is redirected to the login screen

\textbf{Test Case Derivation:} It is a must to request users' consent before collecting their information or using any hardware component
					
\textbf{How test will be performed:} The tester will first clear the app cache and data. Then the tester will run the application, accept the consent form and verify login screen shows up

\textbf{Related Requirement(s):} FR-1-1
					
\item{\textbf{FRT-PG2}}

\textbf{Name:} Disagree To Consent Form

\textbf{Control:} Manual
					
\textbf{Initial State:} The user is not logged in to the application, and all features in the application are not accessible to the user. A consent form appears asking for access to the device and permission to collect user data
					
\textbf{Input:} The user rejects the terms and conditions and clicks 'Disagree'
					
\textbf{Output:} The user is not redirected to the login screen

\textbf{Test Case Derivation:} It is a must to request users' consent before collecting their information or using any hardware component

\textbf{How test will be performed:} The tester will first clear the app cache and data. Then the tester will run the application, reject the consent form and verify login screen does not show up

\textbf{Related Requirement(s):} FR-1-1

\end{enumerate}

\subsubsection{User Account}

This section includes all test cases related to user accounts. This will include the requirements for account creation (FR-2-1), login (FR-2-3), deletion (FR-2-2) and email verification(FR-2-8), along with user avatar settings (FR-2-6, FR-2-7). User accounts and virtual avatars are necessary for all users who want to use this application.

\begin{enumerate}

\item{\textbf{FRT-UA1}}

\textbf{Name:} Successful Account Creation

\textbf{Control:} Manual
					
\textbf{Initial State:} The user does not have an account and is not logged in to the application

\textbf{Input:} All information needed to create an account
					
\textbf{Output:} An Account with corresponding information is created in the database with the account initialized to INITIAL\_USER\_STATE

\textbf{Test Case Derivation:}  User account operations are the foundation of the application. All functionalities work only with an existing account. Also, it is essential to ensure the account information reflects the input information when creating accounts
					
\textbf{How test will be performed:} The tester will create an account with all information and verify that the account can be logged into

\textbf{Related Requirement(s):} FR-2-1

\item{\textbf{FRT-UA2}}

\textbf{Name:} Unsuccessful Account Creation

\textbf{Control:} Manual
					
\textbf{Initial State:} The user does not have an account and is not logged in to the application

\textbf{Input:} All information needed to create an account, including an existing username
					
\textbf{Output:} Account creation fails with an error message telling the user the username already exists

\textbf{Test Case Derivation:} Username is the identifier of a user account, which should be unique
					
\textbf{How test will be performed:} The tester will create an account with an existing username and verify that the creation fails

\textbf{Related Requirement(s):} FR-2-1

\item{\textbf{FRT-UA3}}

\textbf{Name:} Successful Account Login

\textbf{Control:} Automated
					
\textbf{Initial State:} The user has an account and is not logged in to the application

\textbf{Input:} Username and valid password
					
\textbf{Output:} User successfully logs into the application

\textbf{Test Case Derivation:} User account operations are the foundation of the application. All functionalities work only with an existing account. Also, it is essential to verify that users can log in with only the correct password 
					
\textbf{How test will be performed:} The tester will create an automated test that inputs a valid password and verifies that the login is completed

\textbf{Related Requirement(s):} FR-2-3

\item{\textbf{FRT-UA4}}

\textbf{Name:} Unsuccessful Account Login

\textbf{Control:} Automated
					
\textbf{Initial State:} The user has an account and is not logged in to the application

\textbf{Input:} Username and wrong password
					
\textbf{Output:} Login fails with an error message telling the user the password is wrong

\textbf{Test Case Derivation:} User account operations are the foundation of the application. All functionalities work only with an existing account. Also, it is essential to verify that users can log in with only the correct password
					
\textbf{How test will be performed:} The tester will create an automated test that inputs an invalid password and verifies that a corresponding error is returned

\textbf{Related Requirement(s):} FR-2-3

\item{\textbf{FRT-UA5}}

\textbf{Name:} Account Deletion

\textbf{Control:} Manual
					
\textbf{Initial State:} The user has an account and is logged into the application

\textbf{Input:} Corresponding user request
					
\textbf{Output:} Data pertaining to the given username is deleted

\textbf{Test Case Derivation:} User account operations are the foundation of the application. When users want to quit, they should be able to delete all related information in the application by deleting their account
					
\textbf{How test will be performed:} The tester will delete the test account and verify the account does not exist anymore in-game

\textbf{Related Requirement(s):} FR-2-2

\item{\textbf{FRT-UA6}}

\textbf{Name:} Reset Account Password

\textbf{Control:} Automated
					
\textbf{Initial State:} The user has an account

\textbf{Input:} New password and answers to security questions for password recovery
					
\textbf{Output:} Password is successfully reset

\textbf{Test Case Derivation:} In case the user misplaces or forgets the password, they can still get their account back
					
\textbf{How test will be performed:} The tester will first reset the password. Then the tester will try to log in with the new password and verify the new password works properly

\textbf{Related Requirement(s):} FR-2-4

\item{\textbf{FRT-UA7}}

\textbf{Name:} Human Verification Test

\textbf{Control:} Manual
					
\textbf{Initial State:} The user does not have an account and is not logged into the application

\textbf{Input:} All information needed to create an account
					
\textbf{Output:} The tester passes the test and creates an account successfully

\textbf{Test Case Derivation:} To prevent malicious automated systems from creating bot accounts
					
\textbf{How test will be performed:} The tester will verify the CAPTCHA test appears and functions well when creating an account

\textbf{Related Requirement(s):} FR-2-5

\item{\textbf{FRT-UA8}}

\textbf{Name:} Avatar Creation and Modification

\textbf{Control:} Manual
					
\textbf{Initial State:} The user has an account with INITIAL\_AVATAR

\textbf{Input:} Avatar and corresponding user request
					
\textbf{Output:} User creates an avatar and changes it

\textbf{Test Case Derivation:} To help users enjoy social networking when using this application, users shall be able to personalize their account, therefore it is necessary to verify they can create an avatar and change it whenever they want
					
\textbf{How test will be performed:} The tester will create an avatar from the initial one and modify it. Then the tester will verify the changes are visible for all friends from another test account

\textbf{Related Requirement(s):} FR-2-6, FR-2-7

\item{\textbf{FRT-UA9}}

\textbf{Name:} Email verification

\textbf{Control:} Manual
					
\textbf{Initial State:} The user adds McMaster email to user profile

\textbf{Input:} User verifying after receving the verification email
					
\textbf{Output:} User email is verified and all access is granted for the account

\textbf{Test Case Derivation:} The user has to be identified as a McMaster student to have enough access to all functionalities (Event and Lecture information). So it is essential to test the email verification functionality
					
\textbf{How test will be performed:} The tester will add a test McMaster email to the test account and verify the email can be verified and all functionalities work for the account after email verification

\textbf{Related Requirement(s):} FR-2-8
\end{enumerate}

\subsubsection{Social Networking System}

This section includes all test cases related to the social networking system. This will include the requirements for adding (FR-3-1), deleting (FR-3-2), messaging friends of the user (FR-3-3, FR-3-4) and sharing location (FR-3-5). Expanding student social network is the most significant motivation of this application and it is necessary to verify all associated requirements are fully fulfilled.

\begin{enumerate}
\item{\textbf{FRT-SN1}}

\textbf{Name:} Successful Friend Request

\textbf{Control:} Manual
					
\textbf{Initial State:} The user is logged in

\textbf{Input:} A valid username and corresponding user request
					
\textbf{Output:} A Request is sent to the given user

\textbf{Test Case Derivation:} Users should be able to make friends on this social media platform by searching for names and sending requests
					
\textbf{How test will be performed:} The tester will first send a friend request to a test account. Then the tester will verify that test account receives a friend request

\textbf{Related Requirement(s):} FR-3-1

\item{\textbf{FRT-SN2}}

\textbf{Name:} Friend Request Acceptance

\textbf{Control:} Manual
					
\textbf{Initial State:} A friend request was sent

\textbf{Input:} User accepts the request
					
\textbf{Output:} Two users are added to each other's friend lists

\textbf{Test Case Derivation:} Users should be able to expand networking by accepting friend requests
					
\textbf{How test will be performed:} The tester will first accept the request. Then the tester will verify that the two accounts become friends of each other

\textbf{Related Requirement(s):} FR-3-1

\item{\textbf{FRT-SN3}}

\textbf{Name:} Friend Request Rejection

\textbf{Control:} Manual
					
\textbf{Initial State:} A friend request was sent

\textbf{Input:} User rejects the request
					
\textbf{Output:} The request is declined and no friend is added to the list

\textbf{Test Case Derivation:} Users should be able to decline friend requests from strangers
					
\textbf{How test will be performed:} The tester will first reject the request. Then the tester will verify that no friendship relation is generated between the two accounts

\textbf{Related Requirement(s):} FR-3-1

\item{\textbf{FRT-SN4}}

\textbf{Name:} Friend Deletion

\textbf{Control:} Manual
					
\textbf{Initial State:} A friend exists in the friend list

\textbf{Input:} User deletes the chosen friend
					
\textbf{Output:} The corresponding friend is deleted from the list

\textbf{Test Case Derivation:} Users should be able to remove friends from their list to make space for new friends
					
\textbf{How test will be performed:} The tester will delete a test friend account and verify the friend is removed from the list

\textbf{Related Requirement(s):} FR-3-2

\item{\textbf{FRT-SN5}}

\textbf{Name:} Friend Messaging

\textbf{Control:} Manual
					
\textbf{Initial State:} A friend exists in the friend list

\textbf{Input:} Audio/text message to send
					
\textbf{Output:} The corresponding message is sent to the friend

\textbf{Test Case Derivation:} Chatting with friends is the core functionality of expanding social networks. Friends should be able to send and receive messages from each other
					
\textbf{How test will be performed:} The tester will send an audio and text message to another test friend account. Then the tester will verify the other account received the correct message

\textbf{Related Requirement(s):} FR-3-3, FR-3-4

\item{\textbf{FRT-SN6}}

\textbf{Name:} Friend Location Sharing

\textbf{Control:} Manual
					
\textbf{Initial State:} A friend exists in the friend list

\textbf{Input:} User Location
					
\textbf{Output:} User location is visible to the corresponding friend

\textbf{Test Case Derivation:} Users shall be able to share their current location with friends to meet in person. It is essential to verify the location is accurate and up-to-date
					
\textbf{How test will be performed:} The tester will share the location with a friend, and then verify the location is visible and accurate (accuracy of GPS\_ACCURACY) from the other test account

\textbf{Related Requirement(s):} FR-3-5
\end{enumerate}

\subsubsection{AR Campus}

This section includes all test cases related to campus navigation with AR technology. This will include the requirements for building recognition (FR-4-1) and demonstration building information (FR-4-2, FR-4-3). Augmented reality provides an immersive user experience and it is the unique selling point of the application.

\begin{enumerate}
\item{\textbf{FRT-AR1}}

\textbf{Name:} Successful Building Recognition

\textbf{Control:} Manual
					
\textbf{Initial State:} User looks at a building on campus

\textbf{Input:} Clear Camera view
					
\textbf{Output:} The building is recognized and its name is given

\textbf{Test Case Derivation:} Building recognition is the essential functionality for AR technology applied in this application. The building should be recognizable from a certain angle when users look at their camera
					
\textbf{How test will be performed:} The tester will walk around a building on campus and look into the camera, verifying the building is recognized and a list of events/lectures is displayed on the screen

\textbf{Related Requirement(s):} FR-4-1, FR-4-2, FR-4-3

\item{\textbf{FRT-AR2}}

\textbf{Name:} Unsuccessful Building Recognition

\textbf{Control:} Manual
					
\textbf{Initial State:} User looks at a building off-campus

\textbf{Input:} Camera view
					
\textbf{Output:} The building is not recognized correctly

\textbf{Test Case Derivation:} Building recognition is the essential functionality for AR technology applied in this application. The building should be recognizable from a certain angle when users look at their camera
					
\textbf{How test will be performed:} The tester will walk around a building that is not on campus and look into the camera, verifying the building is not recognized or the system recognizing it as a school building

\textbf{Related Requirement(s):} FR-4-1, FR-4-2, FR-4-3
\end{enumerate}

\subsubsection{Lectures and Events}

This section includes all test cases related to events and lectures happening on campus. This will include the requirements for events and lectures themselves, users interacting with these properties, and the power of administration accounts (All FR-5 requirements in the SRS document). The test cases ensure that the lectures and events information is accurate and helpful for all users.

\begin{enumerate}
\item{\textbf{FRT-LE1}}

\textbf{Name:} Interest/Disinterest Event

\textbf{Control:} Manual
					
\textbf{Initial State:} An event exists in a specific building

\textbf{Input:} Corresponding user request
					
\textbf{Output:} The event with all necessary information is added/removed from the user's event list

\textbf{Test Case Derivation:} The user should be able to interact with events available on campus and share their activities with friends. Therefore, they should be able to switch between 'interest' and 'disinterest' states for all events 
					
\textbf{How test will be performed:} The tester will pin and unpin the event and verify the event shows up and disappears from the user's interested event list. Then the tester will verify the event has all associated information including club/department, location and time

\textbf{Related Requirement(s):} FR-5-1, FR-5-2, FR-5-7

\item{\textbf{FRT-LE2}}

\textbf{Name:} Pin/unpin Lectures

\textbf{Control:} Manual
					
\textbf{Initial State:} A lecture with all necessary information exists in a specific building

\textbf{Input:} Corresponding user request
					
\textbf{Output:} The lecture is added/removed from the user's lecture list

\textbf{Test Case Derivation:} The user should be able to interact with lectures shown in the app and share their schedule with friends. Therefore, they should be able to switch between 'pinned' and 'unpinned' states for all lectures 
					
\textbf{How test will be performed:} The tester will pin and unpin the lecture and verify the lecture shows up and disappears from the user's pinned lecture list. Then the tester will verify the event has all associated information including instructor, location and time

\textbf{Related Requirement(s):} FR-5-3, FR-5-4, FR-5-8

\item{\textbf{FRT-LE3}}

\textbf{Name:} Administrator Add Events

\textbf{Control:} Manual
					
\textbf{Initial State:} User is logged in as an admin

\textbf{Input:} Sample event with all necessary information
					
\textbf{Output:} The event is posted on the building and is available for all users to pin

\textbf{Test Case Derivation:} Administrators work as a source of truth in this application, therefore they shall be able to create new events for all users
					
\textbf{How test will be performed:} The tester will log in as an administrator and add the sample event. Then the tester will verify the event shown in the corresponding building is available for all users

\textbf{Related Requirement(s):} FR-5-5

\item{\textbf{FRT-LE4}}

\textbf{Name:} Administrator Change Events

\textbf{Control:} Manual
					
\textbf{Initial State:} User is logged in as an admin

\textbf{Input:} New event information and corresponding user request
					
\textbf{Output:} The posted event information is updated

\textbf{Test Case Derivation:} Administrators work as a source of truth in this application, therefore they shall be able to update event information to get users noticed
					
\textbf{How test will be performed:} The tester will log in as an administrator and update the sample event. Then the tester will verify the event shown in the corresponding building is updated for all users

\textbf{Related Requirement(s):} FR-5-5

\item{\textbf{FRT-LE5}}

\textbf{Name:} Administrator Delete Events

\textbf{Control:} Manual
					
\textbf{Initial State:} User is logged in as an admin

\textbf{Input:} Existing event and corresponding user request
					
\textbf{Output:} The event is deleted

\textbf{Test Case Derivation:} Administrators work as a source of truth in this application, therefore they shall be able to clean out-of-date event
					
\textbf{How test will be performed:} The tester will log in as an administrator and delete the sample event. Then the tester will verify the event shown in the corresponding building is deleted for all users 

\textbf{Related Requirement(s):} FR-5-5

\item{\textbf{FRT-LE6}}

\textbf{Name:} Administrator Add Lectures

\textbf{Control:} Manual
					
\textbf{Initial State:} User is logged in as an admin

\textbf{Input:} Sample lecture with all necessary information
					
\textbf{Output:} The lecture is posted on the building and is available for all users to pin

\textbf{Test Case Derivation:} Administrators work as a source of truth in this application, therefore they shall be able to create new lectures for all users when new semesters come
					
\textbf{How test will be performed:} The tester will log in as an administrator and add the sample lecture. Then the tester will verify the lecture shown in the corresponding building is available for all users 

\textbf{Related Requirement(s):} FR-5-6

\item{\textbf{FRT-LE7}}

\textbf{Name:} Administrator Change Lecture

\textbf{Control:} Manual
					
\textbf{Initial State:} User is logged in as an admin

\textbf{Input:} New lecture information and corresponding user request
					
\textbf{Output:} The posted lecture information is updated

\textbf{Test Case Derivation:} Administrators work as a source of truth in this application, therefore they shall be able to update lecture information to get users noticed
					
\textbf{How test will be performed:} The tester will log in as an administrator and update the sample lecture. Then the tester will verify the lecture shown in the corresponding building is updated for all users

\textbf{Related Requirement(s):} FR-5-6

\item{\textbf{FRT-LE8}}

\textbf{Name:} Administrator Delete Lectures

\textbf{Control:} Manual
					
\textbf{Initial State:} User is logged in as an admin

\textbf{Input:} Existing lecture and corresponding user request
					
\textbf{Output:} The lecture is deleted

\textbf{Test Case Derivation:} Administrators work as a source of truth in this application, therefore they shall be able to clean out-of-date lectures after each semester
					
\textbf{How test will be performed:} The tester will log in as an administrator and delete the sample lecture. Then the tester will verify the lecture shown in the corresponding building is deleted for all users

\textbf{Related Requirement(s):} FR-5-6
\end{enumerate}

\subsection{Tests for Nonfunctional Requirements}

The following section includes tests for non-functional requirements defined in the \href{https://github.com/beatlepie/4G06CapstoneProjectTeam2/blob/main/docs/SRS-Volere/SRS.pdf}{SRS} document. Areas of testing follow the subsections of requirements in the previous document, which mainly include Look and Feel, Usability and Humanity, Performance and Security.\\ Usability requirements will be tested by asking users to do a survey, whose content can be found in section \ref{sec:survey}. Most of the tests here are dynamic and will be done manually or automatically, but there are some tests that need non-dynamic testing like peer code review or code walkthrough.

\subsubsection{Look and Feel}

\begin{enumerate}
\item{\textbf{NFRT-LF1}}

\textbf{Name:} Survey for feedback on application layout

\textbf{Type:} Non-functional, Dynamic, Manual
					
\textbf{Initial State:} User has an account

\textbf{Input/Condition:} User is logged into the homepage and pokes around the application
					
\textbf{Output/Result:} Get feedback and verify layout is user-friendly. This test is a pass if average individual score is over MIN\_SCORE in the "User-friendly Layout",\
“Intuitive icons”, "Immediate Visual Response when Clicking" and "Appealing Colour Scheme" categories of the User Experience Survey
					
\textbf{How test will be performed:} A survey will be given to at least
SURVEY\_SAMPLE\_SIZE users where they will give feedback to different elements of the interface. All of the survey takers are selected randomly from McMaster University

\textbf{Related Requirement(s):} LF-A1, LF-A2, LF-S2

\item{\textbf{NFRT-LF2}}

\textbf{Name:} Visual inspection of interface

\textbf{Type:} Non-functional, Dynamic, Manual
					
\textbf{Initial State:} User has the application downloaded on their phone
					
\textbf{Input/Condition:} User has the application open on their phone
					
\textbf{Output/Result:} The test passes if, for all different pages:
\begin{itemize}
  \item The colour scheme is the same
  \item All visual elements on the screen are within the borders of the screen for all screens in the SCREEN\_VIEWPORTS list
\end{itemize}

\textbf{How test will be performed:} The application will be opened on all screens in the SCREEN\_VIEWPORTS list.\
The testers will visually inspect each page and each screen to make sure that:
\begin{itemize}
  \item All visual elements on the screen are within its borders
  \item The colour scheme on each page is the same
\end{itemize}

\textbf{Related Requirement(s):} LF-A3, LF-S1

\end{enumerate}

\subsubsection{Usability and Humanity}

\begin{enumerate}
\item{\textbf{NFRT-UH1}}

\textbf{Name:} Survey for feedback on application layout

\textbf{Type:} Non-functional, Dynamic, Manual
					
\textbf{Initial State:} User has an account

\textbf{Input/Condition:} User is logged into the homepage and pokes around the application
					
\textbf{Output/Result:} Get feedback and verify layout is user-friendly. This test is a pass if average individual score is over MIN\_SCORE in the "Easy to Navigate",\
“Helpful Tutorial” and "No Technical or Software-Specific Language" categories of the User Experience Survey
					
\textbf{How test will be performed:} A survey will be given to at least
SURVEY\_SAMPLE\_SIZE users where they will give feedback to different elements of the interface. All of the survey takers are selected randomly from McMaster University

\textbf{Related Requirement(s):} UH-EOU1, UH-L2, UH-UP1

\item{\textbf{NFRT-UH1}}

\textbf{Name:} Visual inspection of interface

\textbf{Type:} Non-functional, Dynamic, Manual
					
\textbf{Initial State:} User has the application downloaded on their phone
					
\textbf{Input/Condition:} User has the application open on their phone
					
\textbf{Output/Result:} The test passes if:
\begin{itemize}
  \item The name, time, and location of all pinned events and lectures are displayed on a user's personal page
  \item The user is able to update their USER\_PROFILE and avatar
  \item A tutorial explaining the features of the application appears on first launch, and is available upon request from the user
\end{itemize}

\textbf{How test will be performed:} The application will be opened on all screens in the SCREEN\_VIEWPORTS list.\
The testers will visually inspect the application to check that the above conditions are satisfied.

\textbf{Related Requirement(s):} UH-PI1, UH-PI2, UH-L1

\item{\textbf{test-id}}

\textbf{Name:}

\textbf{Type:} Non-functional, Static, Dynamic, Manual, Automatic, etc.
					
\textbf{Initial State:}

\textbf{Input/Condition:}
					
\textbf{Output/Result:}

\textbf{How test will be performed:}

\textbf{Related Requirement(s):}

\end{enumerate}

...

\subsection{Traceability Between Test Cases and Requirements}

\wss{Provide a table that shows which test cases are supporting which
  requirements.}

\section{Unit Test Description}

\wss{This section should not be filled in until after the MIS (detailed design
  document) has been completed.}

\wss{Reference your MIS (detailed design document) and explain your overall
philosophy for test case selection.}  

\wss{To save space and time, it may be an option to provide less detail in this section.  
For the unit tests you can potentially layout your testing strategy here.  That is, you 
can explain how tests will be selected for each module.  For instance, your test building 
approach could be test cases for each access program, including one test for normal behaviour 
and as many tests as needed for edge cases.  Rather than create the details of the input 
and output here, you could point to the unit testing code.  For this to work, you code 
needs to be well-documented, with meaningful names for all of the tests.}

\subsection{Unit Testing Scope}

\wss{What modules are outside of the scope.  If there are modules that are
  developed by someone else, then you would say here if you aren't planning on
  verifying them.  There may also be modules that are part of your software, but
  have a lower priority for verification than others.  If this is the case,
  explain your rationale for the ranking of module importance.}

\subsection{Tests for Functional Requirements}

\wss{Most of the verification will be through automated unit testing.  If
  appropriate specific modules can be verified by a non-testing based
  technique.  That can also be documented in this section.}

\subsubsection{Module 1}

\wss{Include a blurb here to explain why the subsections below cover the module.
  References to the MIS would be good.  You will want tests from a black box
  perspective and from a white box perspective.  Explain to the reader how the
  tests were selected.}

\begin{enumerate}

\item{test-id1\\}

Type: \wss{Functional, Dynamic, Manual, Automatic, Static etc. Most will
  be automatic}
					
Initial State: 
					
Input: 
					
Output: \wss{The expected result for the given inputs}

Test Case Derivation: \wss{Justify the expected value given in the Output field}

How test will be performed: 
					
\item{test-id2\\}

Type: \wss{Functional, Dynamic, Manual, Automatic, Static etc. Most will
  be automatic}
					
Initial State: 
					
Input: 
					
Output: \wss{The expected result for the given inputs}

Test Case Derivation: \wss{Justify the expected value given in the Output field}

How test will be performed: 

\item{...\\}
    
\end{enumerate}

\subsubsection{Module 2}

...

\subsection{Tests for Nonfunctional Requirements}

\wss{If there is a module that needs to be independently assessed for
  performance, those test cases can go here.  In some projects, planning for
  nonfunctional tests of units will not be that relevant.}

\wss{These tests may involve collecting performance data from previously
  mentioned functional tests.}

\subsubsection{Module ?}
		
\begin{enumerate}

\item{test-id1\\}

Type: \wss{Functional, Dynamic, Manual, Automatic, Static etc. Most will
  be automatic}
					
Initial State: 
					
Input/Condition: 
					
Output/Result: 
					
How test will be performed: 
					
\item{test-id2\\}

Type: Functional, Dynamic, Manual, Static etc.
					
Initial State: 
					
Input: 
					
Output: 
					
How test will be performed: 

\end{enumerate}

\subsubsection{Module ?}

...

\subsection{Traceability Between Test Cases and Modules}

\wss{Provide evidence that all of the modules have been considered.}
				
\bibliographystyle{plainnat}

\bibliography{../../refs/References}

\newpage

\section{Appendix}

This is where you can place additional information.

\subsection{Symbolic Parameters}

The definition of the test cases will call for SYMBOLIC\_CONSTANTS.
Their values are defined in this section for easy maintenance.

\begin{table}[h]
\caption{\bf Symbolic Parameter Table}
\begin{tabular}{|p{0.4\linewidth} | p{0.3\linewidth}| p{0.3\linewidth} |}
\hline
\multicolumn{1}{|l}{\bfseries Symbolic Parameter} & \multicolumn{1}{|l|}{\bfseries Description} & \multicolumn{1}{l|}{\bfseries Value}\\
\hline
INITIAL\_USER\_STATE & The default user state with all default user information & All entries empty except username and password \\
\hline
INITIAL\_AVATAR & The default virtual avatar & An unisex virtual avatar with default settings \\
\hline
MIN\_SCORE & The passing grade for a category in the survey & 7/10\\
\hline
SURVEY\_SAMPLE\_SIZE & Size of the user experience survey & 50\\
\hline
GPS\_ACCURACY & Accuracy of location sharing & 25 meter\\
\hline
SCREEN\_VIEWPORTS & List of all popular mobile screen sizes & 360X740, 390X844, 820X1180\\
\hline
USER\_PROFILE & All information about the user & username, password, gender, age, email, area of study\\
\hline
\end{tabular}
\end{table}

\subsection{Usability Survey Questions?}
\label{sec:survey}

\wss{This is a section that would be appropriate for some projects.}

\newpage{}
\section*{Appendix --- Reflection}

The information in this section will be used to evaluate the team members on the
graduate attribute of Lifelong Learning.  Please answer the following questions:

\newpage{}
\section*{Appendix --- Reflection}

\wss{This section is not required for CAS 741}

The information in this section will be used to evaluate the team members on the
graduate attribute of Lifelong Learning.  Please answer the following questions:

\begin{enumerate}
  \item What knowledge and skills will the team collectively need to acquire to
  successfully complete the verification and validation of your project?
  Examples of possible knowledge and skills include dynamic testing knowledge,
  static testing knowledge, specific tool usage etc.  You should look to
  identify at least one item for each team member.
  \item For each of the knowledge areas and skills identified in the previous
  question, what are at least two approaches to acquiring the knowledge or
  mastering the skill?  Of the identified approaches, which will each team
  member pursue, and why did they make this choice?
\end{enumerate}

\end{document}