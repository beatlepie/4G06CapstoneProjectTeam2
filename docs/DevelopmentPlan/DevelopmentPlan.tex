\documentclass{article}

\usepackage{booktabs}
\usepackage{tabularx}

\title{Development Plan\\\progname}

\author{\authname}

\date{}

\input{../Comments}
%% Common Parts

\newcommand{\progname}{Campus Connections} % PUT YOUR PROGRAM NAME HERE
\newcommand{\authname}{Team \#2, Campus Connections
\\ Waseef Nayeem
\\ Zihao Du 
\\ Matthew Miller
\\ Firas Elayan
\\ Abhiram Neelamraju
\\ Michael Kim} % AUTHOR NAMES                  

\usepackage{hyperref}
    \hypersetup{colorlinks=true, linkcolor=blue, citecolor=blue, filecolor=blue,
                urlcolor=blue, unicode=false}
    \urlstyle{same}
                                


\begin{document}

\maketitle

\begin{table}[hp]
\caption{Revision History} \label{TblRevisionHistory}
\begin{tabularx}{\textwidth}{llX}
\toprule
\textbf{Date} & \textbf{Developer(s)} & \textbf{Change}\\
\midrule
Date1 & Name(s) & Description of changes\\
Date2 & Name(s) & Description of changes\\
... & ... & ...\\
\bottomrule
\end{tabularx}
\end{table}

\wss{Put your introductory blurb here.}

\section{Team Meeting Plan}

\section{Team Communication Plan}

\section{Team Member Roles}

\section{Workflow Plan}

The following steps will be used for workflow:
\begin{enumerate}
	\item Pull new changes from the main branch
	\item Create a new branch from the main branch to develop on
    \begin{itemize}
        \item \textbf{NOTE:} Never develop on the main branch
        \item \textbf{Branch naming convention:} Branch Name should start with the following descriptive keywords: \textbf{chores/docs/feat/fix}, followed by brief descriptions linked with \textbf{-}
        \item \textbf{Branch naming example:} \textit{chores-update-readme}
    \end{itemize}
	\item Develop new code, as well as tests for that code if applicable
	\item Commit new code with descriptive messages
        \begin{itemize}
        \item \textbf{Commit message convention:} Commit message should start with the following descriptive keywords: \textbf{chores/docs/feat/fix}, followed related issue number, wrapped with \textbf{():}. After that, write a brief description of the commit
        \item \textbf{Commit Message example:} \textit{chores(\#2): update readme}
        \end{itemize}
	\item Push those changes to the branch you created
	\item Merge the branch you created into the main branch via a pull/merge request
    \begin{itemize}
        \item \textbf{NOTE:} Make sure all the tests pass before merging \\
    \end{itemize}
\end{enumerate}

The following tags will be used:
\begin{itemize}
    \item \textbf{rev0:} Use for revision 0 of the project
    \item \textbf{rev1:} Use for revision 1 of the project \\
\end{itemize}

The following labels will be used for issue tracking:
\begin{itemize}
    \item \textbf{documentation:} Use when documentation needs to be updated
    \item \textbf{meeting:} Use when meeting minutes need to be added
    \item \textbf{question:} Use when more information is needed
    \item \textbf{enhancement:} Use when a new feature or request needs to be added
    \item \textbf{bug:} Use when there is a bug in the code that needs to be fixed
    \item \textbf{invalid:} Use when there is something that needs improvement
    \item \textbf{help wanted:} Use when help is wanted on a piece of code
    \item \textbf{duplicate:} Use when there is another identical issue or pull request
    \item \textbf{other:} Use when the issue does not match any of the other labels
\end{itemize}

\section{Proof of Concept Demonstration Plan}

What is the main risk, or risks, for the success of your project?  What will you
demonstrate during your proof of concept demonstration to convince yourself that
you will be able to overcome this risk?

\section{Technology}

\begin{itemize}
\item Specific programming language
\item Specific linter tool (if appropriate)
\item Specific unit testing framework
\item Investigation of code coverage measuring tools
\item Specific plans for Continuous Integration (CI), or an explanation that CI
  is not being done
\item Specific performance measuring tools (like Valgrind), if
  appropriate
\item Libraries you will likely be using?
\item Tools you will likely be using?
\end{itemize}

\section{Coding Standard}

\section{Project Scheduling}

\wss{How will the project be scheduled?}

\end{document}